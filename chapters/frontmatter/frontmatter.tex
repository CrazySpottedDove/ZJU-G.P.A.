	\chapter{前言}
	在普物(H)的学习中,深感几个问题:
	\begin{Itemize}
		\item 英文好难,不想看ppt
		\item 老师讲得好抽象(我的问题,老师其实讲得很好了),听不懂
		\item 作业好烦,看不懂也找不到答案
		\item 小测好杂,内容好广
		\item 资料好少,复习好难
		\item 考试好可怕,还要读英文题目
	\end{Itemize}
	
	因此,我产生了一个念头:编写一份中英混搭的普物资料,把ppt和其它教科书全都迭代掉。这个想法可能有些大胆,但我依旧相信,我们可以做到。
	
	感谢普物H笔记计划的所有人\footnote{此处无先后顺序:陈锦浩,倪晟翔,陈若轩,刘远鉴,薛宇航,杨宏毅。},感谢大家的热情加入与支持。
	
	另外,感谢学长薛宇航及其朋友提供的普物笔记,没有他们提供的资料,这份文档的出生将会走更长更长的路。
	
	该笔记的内容基本基于2023学年春夏学期路欣老师的ppt,不保证之后普物(H)教学内容是否有所调整,请使用者自行斟酌。另外,习题部分也取自2023学年春夏学期的课后作业与小测,并对部分题目做了改编。
	
	本书中用到的图片,如果是非矢量图的,且未说明出处的,则来自ppt;其它非矢量图则都注明了出处;矢量图均使用tikz绘制。
	\chapter{约定}
	不要跳过一本书的约定,因为它能帮助你更好地阅读学习。
	
	本书约定如下:
	\setcounter{chapter}{-1}
	\refstepcounter{chapter}
	
	\begin{description}
		\item[语言风格]中英混搭。可能一个中英混搭的文档看起来不伦不类,但我希望它能让还没那么习惯英文的学生能更好地度过这一段过渡期。本文档内,\textbf{解释说明}的语句将采用\textbf{中文},而对于\textbf{问题的叙述},以及一些\textbf{关键词},将采用\textbf{英文},以保证大家的英语阅读思维能得到培养,不至于面对作业题与考试题束手无措。
		\item[交互图层] 很多关键词往往头一次见不是那么容易认得,因此,本文档提供交互图层:我们默认,当文字的颜色显示为{\color{plaincyan}青色}时,它将是可以交互的(这个除外),读者可以通过在PDF阅读器中打开并用鼠标点击的方式切换它的中英文显示。当然,一些并非关键词,但初见可能不认识的词,也会用该方式处理,便于读者查找中文意思。
		\item[超链接]为了便于阅读,在文档中将出现许多的超链接。我们默认,当文字的颜色显示为{\color{blue}蓝色}时(这个也除外),它将可以作为超链接被点击。
		\item[跳转卡片]很多理科书目的前后引用在实际阅读中其实是一件麻烦的事情,本书将提供可以反向跳转的的超链接跳转卡片(如右)\labelroot{frontmatter:ref},读者可以通过在PDF阅读器中打开并用鼠标点击的方式点击跳转(\refleaftext{frontmatter:ref})。
		\item[*标记]普物\Romannumeral{1}中的部分内容较为困难,在实际考核中较少涉及。本书中部分内容用*标记,表示这些内容只需了解即可,不必掌握证明。
		\item [矢量与标量]本书中,不加粗的符号(如$v$)表示标量或取矢量的大小;加粗的符号(如$\vec{v}$)则表示矢量。
		\item [建议的阅读方式] 本书因为叙述要求,在内容顺序上与课堂可能有一定出入。如果决定使用本书学习普物(H),请确保学习的连续性,不要碎片式学习,以免在写作业时遇到困难。
		
		对于初学者,建议兼顾笔记的正文部分和证明部分;若是补天选手,如无特别说明,阅读正文部分即可。本书的习题部分都有详细的解答,可以自行练习。
	\end{description}
