\newpage
\section{课后习题:狭义相对论}
\begin{example}[Fizeau effect---\refleaftext{solution6.1}]
    In this problem we show that the relativistic velocity addition law can be used to
    explain the Fizeau experiment without invoking the existence of \itr{ether}{以太}. The speed of light in stationary water is less than its speed $c$ in \itr{vacuum}{真空}.
    Traditionally it is written as $\frac{c}{n}$, where $n\approx \frac{4}{3}$ is the \itr{index of refraction}{折射率} of water. The water flowed in the \itr{pipe}{管道} with
    velocity $v$. In the lower arm $T_2$ of the \itr{interferometer}{干涉仪} (as shown in the figure), one would expect that, from the nonrelativistic addition law, the
    speed of light in the moving water would be its speed in stationary
    water increased by the speed of the water in the pipe $w=\frac{c}{n}+v$. Show
    that the relativistic velocity addition law leads to, up to higher-order
    corrections:
    \begin{equation*}
    	w=\frac cn+v\left(1-\frac1{n^2}\right)
    \end{equation*}
    The result was observed by Fizeau in 1851, but for long time viewed as
    a confirmation of a rather \itr{elaborate}{复杂的} \itr{contemporary}{同时代的} ether-theoretical
    calculation based on the idea that the water was partially successful in
    dragging ether along with it. Einstein later said that it was of
    fundamental importance in his thinking.
\begin{singlefigure}{chapter_6_7}[0.6]  
\end{singlefigure}
\end{example}
\begin{example}[{\large\itr{Terrel Rotation}{特雷尔旋转}}---\refleaftext{solution6.2}]
	\ctikzfig{chapter6_solution_6_2_1}
	
    A square with proper-length-$L$ sides flies past you at a speed $v$, in a direction parallel to two of its sides. You stand in the plane of the square. When you see the square at its nearest point to you, show that it looks to you like it is rotated, instead of contracted. (Assume that $L$ is small compared with the distance between you and the square.)
\end{example}
\begin{example}[Lots of transformations---\refleafmark{solution6.3}]
A train with proper length $L$ moves at speed $\dfrac{c}{2}$ with respect to the ground. A ball is thrown from the back to the front, at speed $\dfrac{c}{3}$ with respect to the train. How much time does this take, and what distance does the ball cover, in:
\\(a) The train frame?
\\(b)The ground frame? Solve this by:
\\\hspace*{2em}i. Using a velocity-addition argument.
\\\hspace*{2em}ii. Using the Lorentz transformations to go from the train
frame to the ground frame.
\\(c)The ball frame?
\\(d) Verify that the \itr{invariant interval}{不变(时空)间隔} is indeed the same in all three
frames.
\end{example}
\begin{example}[Train And \itr{Tunnel}{隧道} \itr{Paradox}{悖论}---\refleaftext{solution6.4}]
Consider a train running at a constant speed $V$ on the straight track in the $x$ direction, and passing through a tunnel (see the figure). The proper length of the train is $L$, and the proper length of the tunnel is $D$. Here we assume $L>D$. Define $(x,ct)$ as the time and the space coordinates of the track frame, and $(x',ct')$ as those of the train frame. Here, $x$ and $x' $\textbf{ are in the same direction}.
\begin{singlefigure}{chapter_6_20}[1]    
\end{singlefigure}
(a) Suppose that an observer standing on the ground sees that the train is shorter than the tunnel, so that the whole train can be inside the tunnel. Determine the smallest possible speed of the train.

(b) Suppose that the rear end of the tunnel (see the figure) is at $x=0$, and set the time $t=t'=0$ when the rear end of the train reaches the rear end of the tunnel. Draw the Minkowski diagram \textbf{ taking $x$ coordinate for the horizontal axis and $ct$ coordinate for the vertical axis.} In addition, \itr{specify}{指明} $L$ and $D$ in the diagram.

(c) When the rear end of the train enters the rear end of the tunnel, the rear-end and front-end sliding doors of the tunnel (see the figure) are closed at the same time in the track frame. These two events are denoted by $R_{close}$ and $F_{close}$, respectevely. Then, when the front-end of the train reaches the front end of the tunnel, both the rear-end and front-end sliding doors are opened at the same time in the track frame. These events are denoted by $R_{open}$ and $F_{open}$, respectively.

Show the events $R_{close},F_{close},R_{open}$, and $F_{open}$ in the Minkowski diagram in (b),
and put the four events in the order of being seen by an observer in the train.
\end{example}
\begin{example}[Conservation of momentum in SR---\refleaftext{solution6.5}]
    In class, we showed that the classical definition of the linear
    momentum cannot be right in the relativistic case. We illustrated by the
    example of the collision of two particles with equal mass $m$. In the rest
    frame (for the center of mass) $K$, the two particles have velocities with
    the same amplitude $v$ but opposite directions along $x$ axis before the
    collision, as illustrated in Fig.(a). After the collision, they move away along $y$ axis with the same speed $v$, as illustrated in Fig.(b). \\Now, in a frame $K'$ that moves with speed $v$ along the positive $x$ direction with respect to the rest frame $K$, as illustrated in Fig.(c), one particle is at rest before the
    collision. 
    \\(i) What is the velocity of the other particle before the
    collision? 
    \\(ii) After the collision, as illustrated in Fig.(d), what are the velocities of
    the two particles? Specify the components of the velocities along $x$ and
    $y$ axes.
    \\(iii) Show that if you use the definition of the relativistic
    momentum, you will maintain the conservation of linear momentum in
    the moving frame $K'$.
	\begin{singlefigure}{chapter_6_23}[0.6]   
	\end{singlefigure}
\end{example}
\begin{example}[Perfectly Inelastic Collison of two Relastivistic Particles---\refleaftext{solution6.6}]
    Consider a perfectly inelastic collision of two relativistic particles $A$ and $B$ with equal rest mass $m$. In the center-of-mass frame $K$, the two particles have velocities with the same magnitude $v$, but opposite directions along $x$ axis before the collision, as illustrated in the top left part of the figure. After the collision, they stick together, as illustrated in the bottom left part of the figure. Now, in another frame $K'$ that moves with speed $v$ along the negative $x$ direction with respect to the $K$-frame (right part of the figure), particle $A$ is at rest before the collision.
    \\(a) Considering the energy conservation for the collision in $K$-frame, calculate the
    rest mass $M$ of each particle after the collision.
    \\(b) In $K'$-frame, what is the velocity $u$ of particle $B$ before the collision?
    \\(c) In $K'$-frame, show that the linear momentum and energy are conserved in the collision process.
    \begin{singlefigure}{chapter_6_24}[0.6]    
    \end{singlefigure}
\end{example}
\begin{example}[Relastivistic Scattering between a Photon and an Electron---\refleaftext{solution6.7}]
    In this problem, a particular scattering process between a photon and an electron known as \itr{Compton Scattering}{康普顿散射} will be addressed. For simplicity, we will consider only one spatial dimension so that spatial vectors $\vec{a}=a\vec{e}_x$  \itr{possess}{拥有} only one non-zero component and where $\vec{e}_x$ is the unit vector along the $x$ axis. In this setting, a photon of energy $E_\mathrm{ph}$ is \itr{propagating}{传播} along the $x$ axis and hits an electron. We want to understand with which energy the photon is scattered back along the $x$ axis in terms of the initial parameters. Let $c$ be the speed of light.
\\(a) As a first step, write down
\\\hspace*{2em}(i) the relativistic expressions for the energy $E$ and momentum $\vec{p}$ of a particle of mass $m$ and velocity $\vec{v} = v\vec{e}_x$.
\\\hspace*{2em}(ii) the expressions for $\dfrac{E'}{c}$ and $\vec{p}'$ in terms of $\dfrac{E}{c}$ and $\vec{p}$ in an inertial frame that moves with
velocity $\vec{u}=u\vec{e}_{x}$ relative to the one where the particle has energy $\dfrac{E}{c}$ and momentum $\vec{p}$.\\
Hint: the Lorentz transformation of the position four-vector is \[(ct^{\prime},x^{\prime},0,0)=(\gamma ct-\beta\gamma x,\gamma x-\beta \gamma ct, 0, 0), \beta = \dfrac{u}{c}, \gamma = \dfrac{1}{\sqrt {1- \beta ^2}}\]. 
(b) Consider the energy-momentum four-vector $\vec{P}$ defined as $\vec{P}=(\dfrac{E}{c},\vec{p})$.\\
\hspace*{2em}(i) Show that $\vec{P}\cdot\vec{P}$ yields the energy-momentum relation, where $\vec{P}\cdot\vec{P}$ denotes the scalar product of four-vector $\vec{P}$ with itself.\\ Hint: For four-vectors $\vec{A}=(a_0,\vec{a})$ and $\vec{B}=(b_0,\vec{b})$, the scalar product is defined as  \[\vec{A}\cdot\vec{B}=a_{0}b_{0}-\vec{a}\cdot\vec{b}\]
\hspace*{2em}(ii) Show that $\vec{P}'\cdot\vec{P}'=\vec{P}\cdot\vec{P}$ where $\vec{P}'=(\dfrac{E'}{c},\vec{p}')$.\\
\hspace*{2em}(iii) The energy-momentum relation for a photon is that of a particle \itr{of vanishing rest mass}{无静质量}. If $\vec{K} = ( \dfrac{E_{\mathrm{ph}}}{c}, \vec{k} )$ is the energy-momentum four-vector of a photon, what is the value of $\vec{K}\cdot\vec{K}$?\\
(c) Consider now a photon of 4-momentum $\vec{K}$ that is propagating along the $x$ axis and hits an electron with mass $m$ and 4-momentum $\vec{P}$ and is scattered back elastically along the x axis with energy $E_{\mathrm{ph}}^{\mathrm{fi}}.$ What is $E_\mathrm{ph}^\mathrm{fi}$ in terms of $m$ and $E_\mathrm{ph}?$ The result is commonly quoted in terms of $\dfrac{1}{E_\mathrm{ph}^\mathrm{fi}}$.\\
Hint: One way to proceed is to choose the rest frame of the electron before the collision, write down the energy-momentum four-vectors before and after the collision and relate the four-vectors before and after the collision using energy and momentum conservation.
\end{example}