\chapter[质点运动学与动力学基础]{\itr{Fundamentals of particle kinematics and dynamics}{质点运动学与动力学基础}\mgnote*{\raggedleft 作者:23级-刘远鉴}}
\textbf{质点运动},即一个理想化的、质量集中于单一点的物体的运动。尽管现实世界中的物体都有一定的尺寸和形状,但通过将其简化为质点,我们可以更容易地分析和描述它们的运动特性。质点运动学就是研究质点运动的学科。它不关心引起运动的力,而是专注于描述运动本身:如位置、速度、加速度等随时间变化的规律。通过掌握这些基础的运动学概念和方程式,我们可以更好地理解和预测各种自然现象和工程应用中的运动行为。

相信大家在高中的时候都和质点运动学打过交道,那么大学的质点运动学和高中的又有什么不同呢?简而言之,就是加入了微积分这个强大的工具。

了解了运动学相关的知识后,我们将研究学习引起运动的力及其作用,包括功、能、动量等概念,也就是动力学的内容。
\section[一维运动]{\itr{Motion in 1D}{一维运动}}
讲质点运动,首先从一维的运动说起。
\subsection[质点运动学的基本概念]{\itr{Basic Concepts of Particle Kinematics}{质点运动学的基本概念}}
相信大家在高中时候以及对质点运动学的基本概念比较了解了,不多赘述,汇总如下\footnote{此时,我们还没有引入向量的观点,因此这里的符号全都是标量形式。}:
\begin{Itemize}
       \item \itr{Displacement}{位移} $\Delta x=x_{f} -x_{i}$
       \item \itr{Average Velocity}{平均速度} $v_{avg} =\dfrac{x_{f}-x_{i}  }{t_{f}-t_{i}  } $
       \item \itr{Average Speed}{平均速率} $s_{avg} =\dfrac{total\ distance}{total\ time} $ 
       \item \itr{Instantaneous Velocity}{瞬时速度} $\displaystyle v=\lim_{\Delta t \to 0} \frac{\Delta x}{\Delta t} =\frac{\dif  x}{\dif  t} $
\end{Itemize}
\begin{Itemize}
       \item \itr{Instantaneous Speed}{瞬时速率} $\left | v \right | >0$ ,即瞬时速度的大小
       \item \itr{Average Acceleration}{平均加速度} $a_{avg} =\dfrac{v_{f}-v_{i}  }{t_{f}-t_{i}  } $
       \item \itr{Instantaneous Acceleration}{瞬时加速度} $\displaystyle a=\lim_{\Delta t \to 0} \frac{\Delta v}{\Delta t} =\frac{\dif  v}{\dif  t} = \frac{\dif ^2x}{\dif  t^2} $
\end{Itemize}
注意这里的下标,avg代表average,i代表initial,f代表final。做题的时候自己写上这样的下标也是很清晰的。
\subsection[质点运动学有关的计算]{\itr{Calculations Related to Particle Kinematics}{质点运动学有关的计算}}
\eg 已知位移与时间的关系$x=At^{n} $,可求得t时刻的瞬时速度为$v=\dfrac{\dif  x}{\dif  t} =Ant^{n-1} $。进一步也可以求得t时刻的瞬时加速度,留给读者。

那么,上面例子中的过程能否反向进行呢?知道瞬时加速度与时间的关系,能否推出速度、位移与时间的关系呢?下面是一个例子:

已知加速度关于时间的表达式为$a=a(t)$,初始时刻为$t_{i} $,那么由于$a=\dfrac{\dif  v}{\dif  t}$,形式上移项可得 $\dif  v=a(t)\dif  t$。等式两边从初始时刻$t_{i}$到当前时刻$t$积分,可得\mgnote{左右两边的常数项合并了。}
\[v(t)=\int_{t_{i}}^{t}a(t') \dif  t'+C_{1}\]
代入$t=t_{i}$,可求得$C_{i}=v_{i}$。如此,我们便得到了速度关于时间的表达式
\[v(t)=\int_{t_{i}}^{t}a(t')\dif  t'+v_{i}\]

在上面计算的基础上,进一步可以求得位移关于时间的表达式(已知初始位移为$x_{i}$):
\begin{equation}
    \begin{aligned}
    x(t)&=\int_{t_{i}}^{t}v(t')\dif  t'+x_{i}\\
    & =\int_{t_{i}}^{t} \dif  t'\left[\int_{t_{i}}^{t'}a(t'')\dif  t''+v_{i}\right]+x_{i}\\
    & =x_{i}+(t-t_{i})v_{i}+\int_{t_{i}}^{t}\dif  t'\int_{t_{i}}^{t'}\dif  t''a(t'')  
     \end{aligned}
    \nonumber
\end{equation}

至此,我们已经得到了相当具有普遍性的两个公式,尽管它们显得十分复杂。如果加速度是一个常数$a$,那么化简这两个公式,就可以得到大家高中时候倒背如流的几个公式:
\begin{equation}
    \begin{aligned}
    &v(t)=v_{i}+at\\
    &x(t)=x_{i}+v_{i}t+\frac{1}{2}at^{2}\\
    &x(t)-x_{i}= \frac{1}{2} (v_{i}+v_{f})t=\frac{v_{f}^{2}-v_{i}^{2}}{2a} 
    \end{aligned}
    \nonumber
\end{equation}

计算练习:已知$a(t)=at^{\alpha }$,其中$\alpha$是正的常数,$v_{i}$,$x_{i}$都是已知的,试求出$v(t)$,$x(t)$这两个表达式。
\section[高维的质点运动]{\itr{Motion in High Dimensions}{高维的质点运动}}
看起来很高大上的节标题,其实就是引入向量来更好地研究质点运动。

首先让我们来回忆一下,高中时期我们是怎么研究一个小球的平抛运动的。很自然地,我们一般会从竖直方向和水平方向两个方向来分解这个平抛运动,得到两个还算简单的表达式,并认为我们成功描述了这个平抛运动。

但是,如果大家尝试从水平和竖直两个方向去描述一个斜抛运动,或是在一个倾斜的坐标系下去描述一个平抛运动,就会发现,结果是一个带着三角函数的复杂式子。这仅仅只是对于二维的、一个质点的讨论。可以想象,用这种track each component的研究分量的方式讨论更高的维度或是更多的质点,会有更多的麻烦。
\ctikzfig{chapter2_without_vectors}
\begin{center}
	在倾斜的坐标系下描述平抛运动
\end{center}

那么,有什么方法能够让我们避开对于多个分量的繁琐讨论呢?没错,就是 \itr{vector}{向量}。
\subsection[向量]{\itr{Vector}{向量}}
\begin{Itemize}
	\item \itr{vector}{向量}: \En{A \itr{quantity}{量} that has both \itr{direction}{方向} and \itr{magnitude}{大小} and also obeys \itr{the laws of vector addition}{向量加法规律}\mgnote{我们假设读者已经知道向量加法规则,此略。}.}
\end{Itemize}

向量的常见表示方法:$\mathbf{a}$(boldface粗体表示)或者$\vec{a}$。本书中采用后者。

用向量来研究高维质点运动的好处:
\begin{Itemize}
    \item \itr{concise}{精确}:精确而简洁。
    \item \itr{independent of the choice of the coordinate axes}{与坐标系的选取无关}:用向量研究质点运动时,确定坐标系不再是必要的。
\end{Itemize}
虽然说向量与坐标系的选取无关,但我们常常在一个给定的坐标系下去描述一个向量。对于不同的坐标系及不同的\textbf{基底}(base),同一个向量的表示也是不一样的。
如何理解这句话呢?给大家看一个例子\labelroot{chapter2_vector_expression}:

首先,我们从最常见的\textbf{笛卡尔坐标系}(cartesian coordinate)说起。对于向量$\vec{r}$,高中的时候,我们会假设$\vec{r}=(x,y)$ ——其实这里我们省略了基的选取。更加严谨而普遍的表示方式应该是:设$\displaystyle\vec{r} =\left \{ \vec{e }_{1} ,\vec{e }_{2}  \right \} \binom{x}{y} =x\vec{e }_{1}+y\vec{e }_{2}$\footnote{这里的花括号其实就是矩阵的括号,用花括号是为了表示基底的更加清晰(但不严谨)的写法。大家可以将这个表达式理解为一个行向量和列向量的矩阵乘法(向量内积),这样更好记忆。但事实上$\left \{ \vec{e }_{1} ,\vec{e }_{2}  \right \}$应该是一个分块矩阵的表示,因为$\vec{e }_{1}$和$\vec{e }_{2}$本身也是列向量。}。
对于 \itr{Cartesian Coordinate}{笛卡尔坐标系},$\vec{e }_{1}$是$\displaystyle\binom{1}{0} $,$\displaystyle
\vec{e }_{2}$是$\displaystyle\binom{0}{1} $(也就是x轴,y轴方向的单位向量),代入上面假设的表达式,可以得到$\displaystyle\vec{r}=\binom{x}{y}  $。
\begin{center}
\begin{tikzpicture}[scale=0.8]

  \draw[->] (-3,0) -- (3,0) node[right] {$x$}; % x轴
  \draw[->] (0,-3) -- (0,3) node[above] {$y$}; % y轴

  \draw[->, thick, thisblue] (0,0) -- (1,0) node[midway, below] {$\vec{e}_1$}; % e1向量
  \draw[->, thick, thisred] (0,0) -- (0,1) node[midway, right] {$\vec{e}_2$}; % e2向量
\end{tikzpicture}

	\em 笛卡尔坐标系的基底

\end{center}

进一步,我们应当跳脱笛卡尔坐标系,去考量一些更加普遍的情况。在线性代数中,我们学习过线性空间和它的基。对于我们要考虑的n维空间,只要取n个\linebreak\itr{linearly independent}{线性无关} 的$\vec{e_{i} }$就可以了。这个时候,一个向量可能会表示成这个样子:
\[\vec{a} =\left \{ \vec{e_{1} } ,\vec{e_{2} } ,\ldots\vec{e_{n} }\right \} 
\begin{bmatrix}
  a_{1} \\
  a_{2} \\
  ...\\
  a_{n}
\end{bmatrix}\]

\subsection[向量与标量]{\itr{Vector versus Scalar}{向量vs标量}}
为了让大家更好地理解向量,这里区分一组概念:\itr{vertor}{向量} 和 \itr{scalar}{标量}。
\begin{Itemize}
	\item \itr{scalar}{标量} \En{A \itr{quantity}{量} having \itr{magnitude}{大小} but no \itr{direction}{方向} is a \itr{scalar}{标量}.}
\end{Itemize}
常见的标量:T(\itr{temperature}{温度}), m(\itr{mass}{质量}), s(\itr{speed}{速率})\mgnote{注意,向量的大小就是一个标量。}
\subsection[向量的大小与单位向量]{\itr{Magnitude and Unit Vector}{向量的大小与单位向量}}
向量$\vec{A}$的大小被表示为$A$或$\left | \vec{A} \right | $。说两个向量相等,等价于说两个向量大小相等,方向相同。

单位向量就是大小为1个单位的向量。
\begin{Itemize}
	\item \itr{unit vector}{单位向量} \En{A unit vector is a \itr{dimensionless}{无量纲的} vector having a magnitude of exactly 1.}
	\\
	注意,量纲的英文是dimension,题目中出现的时候要注意上下文语义。
\end{Itemize}

欲求一个向量同方向的单位向量,可取$\displaystyle\hat{\vec{r}} =\frac{\vec{r} }{\left | \vec{r} \right | } $。
\subsection[向量的运算]{\itr{Vector Algebra}{向量的运算}}
向量加法(vector addition),大家高中的时候都学过,共起点的时候使用平行四边形法则,首尾相连的时候使用三角形法则。

加法运算满足下面两个规律:
\begin{Itemize}
     \item \itr{Associaive law of addition}{加法结合律}:$\vec{A}+(\vec{B}+\vec{C})=(\vec{A}+\vec{B})+\vec{C}$
     \item \itr{Commutative law of addition}{加法交换律}:$\vec{A}+\vec{B}=\vec{B}+\vec{A}$
\end{Itemize}
这两个规律的证明,除了图像几何证明之外,还可以用上面我们提到的向量的表示方法\mgnote{见\refleaftext{chapter2_vector_expression}}来证明,推荐大家自己尝试一下。尝试之后可以发现,向量的加法交换律和结合律,都是基于实数的加法交换律和结合律。

一个向量$\vec{A}$的负向量(negative) $-\vec{A}$定义为与$\vec{A}$的向量和为$\vec{0}$(零向量)的向量,也就是与原向量等大反向的向量。

向量的减法(vector subtraction)可以用向量的加法与负向量定义:
\[\vec{A}-\vec{B}=\vec{A}+(-\vec{B})\]

在新的视角下,向量的数乘也可以这样理解($\alpha$是参与数乘的标量):
\[\alpha \vec{a}=\alpha \sum_{i=1}^{n}  a_{i} \vec{e }_{i} =\sum_{i=1}^{n} (\alpha a_{i} )\vec{e }_{i}\]         
          
向量的内积(scalar product)\labelroot{chapter2_scalar_product}:
\[\vec{a} \cdot\vec{b}=(\sum_{i}^{}a_{i}\vec{e}_{i}  )(\sum_{j}^{}b_{j}\vec{e}_{j}  ) =\sum_{i}^{}\sum_{j}^{}a_{i}b_{j}(\vec{e}_{i}\cdot\vec{e}_{j})  \]
特别的,对于 \itr{cartesian coordinate}{笛卡尔坐标系}\mgnote{这里拓展到多维情形。},有\mgnote{$\delta$念作delta。}
\[\vec{e}_{i}\cdot\vec{e}_{j}=\delta _{ij}=\begin{cases}
    1 & \text{ if } i=j \\
     0& \text{ if } i\ne j
   \end{cases}\]
于是有
\[\vec{a} \cdot \vec{b}=\sum_{i}^{} \sum_{j}^{} a_{i} b_{j} \delta _{ij} =\begin{bmatrix}
    a_1 & a_2 & \cdots & a_n
  \end{bmatrix}
  \begin{bmatrix}
    b_1 \\
    b_2 \\
    \vdots \\
    b_n
  \end{bmatrix}\]

向量内积满足下面几个基本律:
\begin{Itemize}
    \item \itr{Commutative Law}{交换律} 
    $\vec{a} \cdot \vec{b} = \vec{b} \cdot \vec{a}$
    \item \itr{Distributive Law}{分配律} 
    $\vec{a} \cdot (\vec{b} + \vec{c}) = \vec{a} \cdot \vec{b} + \vec{a} \cdot \vec{c}$
    \item \itr{Scalar Multiplication Property}{数乘律} 
    $k(\vec{a} \cdot \vec{b}) = (k\vec{a}) \cdot \vec{b} = \vec{a} \cdot (k\vec{b})$,其中 $k$ 是实数。
\end{Itemize}

向量的模长可以用内积表示:$r=\left | \vec{r} \right | =\sqrt{\vec{r}\cdot\vec{r}} $。

\subsection[极坐标]{\itr{Polar Coordinate}{极坐标}\labelroot[16pt]{chapter2_polar_coordinate}}
二维的情况: $\begin{cases}
    x=r\cos \theta \\
    y=r\sin \theta
   \end{cases}$
\begin{center}
\begin{tikzpicture}[scale=1]
    % 绘制坐标轴
    \draw[->] (0,0) -- (3,0) node[right] {$x$}; % x轴
    \draw[->] (0,0) -- (0,3) node[above] {$y$}; % y轴

    % 绘制极坐标点
    \draw[->, thick] (0,0) -- (1.5,1.5) node[above] {$P(r,\theta)$};
    \node[left] (0,0) {$O$};
    % 角度标记
    \draw[->] (0.5,0) arc (0:45:0.5) node[midway, right] {$\theta$};
    
    % r的长度标记
    \draw[dotted] (0,0) -- (1.5,1.5) node[midway, right] {$r$};
\end{tikzpicture}

\text{二维极坐标}
\end{center}

二维极坐标系的基本要素是极点和极轴,其中,极轴是一条从极点引出的有向轴。在上图中,极点即为$O$,而极轴则与$x$轴方向相同。我们用$(|\vec{r}|,\theta)$来表示一个点的位置。其中,$\vec{r}$表示以极点为起点,以被描述点为终点的向量,称为极径矢量;$\theta$则表示从极轴出发逆时针旋转到极径矢量的角度。可以发现,$\theta$并不唯一,我们一般取$2\pi$以内的角。

注意极坐标系同样也是有基底的,而且也是正交的。在极坐标中,特别定义$\hat{\vec{r}}$为$\vec{r}$增加方向的单位矢量,$\hat{\vec{\theta}}$为$\theta$增加方向的单位矢量,两者构成一对单位正交基。

\ctikzfig{chapter2_polar_coordinate_2d}
\begin{center}
	二维极坐标系的基底
\end{center}

三维的情况:$\begin{cases}
    x=r\sin\theta \cos\varphi \\
    y=r\sin\theta \sin\varphi \\
    z=r\cos\theta
   \end{cases}$
\begin{center}
	\resizebox{15em}{30ex}{
		\begin{tikzpicture}
			%draw the axes
			\draw[->] (0,0,0) -- (2.5,0,0) node[anchor=west]{$x$};
			\draw[->] (0,0,0) -- (0,2.5,0) node[anchor=west]{$z$};
			\draw[->] (0,0,0) -- (0,0,2.5) node[anchor=west]{$y$};
			%draw the top and bottom of the cube
			\draw[dotted,gray!50!thiswhite] (0,2,0) -- (2,2,0) -- (2,0,0);
			\draw[dotted,gray!50!thiswhite] (2,0,2) -- (0,0,2) -- (0,2,2) -- (2,2,2);
			
			%draw the edges of the cube
			%\draw[dashed,gray] (0,0,0) -- (0,0,2);
			\draw[dotted,gray!50!thiswhite] (0,2,0) -- (0,2,2);
			\draw[dashed,thisblack] (2,0,0) -- (2,0,2);
			\draw[dotted,gray!30!thiswhite] (2,2,0) -- (2,2,2);
			
			%  
			\draw[thick,thisblue!50!black,->] (0,0,0) -- (2,2,2) node[anchor=west,above=-14pt,right=-14pt]{$\boldsymbol{r}$};
			\draw[dashed,black] (0,0,0) -- (2,0,2);
			%\draw[dashed,black] (0,2,0) -- (2,2,2);
			\draw[dashed,black] (2,2,2) -- (2,0,2);
			
			% arcs
			%\coordinate (O) at (0,0,0);
			\tdplotdrawarc[<-,rotate around x=90,thisgreen!50!black]{(0,0,0)}{0.5}{0}{45}{above=-1pt,right=0pt,anchor=mid west}{$\varphi$};%\tdplotdrawarc[]{(O)}{r}{theta}{theta+Delta_theta}
			\tdplotdrawarc[<-,rotate around y=-45,thisred!80]{(0,0,0)}{0.5}{35.264}{90}{right=9pt,above=4pt,anchor=mid east}{$\theta$}
			\node[above=2pt, right=1pt,thisblack!70,scale=0.8] at (2,2,2) {$(x,y,z)$};
		\end{tikzpicture}
	}
	
	三维极坐标(来自 \url{https://www.latexstudio.net/index/details/index/mid/2316 })
\end{center}

\subsection[基变换]{\itr{Basis Transformation}{基变换}}
我们考虑一个简单的坐标系旋转的变换:
\ctikzfig{chapter2_basis_transformation}
\[\begin{cases}
	x'=OD+DF=x\cos\theta+y\sin\theta\\
	y'=PC-FC=y\cos\theta-x\sin\theta
\end{cases}\]

两个式子可以用一个矩阵乘法的式子表示:
\[\left [ x',y' \right ] =\left [ x,y \right ]\begin{bmatrix}
  \cos\theta &-\sin\theta \\
   \sin\theta&\cos\theta
 \end{bmatrix}\]

对于原坐标系中的一点$(x,y)$,只需要乘上基变换矩阵$\begin{bmatrix}
  \cos\theta &-\sin\theta \\
   \sin\theta&\cos\theta
 \end{bmatrix}$,就能得到该点在新坐标系下的坐标$(x',y')$。

 \subsection[质点运动的向量表示法]{\itr{Vector Representation of Particle Motion}{质点运动的向量表示}}
 无论是一维质点运动还是高维质点运动,都可以统一用向量来表示。
 \begin{Itemize}
  \item \itr{Position}{位置} $\vec{r}$
  \item \itr{Displacement}{位移} $\Delta\vec{r}=\vec{r}_{f}-\vec{r}_{i}$
  \item \itr{Average Velocity}{平均速度} $\vec{v}_{avg}=\dfrac{\Delta \vec{r}}{\Delta t}$
  \item \itr{Instantaneous Velocity}{瞬时速度} $\displaystyle\vec{v}=\lim_{\Delta t \to 0} \frac{\Delta \vec{r}}{\Delta t}=\frac{\dif \vec{r}}{\dif  t}$
  \item \itr{Average acceleration}{平均加速度} $\vec{a}_{avg}=\dfrac{\vec{v}_{f}-\vec{v}_{i}}{t_{f}-t_{i}}$
  \item \itr{Instantaneous acceleration}{瞬时加速度} $\displaystyle\vec{a}=\lim_{\Delta t \to 0}\frac{\Delta \vec{v}}{\Delta t}=\frac{\dif \vec{v}}{\dif  t}$
\end{Itemize}

而且向量表示也为我们对于基底以及分量的分析提供了便利:
\begin{equation}
  \begin{aligned}
    &\vec{r} =x\hat{\vec{i}}\mgnote{$\hat{\vec{i}}$的上标符号用于表示基底。} +y\hat{\vec{j}} +z\hat{\vec{k}} \\
\Longrightarrow &\vec{v}=\frac{\dif \vec{r}}{\dif  t}=\frac{\dif  x}{\dif  t}\hat{\vec{i}}+\frac{\dif  y}{\dif  t}\hat{\vec{j}}+\frac{\dif  z}{\dif  t}\hat{\vec{k}}=v_{x}\hat{\vec{i}}+v_{y}\hat{\vec{j}}+v_{z}\hat{\vec{k}}\\
   \end{aligned}
  \nonumber
\end{equation}

类似的,对于加速度也有:
\begin{equation}
  \begin{aligned}
   \vec{a}=a_{x}\hat{\vec{i}}+a_{y}\hat{\vec{j}}+a_{z}\hat{\vec{k}}
   \end{aligned}
  \nonumber
\end{equation}

\subsection[抛体运动]{\itr{Projectile Motion}{抛体运动}}
\begin{center}
  \begin{tikzpicture}[scale=1.5]

      % 轴
      \draw[->] (0,0) -- (4,0) node[right] {$x$};
      \draw[->] (0,0) -- (0,3) node[above] {$y$};
      
      % 斜抛轨迹
      \draw[thick, thisblue, domain=0:2.5] plot (\x, -\x^2+\x*2);

      % 初始速度矢量
      \draw[->, thick, thisred] (0,0) -- (1,2) node[right] {$\vec{v_0}$};

      % 角度标记
      \draw[->] (0,0) -- (0.5,0) arc (0:40:0.9) node[midway, right] {$\phi_0$};

      % 标注初始点
      \node at (0, -0.2) {O};

  \end{tikzpicture}
\end{center}

普遍的,对于一个匀加速运动,有$\vec{r}=\vec{r}_{0}+\vec{v}_{0}t+\dfrac{1}{2}\vec{a}t^{2}$。

\vspace*{0.4ex}
对于上图所示的简单的抛体运动,我们这样计算物体重新落到$x$轴上时,距离开始的时间和距离:
\begin{equation}
  \begin{aligned}
   \vec{r}_{f}&=s\hat{\vec{i}}+0\hat{\vec{j}}\\
   \Rightarrow\begin{pmatrix}
    s\\
    0
   \end{pmatrix}&=\begin{pmatrix}
                   v_{0}\cos\phi_{0}t\\
                   -v_{0}\sin\phi_{0}t 
                 \end{pmatrix}+\begin{pmatrix}
                                            0\\
                                            -\dfrac{1}{2}gt^{2}
                                          \end{pmatrix}\\
   \Longrightarrow t&=\frac{2v_{0}\sin\phi_{0}}{g}\\
   \Longrightarrow s&=\frac{v_{0}^{2}\sin2\phi_{0}}{g}
  \end{aligned}
  \nonumber
\end{equation}

再来看一个射击自由落体的靶子的例子:
\begin{center}
  \begin{tikzpicture}[scale=1.5]

      % 轴
      \draw[->] (0,0) -- (4,0) node[right] {$x$};
      \draw[->] (0,0) -- (0,3) node[above] {$y$};
      
      % 斜抛轨迹
      \draw[thick, thisblue, domain=0:2] plot (\x, -\x^2+\x*2);

      % 初始速度矢量
      \draw[->, thisblack] (0,0) -- (1,2) node[right] {T target};
      \draw[->, thick, thisred] (0,0) -- (0.5,1) node[left] {$\vec{v}_0$};
      \draw[->,thisblue] (1,2) -- (1,1) node[above right] {hit};
      % 角度标记
      \draw[->] (0,0) -- (0.5,0) arc (0:40:0.9) node[midway, right] {$\phi_0$};

      % 标注初始点
      \node at (0, -0.2) {O};

  \end{tikzpicture}
\end{center}

Q1:在靶子下落的一瞬间开枪,问枪应该瞄准哪里?

一个很巧妙的方法,就是进行一个坐标系变换,即考虑竖直向下以加速度$g$运动的非惯性参考系。在该参考系中,靶子是静止的,而枪的子弹不会下落而是作直线运动。那自然就应该直接瞄准靶子就行了。

Q2:子弹发射的最小速度是多少?

在上一问的参考系中,很容易得出子弹飞行的时间是$\dfrac{|OT|}{v_{0}}$。$OT$的长度是给定的,那么要求的最小速度,可以通过求最大飞行时间来得到。最大飞行时间是多少呢?很简单,就是在靶子要落地的瞬间才击中,这样时间最长,这个问题也就迎刃而解了。

当然,也可以不用坐标系的变换,直接这样计算:
\[\left\{
\begin{aligned}
  & \text{子弹位置:} \quad \vec{r}_{_P}=\vec{v}_{0}t - \frac{1}{2}gt^{2}\hat{\vec{j}} \\
  & \text{靶子位置:} \quad \vec{r}_{_T}=\left(|OT|\cos\phi_{0}\hat{\vec{i}} + |OT|\sin\phi_{0}\hat{\vec{j}}\right) - \frac{1}{2}gt^2\hat{\vec{j}}
\end{aligned}
\right.\]
加之$\vec{r}_{_P}=\vec{r}_{_T}$,即可解出要打中需要的角度和击中的时间。

\subsection[匀速圆周运动]{\itr{Uniform Circular Motion}{匀速圆周运动}}
下面,用我们用向量方法来研究一下匀速圆周运动的加速度表达式:

\vspace*{2ex}
\begin{minipage}{0.9\textwidth}
	\begin{minipage}{0.6\linewidth}
		\begin{center}
			\thisincludegraphics[width=0.7\textwidth]{chapter2_uniform circular motion}
			
			匀速圆周运动(本图出自\url{https://www.cnblogs.com/1024th/p/10744727.html})
		\end{center}
	\end{minipage}
	\quad
	\begin{minipage}{0.35\linewidth}
			\[\vec{a}=\lim_{\Delta t \to 0} \frac{\Delta \vec{v}}{\Delta t}\]
				由简单的几何关系可知,任何一点的加速度$\vec{a}$都垂直于切线方向,指向圆心。
	\end{minipage}
\end{minipage}
\vspace*{2ex}

确定方向后,再确定大小,即有:
\begin{equation}
  \begin{aligned}
         \left | \Delta \vec{v} \right |&=2v\sin\frac{\theta}{2} \\
    \Longrightarrow \left | \vec{a} \right |&=\lim_{\Delta t \to 0}  \frac{2v\sin\frac{\theta}{2}}{\Delta t}=\lim_{\Delta t \to 0}  \frac{2v\sin\frac{v\Delta t}{2r}}{\Delta t}=\frac{v^{2}}{r} \\
    \Longrightarrow \vec{a}&=-\frac{v^{2}}{r}\hat{\vec{r}}
  \end{aligned}
  \nonumber
\end{equation}

也可以用上极坐标来研究这个问题。之前提到过,极坐标也有一对正交基,如果将其理解为$x,y$轴正交基的旋转的话,就有如下表达式:

\[\left \{ \hat{\vec{r}},\hat{\vec{\theta}} \right \} =\left \{ \hat{\vec{i}},\hat{\vec{j}} \right \}\begin{pmatrix}
  \cos\theta & -\sin\theta\\
   \sin\theta& \cos\theta
 \end{pmatrix} \]
对称地,也有:
\[\left \{ \hat{\vec{i}},\hat{\vec{j}} \right \} =\left \{ \hat{\vec{r}},\hat{\vec{\theta}} \right \}\begin{pmatrix}
  \cos\theta & \sin\theta\\
  - \sin\theta& \cos\theta
 \end{pmatrix} \]

 如此可以推出
 \begin{equation}
  \begin{aligned}
      \frac{\dif \hat{\vec{r}}}{\dif t}&=\frac{\dif \cos\theta}{\dif t}\hat{\vec{i}}+\frac{\dif \sin\theta}{\dif t}\hat{\vec{j}}\\
         &=-\sin\theta\frac{\dif \theta}{\dif t}\hat{\vec{i}}+\cos\theta\frac{\dif \theta}{\dif t}\hat{\vec{j}}\\
&=-\sin\theta\frac{\dif \theta}{\dif t}(\cos\theta\hat{\vec{r}}-\sin\theta\hat{\vec{\theta}})+\cos\theta\frac{\dif \theta}{\dif t}(\sin\theta\hat{\vec{r}}+\cos\theta\hat{\vec{\theta}})\\
&=\frac{\dif \theta}{\dif t}\hat{\vec{\theta}}
  \end{aligned}
  \nonumber
\end{equation}

类似地\footnote{这要仔细想想,$\frac{\dif \theta}{\dif t}$是角速度的大小,但$\frac{\dif \hat{\vec{\theta}}}{\dif t}$却并不是角速度!目前我们讨论的暂时都是角速度的大小,至于角速度本身,将在第三章详细介绍。},也有
\[\displaystyle\frac{\dif \hat{\vec{\theta}}}{\dif t}=-\frac{\dif \theta}{\dif t}\hat{\vec{r}}\]

这两个式子和匀速圆周运动加速度有啥关系?别急,慢慢推导:
\begin{equation}
  \begin{aligned}
    &\vec{v}=\frac{\dif \vec{r}}{\dif t}=\frac{\dif {r\hat{\vec{r}}}}{\dif t}=r\frac{\dif \hat{\vec{r}}}{\dif t}=r\frac{\dif \theta}{\dif t}\hat{\vec{\theta}}=r\omega \hat{\vec{\theta}}\\
\Longrightarrow &\vec{a}=\frac{\dif \vec{v}}{\dif t}=r\omega\frac{\dif \hat{\vec{\theta}}}{\dif t}=r\omega(-\frac{\dif \theta}{\dif t}\hat{\vec{r}} )=-\omega^{2}r\hat{\vec{r}}=-\frac{v^{2}}{r}\hat{\vec{r}}
  \end{aligned}
  \nonumber
\end{equation}

\subsection[变速圆周运动]{\itr{Non-uniform Circular Motion}{变速圆周运动}}
\begin{center}
	\begin{tikzpicture}
		% Draw the circle
		\draw[thick] (0,0) circle(2);
		
		% Define the point on the circle
		\coordinate (P) at (2,0);
		
		% Draw the arrows
		\draw[->, thick] (P) -- ++(-2,0) node[midway, below] {$a_{_R}$}; % Tangential arrow
		\draw[->, thick] (P) -- ++(0,2) node[midway, right] {$a_{_T}$, \itr{tangential acceleration}{切向加速度}}; % Radial arrow
		
		% Draw the point on the circle
		\fill[black] (P) circle (2pt);
	\end{tikzpicture}
\end{center}

\begin{Itemize}
 \item $\displaystyle\vec{v}=\frac{\dif \vec{r}}{\dif t}=\omega r\hat{\vec{\theta}}$
 \item $\displaystyle\omega=\frac{\dif \theta}{\dif t}$
 \item $\displaystyle\vec{a}=\frac{\dif \vec{v}}{\dif t}=\frac{\dif \omega}{\dif t}r\hat{\vec{\theta}}$(\itr{tangential}{切向的})$-\displaystyle\omega^2 r\hat{\vec{r}}$(\itr{centripedal}{径向的})
\end{Itemize}
\section[牛顿定律]{\itr{Newton's Laws}{牛顿定律}}
牛顿定律研究的是什么?简而言之,力与加速度之间的关系。

牛顿定律是普适的吗?不是,有两个限制条件:
\begin{center}
	\itr{low speed}{低速},\itr{macroscopic scale of interaction}{宏观尺度的作用}
\end{center}

突破了前一个限制条件,也就是不再远小于光速,就需要用相对论来讨论,这在之后的章节会详细介绍。突破了后一个限制条件,就需要量子力学来讨论了。
\subsection[牛顿第一定律]{\itr{Newton's First Law of Motion}{牛顿第一运动定律}}
\begin{law}[\itr{Newton's First Law of Motion}{牛顿第一运动定律}---\refleaftext{prove2.1}]
  \En{If no force \itr{acts on}{作用于} a \itr{body}{物体}, the body's \itr{velocity}{速度} cannot change. That is, when no force acts on an object, the acceleration of the object is zero. Any isolated object( one that does not interact with its environment) is either at rest or moving with constant velocity.}
\end{law}
牛顿定律提出的背景:
384BC\textasciitilde 322BC,\itr{Aristotle}{亚里士多德} 提出,物体的运动需要持续的动力(need continuous action of an agent)\mgnote{当然,你知道这是错误的。};
1564\textasciitilde 1642,\itr{Galilei}{伽利略} 提出,物体在没有阻碍的情况下会维持原来的运动状态(objects retain their motion in absence of any \itr{impediments}{阻碍})。
\subsection[牛顿第二定律]{\itr{Newton's Second Law of Motion}{牛顿第二运动定律}}
\begin{law}[\itr{Newton's Second Law of Motion}{牛顿第二运动定律}---\refleaftext{prove2.2}]
	\En{The acceleration of an object is directly \itr{proportional}{成比例} to the \itr{net force}{合外力} acting on it and \itr{inversely proportional}{成反比} to its mass.}
	\[\sum{\vec{F}}=m\vec{a}\]
\end{law}
\subsection[牛顿第三定律]{\itr{Newton's Third Law of Motion}{牛顿第三运动定律}}
\begin{law}[\itr{Newton's Third Law of Motion}{牛顿第三运动定律}---\refleaftext{prove2.3}]
 \En{The \itr{action force}{作用力} is equal in magnitude to the \itr{reaction force}{反作用力} and opposite in direction. }
\end{law}

\subsection[惯性参考系与非惯性参考系]{\itr{Inertial Reference Frame \& Non-inertial Reference Frame}{惯性参考系\&非惯性参考系}}
\begin{Itemize}
  \item 惯性参考系是指一个不受外力作用、未被加速且静止或匀速直线运动的参考系。在这个惯性参考系中,牛顿定律成立,即物体保持原有状态(静止或匀速直线运动)。
	\item 非惯性参考系是指一个相对于惯性参考系不保持匀速直线运动或静止的参考系。在非惯性参考系中,牛顿定律不成立\footnotemark。
\end{Itemize}
\footnotetext{这很好理解,不妨考虑一个不受力的物体,它在非惯性参考系中可以拥有加速度,这与牛顿第一定律冲突,因此牛顿定律不再成立。}

在应用中,人们往往通过牛顿第二定律分析物体的受力与运动的关系。为了在非惯性参考系中可以用牛顿第二定律处理问题,人们引入了称为\textbf{惯性力}的虚拟力。当我们认为非惯性参考系中的物体都额外受到惯性力$\vec{F}_{fictitious}=-m\Delta\vec{a}$\mgnote{其中$\Delta \vec{a}$为非惯性参考系相对惯性参考系的加速度。}时,牛顿第二定律依旧成立(\refleaftext{prove2.4})。


\subsection[惯性系中的相对运动]{\itr{Relative Motion(Still in Inertial Frame)}{惯性系中的相对运动}}
对于不同惯性参考系的观察者来说,对于同一个质点运动,他们观察到的加速度是一样的,但是观察到的位移和速度很可能是不一样的。

下面用伽利略变换\labelroot{Galilean transformation}来证明上面的结论:
\ctikzfig{chapter2_galilean_transformation}
\begin{center}
	伽利略变换
\end{center}
\begin{equation}
  \begin{aligned}
 \vec{r}'&=\vec{r}-\vec{u}t\\
\Rightarrow \frac{\dif \vec{r}'}{\dif t}&=\frac{\dif \vec{r}}{\dif t}-\vec{v_{0}}\\
\Leftrightarrow \vec{v}'&=\vec{v}-\vec{v_{0}}\\
\Rightarrow \frac{\dif \vec{v}'}{\dif t}&=\frac{\dif \vec{v}}{\dif t}\\
\Leftrightarrow \vec{a}'&=\vec{a}
  \end{aligned}
  \nonumber
\end{equation}

\subsection[非惯性系中的相对运动]{\itr{Relative Motion in Non-inertial Frame}{非惯性系中的相对运动}}
先来看一个例子:
\ctikzfig{chapter2_noninertial_frame}
\begin{center}
	向右匀加速运动车厢中的小球(小球与车厢相对静止)
\end{center}

在高中时,我们往往在地面参考系中对这个小球的受力情况进行分析,并求解车厢的加速度。而现在,我们尝试在车厢这个非惯性参考系中研究它。

在车厢这个参考系中,小球受到绳子的拉力,重力,以及水平向左的惯性力$\vec{F}_{fictitious}$。由于以车厢为参考系,小球是静止的,那么以水平方向受力为$\vec{0}$和竖直方向受力为$\vec{0}$就能列出两个式子,并解得
\[F_{fictitious}=T\sin\theta=mg\tan\theta\]
再结合惯性力的定义,有$F_{fictitious}=-m\Delta a$,即得车厢的加速度大小为$g\tan\theta$。


%还有一个例子,就是匀速圆周运动。我们是否能用非惯性参考系来研究匀速圆周运动呢?显然是可以的。比如以绳子牵引下作匀速圆周运动的小球为参考系,小球自己就受到一个沿径向向外的虚拟力的作用,这个力与绳子的拉力等其他外力的合力平衡。


\section[功与动能]{\itr{Work and Kinetic Energy}{功与动能}}
\subsection[恒力做功]{\itr{Work Done by a Constant Force}{恒力做功}}
\begin{Itemize}
    \item \En{Definition: The work \textit{W} done on an object by an \itr{agent}{施力者} exerting a constant force on the object is the \itr{product}{乘积} of the \itr{component}{分量} of the force in the direction of the displacement and the magnitude of the displacement.}
    \[W=\vec{F} \cdot \vec{d}=Fd\cos \theta \]
\end{Itemize}

在高中时期,我们反复提及两种不做功的情况:力作用在物体上但物体没有移动,或是力的作用方向 \itr{perpendicular to}{垂直于} 物体的位移方向。

由定义可知,功是力与位移的 \itr{scalar product}{数量积、内积}\mgnote{若忘了内积相关知识,可参见\refleaftext{chapter2_scalar_product}。}。因此,功是一个标量,其单位为$\mathrm{N}\cdot \mathrm{m}$或$\mathrm{J}$。

\subsection[变力做功]{\itr{Work Done By a Varying Force}{变力做功}}
由于功的定义只与力和位移有关,所以这个变力不宜理解为力随时间变化而变化。更准确的理解是,力会随力所作用的物体的位移变化而变化。

变力与恒力做功之间,能够用微分的思想架起桥梁,即
\[W=\lim_{\Delta x \to 0} \sum_{x_{i}}^{x_{f}}F_{x}\Delta x=\int_{x_{i}}^{x_{f}}F_{x}\dif x\]
在这里,我们将变力做功转化为无数个小位移时的恒力做功的和。计算出来的结果,也就是这个力在$F-x$图上的曲线与$x$轴围成的面积,如下。
\ctikzfig{chapter2_work_done_by_varying_force}
\begin{center}
	变力做功的计算
\end{center}
力会随位移而改变的一个经典例子是弹簧,这就不得不提到著名的胡克定律\footnote{需要注意,这里的负号十分重要,这表示弹簧提供的力与弹簧被拉伸一段的位移是相反的,这符合$\vec{F}_{s}$与$\vec{x}$的矢量属性。}:
\[\vec{F}_{s}=-k\vec{x}\]
这里的$k$表示弹簧的劲度系数,$\vec{x}$表示弹簧被拉伸的一段相对于其初始位置的位移。

还记得高中时候讲的弹簧形变量为$x_{max}$的时候,弹簧储存的能量是多少吗?没错,是$\dfrac{1}{2} kx_{max}^{2}$\labelroot{chapter2_potential_energy_of_spring}。现在,我们就可以用上面变力做功的公式来证明它:
\[W_{s}=\int_{x_{i}}^{x_{f}}F_{s}\mathrm{d}x=\int_{-x_{max}}^{0}(-kx)\mathrm{d}x=\frac{1}{2} kx_{max}^{2}\]
这时候有人可能要问,这积分的方向咋看,积出来是负的咋办?其实这是要看做功时的位移方向的。我们要计算弹簧储存的弹性势能,这里考察的是弹簧从拉伸状态恢复为原长的过程中释放的能量,那么积分方向就应该是弹簧复原的方向,从$-x_{max}$到$0$或是$x_{max}$到$0$都可以计算出正的结果。

\ctikzfig{chapter2_hook}
\begin{center}
	弹簧的变力做功
\end{center}
\subsection[动能]{\itr{Kinetic Energy}{动能}}
大家应该都知道低速运动物体的动能公式是$\frac{1}{2}mv^{2} $,但是为什么要这样定义?这个$\frac{1}{2}$是怎么来的?其实这样的定律,从一个简单的匀加速运动中就可以看出其合理性:

设一个物体质量为$m$,且在恒力作用下作加速度为$a$的匀加速直线运动,则可作如下推导:
\begin{equation}
    \begin{aligned}
    	&\hspace*{-3em}\left\{\begin{aligned}
    		\sum W&=Fd=(ma)d\\
    		d&=\frac{1}{2}(v_{i}+v_{f})t\\
    		a&=\frac{v_{f}-v_{i}}{t}
    	\end{aligned}\right.\\[1ex]
  \Longrightarrow \sum W&=m\frac{v_{f}-v_{i}}{t}\frac{1}{2}(v_{i}+v_{f})t\\[1ex]
  &=\frac{1}{2}mv_{f}^{2}-\frac{1}{2}mv_{i}^{2}
    \end{aligned}
    \nonumber
\end{equation}
这样,动能公式的定义就显得十分合理了。物体动能的变化量等于外力做功,这就是下面会详细介绍的动能定理。

对于高速运动物体的动能,就需要用相对论来讨论了,这在之后的章节中会详细介绍。事实上,利用相对论意义下的动能公式的近似,也可以推导出$\dfrac{1}{2}mv^{2}$的公式。
\subsection[动能定理]{\itr{Work-kinetic Energy Theorem}{功能关系理论(动能定理)}}
\begin{law}[\itr{Work-kinetic energy theorem}{功能关系理论(动能定理)}---\refleaftext{prove2.5}]
    \En{The net work done on a particle by the net force acting on it is equal to the change in the kinetic energy of the particle.}
\end{law}
用公式表示,即$\sum W =K_{f}-K{i}=\Delta K$。无论这个 \itr{net force}{合外力} 是否恒定,这个公式都是适用的。
\subsection[功率]{\itr{Power}{功率}}
\begin{Itemize}
  \item \En{\itr{Power}{功率}: The \itr{time rate}{速率} of doing work, or energy transfer.}
  \item \itr{Average power}{平均功率}: $\upperline{P} =\dfrac{W}{\Delta t}$
  \item \itr{Instantaneous power}{瞬时功率}:$P=\dfrac{\dif W}{\dif t}=F\cdot\dfrac{\dif s}{\dif t}=\vec{F}\cdot\vec{v}$
  \item SI Unit of power: the watt $W=1\mathrm{J}/\mathrm{s}=1\mathrm{kg}\cdot \mathrm{m}^2/\mathrm{s}^3$
\end{Itemize}

\section[能量守恒]{\itr{Conservation of Energy}{能量守恒}}
\subsection[动能定理复习]{\itr{Revisit Work-kinetic Energy Theorem}{重温功能关系理论(动能定理)}}
\begin{Itemize}
    \item \En{The \itr{net work}{总功} done on a particle by the net force acting on it is equal to the change in the kinetic energy of the particle.}
\end{Itemize}
用公式表示就是:$\sum W=K_{f}-K_{i}=\Delta K$

下面给出一个例子。对于一个匀加速运动,一个质量为$m$的物体受到恒力的作用,以加速度$a$作匀加速运动。尝试用动能定理推导出
\[x_{f}-x_{i}=\frac{v_{f}^{2}-v_{i}^{2}}{2a}\]
\subsection[势能]{\itr{Potential Energy}{势能}}
势能简单来说,就是与一个系统有关的能量。比如说,一个弹簧组成的系统中可能存储弹性势能,一个物体和地球这两者组成的系统中会储存重力势能。还有一种说法,就是势能是相互作用的物体凭借其相对位置而具有的能量。

弹簧弹性势能(\refleaftext{chapter2_potential_energy_of_spring})为$U_{s}=\frac{1}{2}kx^{2}$,其中$k$是弹簧的弹性势能,$x$是弹簧的形变量。

对于重力势能来说,近地表物体的重力势能公式为$U_{g}=mgh$,其中$h$为物体距离参考平面的高度\footnote{课上的ppt认为$h$就是距离地球表面的高度,也许是为了将重力势能标准化。}。

接下来,大家思考一下,摩擦力的产生也需要相互作用的物体,那么摩擦力是否有势能呢?在解答这个问题之前,首先来了解一下\textbf{保守力}和\textbf{非保守力}这一组概念。
\subsection[保守力与非保守力]{\itr{Conservative \& Non-conservative Forces}{保守力与非保守力}}
摩擦力有一个特征,那就是始终沿着相对运动的反方向。那么很容易想象,如果摩擦力大小恒定为$f$,而且它在阻碍一个物体做一定轨迹的运动。那么摩擦力做的功就是$W=-f\cdot l$,其中$l$是轨迹的长度。而重力做功则不一样,根据重力做功的表达式$mgh$,重力做功只与起始位置和终点的高度差有关,而与路径无关。

从这个例子中,我们提取一个规律:重力做功与路径无关,而摩擦力做功与路径有关。所以,我们称重力是 \itr{conservative}{保守的},摩擦力是 \itr{nonconservative}{非保守的}。
\begin{Itemize}
    \item \En{A force is \itr{conservative}{保守的} if the work it does on a particle moving between any two points is independent of the path taken by the particle.}
    \item \En{The work done by a conservative force on a particle moving through any \itr{closed path}{闭合路径} is zero. A closed path is one in which the beginning and end points are identical.}
\end{Itemize}
\subsection[机械能守恒]{\itr{Conservation of Mechanical Energy}{机械能守恒}}
一个保守力$\vec{F}$在一个系统内做功为\mgnote{再次说明,i表示initial,f表示final}
\[\sum W=\int_{i}^{f} F\cdot \dif s=U_{i}-U_{f}\]也即所做的功等于系统前后势能之差。

另一方面,由动能定理,有
\[\sum W=K_{f}-K_{i}=\Delta K\]

综上,两式联立,即得
\[K_{i}+U_{i}=K_{f}+U_{f}\]

如此,我们得到机械能守恒定律如下:

\begin{Itemize}
    \item \En{The total \itr{mechanical energy}{机械能} of a system remains constant in any isolated system of objects that interact only through \itr{conservative forces}{保守力}}.
    \item\ \ ---\En{No energy is added to or removed from the system(isolated).}
    \item\ \ ---\En{No nonconservative forces are doing work within the system.}
    \item\ \ ---\En{Sum over the potential energy is only associated with each conservative force.}
\end{Itemize}

注意第三条性质,总势能只与保守力有关,这也就意味着摩擦力并不具有势能,解答了之前的问题。

还有一个特别的计算方法:

	\En{Any conservative force acting on an object within a system equals the negative derivative of the potential energy of the system with repect to $\vec{r}$.}

也就是说,保守力等于该保守力对应势能对位矢$\vec{r}$的负导数。用向量表示就是
\[\vec{F}=-\hat{\vec{i}}\frac{\partial U}{\partial x}-\hat{\vec{j}} \frac{\partial U}{\partial y} -\hat{\vec{k}} \frac{\partial U}{\partial z} \]

比如对于弹簧,有
\[\vec{F}_{s}=-\frac{\dif U_{s}}{\dif x}\hat{\vec{i}}=-\frac{\dif }{\dif x}(\frac{1}{2}kx^{2})\hat{\vec{i}}=-k\vec{x}\]

对于重力,有
\[\vec{F}_{g}=-\frac{\dif U_{g}}{\dif y}\hat{\vec{j}}=-\frac{\dif (mgy)}{\dif y}\hat{\vec{j}}=-mg\hat{\vec{j}}=m\vec{g}\]

\subsection[系统的平衡性]{\itr{Equilibrium of a System}{系统的平衡性}}
以下考虑只有一维坐标$x$的情形。
\begin{Itemize}
 \item \En{In general, positions of \itr{stable equilibrium}{稳定平衡} correspond to points for which $U(x)$ is a \itr{minimum}{极小值点}, that is, }$\dfrac{\dif U}{\dif x}=0\quad\dfrac{\dif ^2U}{\dif x^{2} }>0$。
 \item \En{In general, positions of \itr{unstable equilibrium}{不稳定平衡} correspond to points for which $U(x)$ is a \itr{maximum}{极大值点}, that is, }$\dfrac{\dif U}{\dif x}=0\quad\dfrac{\dif ^2U}{\dif x^{2} }<0$。
\end{Itemize}
\section[动量守恒]{\itr{Conservation of Momentum}{动量守恒}}

\subsection[动量定理]{\itr{Impulse-Momentum Theorem}{动量定理}}
首先,让我们从一个看似无关的问题开始讨论:

\emph{在伽利略变换(\refleaftext{Galilean transformation})下,动能定理是否依旧成立,即动能定理在伽利略变换下是否具有不变性?}

考虑一个最简单的情况——一个受恒力$F$的质点在做一维的运动。首先,让我们把变换前后的功和动能建立起联系:
\begin{equation}
  \begin{aligned}
     &W'=Fd'=F(d-v_{0}t)=W-v_{0}Ft\\
     &K'=\frac{1}{2}m(v')^{2}=\frac{1}{2}m(v-v_{0})^{2}=K-v_{0}mv+\frac{1}{2}mv_{0}^{2}
  \end{aligned}
  \nonumber
\end{equation}
联立上面两个式子,有
\[W'-(K_{f}'-K_{i}')=W-(K_{f}-K_{i})-v_{0}Ft+v_{0}m(v_{f}-v_{i})\]

根据动能定理$W=K_{f}-K_{i}$,要想$W'=K_{f}'-K_{i}'$依然成立,就必须有$Ft=mv_{f}-mv_{i}$成立。幸运的是,我们发现这个式子确实成立。所以,动能定理在伽利略变换下具有不变性。同时,我们将这个式子命名为\textbf{动量定理}。

\begin{law}[\itr{Impulse-Momentum Theorem}{动量定理}---\refleaftext{prove2.6}]
  \En{The \itr{impulse}{冲量} of the force $\vec{F}$ acting on a particle equals the change in the linear momentum of the particle caused by that force, that is, $\vec{I}=\Delta \vec{p}$.}
\end{law}
其中,\itr{Linear momentum}{动量} 定义为$\vec{p}=m\vec{v}$, \itr{Impulse}{冲量} 定义为$\displaystyle \vec{I}=\int_{t_{i}}^{t_{f}}\vec{F}\dif t $。

\subsection[动量守恒]{\itr{Conservation of Linear Momentum}{动量守恒}}
先考虑两个小球$1$和$2$碰撞的情形(两小球与世隔绝)。由牛顿第二定律,有
\[\vec{F}_{21}=\frac{\dif \vec{p}_{1}}{\dif t}  \quad \vec{F}_{12}=\frac{\dif \vec{p}_{2}}{\dif t}\]
又由牛顿第三定律有
\[\vec{F}_{21}+\vec{F}_{12}=\vec{0}\]
于是有
\[\frac{\dif }{\dif t}(\vec{p_{1}}+\vec{p_{2}})=\vec{0}\]

这个规律可以有如下推广:
\begin{Itemize}
  \item \itr{Conservation of Linear Momentum}{动量守恒} \En{Whenever two or more particles in an isolated system interact, the total momentum of the system remains constant.}
  \item \En{Note: the only requirement is that the forces must be \itr{internal}{内部的} to the system.}
\end{Itemize}

\subsection[碰撞]{\itr{Collision}{碰撞}}
\begin{Itemize}
  \item \En{Definition: Collision is the event of two particles' coming together for a short time and thereby producing \itr{impulsive}{瞬时的} forces on each other.}
\end{Itemize}

在任何没有合外力或是合外力可以忽略的碰撞中,碰撞系统的动量都是守恒的。然而,这个系统的动能不一定守恒。我们把碰撞过程中动能守恒的碰撞称作 \itr{Elastic collision}{弹性碰撞},反之则称为 \itr{Inelastic collision}{非弹性碰撞}。

特别地,在\itr{Inelastic collision}{非弹性碰撞}中,有种碰撞叫\itr{Perfectly Inelastic Collision}{完全非弹性碰撞}。在这种情况下,碰撞后的物体会挨在一起,拥有一个共同速度。此时,动能的损失是最大的。这用柯尼希定理(\refleaftext{law3.2})很容易证明。

关于一维的弹性碰撞,我们常常碰到相关的计算问题,这里提供一个比较容易的背结论的方法:

设两物体$m_{1},m_{2}$发生碰撞,初速度分别为$\vec{v}_{1i}$和$\vec{v}_{2i}$。首先,计算质心速度为
\[\vec{v}=\frac{m_{1}\vec{v}_{1i}+m_{2}\vec{v}_{2i}}{m_{1}+m_{2}}\]
然后,我们就可以直接得到末速度
\[\left\{\begin{aligned}
	\vec{v}_{1f}&=2\vec{v}-\vec{v}_{1i}\\
	\vec{v}_{2f}&=2\vec{v}-\vec{v}_{2i}
\end{aligned}\right.\]

若是遇到有角度的二维碰撞,情况就略复杂一点,但总体思路也是很简单的,无非是动量守恒和能量守恒表达式的联立。其中动量守恒会涉及多个方向的分析,可能不止一个表达式。

\subsection[质心系中的碰撞]{\itr{Collision in the CM(Center of Mass) Frame}{质心系中的碰撞}}
\textbf{质心}是分析系统的运动学性质时非常重要的概念。在向量意义下,选定了参考原点后,一个物体(或是一个多质点体系)的质心的位置被定义为
\[\vec{r}_{_{CM}}=\frac{\sum_{i}^{}m_{i}\vec{r}_{i}}{M}=\frac{1}{M}\int \vec{r}\dif m\]

\ctikzfig{chapter2_CM}
\begin{center}
	质心的位置
\end{center}

对于一个系统,其将质心作为参考系原点的参考系被称为系统的\textbf{质心系}(CM Frame)。因此,在质心系中,有
\[\vec{r}_{_{CM}}=\frac{\sum_{i}^{}m_{i}\vec{r}_{i}}{M}=\vec{0}\]

此外,关于质心系还有一个很重要的结论,即
\begin{law}[质心系中,系统的总动量为$\vec{0}$ ---\refleaftext{prove2.7}]
	记系统在其质心系中的总动量为$\vec{P}^{CM}$,则有
	\[\vec{P}^{CM}=\vec{0}\]
\end{law}

\section[重力]{\itr{The Law of Gravity}{重力相关的定律}}
\subsection[牛顿万有引力定律]{\itr{Newton's Law of Universal Gravitation}{牛顿万有引力定律}}
牛顿观察到三种现象:苹果从树上坠落到地面,月球绕着地球公转,以及行星绕着太阳公转。这三件事在我们看来理所应当,这是因为我们已经知道了万有引力的存在。如果我们能够将自己代入牛顿的年代,几乎没有任何的与引力有关的理论铺垫,没有矢量运算工具,没有微积分工具,我们就会感叹牛顿是有多么天才,才能够将这三件事的内在联系发掘出来。

通过这三类现象和一些计算推导\footnote{在《自然哲学的数学原理》这本书中,牛顿记录了推导万有引力的计算方法。主要运用的工具是欧几里得几何学、代数学以及自己的“微分和积分的思想”。}(牛顿有非常恐怖的数学功底),牛顿统一了天上和地上的运动:
\begin{law}[\itr{The Law of Universal Gravitation}{万有引力定律} ---\refleaftext{prove2.8}]
       \En{Every particle in the Universe attracts every other particle with a force that is directly \itr{proportional}{成比例} to the \itr{product}{乘积} of their masses and \itr{inversely proportional}{反比} to the \itr{square}{平方} of the distance between them.}
       
\end{law}
若用公式表示万有引力的大小,就是$F_{g}=G\dfrac{m_{1}m_{2}}{r^{2}}$。

地月检验进一步证明了万有引力定律的正确性:

设地球质量为 \( M_E \),月球质量为 \( M_M \),两者之间的平均距离为 \( R \)。根据万有引力定律,地球对月球的引力 \( F_E \) 为:

\[
F_E = G \frac{M_E M_M}{R^2}
\]

月球的运动可以用开普勒定律(\refleaftext{law2.8})描述。根据万有引力定律,月球的运动周期$T$应满足:

\[
T^2 = \frac{4 \pi^2 R^3}{G (M_E + M_M)}
\]

因为 \( M_M \) 相对 \( M_E \) 较小,可以近似认为
\[
T^2 \approx \frac{4 \pi^2 R^3}{G M_E}
\]

根据实际观测,月球的运动周期约为 27.3 天。计算得到周期应符合:

\[
T = 2 \pi \sqrt{\frac{R^3}{G M_E}}
\]

通过将 \( R \) 和 \( M_E \) 的已知值代入,可以得到理论计算的周期,并与观测值进行比较。

通过精确测量月球的轨道周期和位置,以及计算得到的引力,科学家们发现万有引力定律能够准确预测月球的运动。这一结果不仅验证了牛顿的理论,而且奠定了经典力学的基础。

\subsection[行星运动以及开普勒定律]{\itr{Motion of the planets \& Kepler's laws}{行星运动以及开普勒定律}}
通过实验观察和数据分析,开普勒总结出了三大定律。
\begin{law}[开普勒三大定律]
	\begin{itemize}
		\item \En{All planets move in \itr{elliptical}{椭圆} orbits with the Sun at one \itr{focal point}{焦点}.}
		\item \En{The radius vector drawn from the Sun to a planet \itr{sweeps out}{扫过} equal areas in equal \itr{time interval}{时间间隔}(Exactly reflects the conservation of angular momentum).}
		\item \En{The square of the orbital period of any planet is proportional to the \itr{cube}{立方} of the \itr{semimajor axis}{半长轴} of the elliptical orbit.}
	\end{itemize}
\end{law}

这里提一下 \itr{Conic Section}{圆锥曲线}。牛顿分析后认为,重力场中可能的环绕轨迹是圆锥曲线\mgnote{圆锥曲线由于它们能够通过切割一个圆锥面得到而获名。}。
\begin{singlefigure}[圆锥曲线]{chapter2_conic_section.png}[0.55]
\end{singlefigure}
\subsection[测量引力常数]{\itr{Measuring the gravitational constant}{测量引力常数}}
测量重力常数的扭摆实验由亨利·卡文迪许 (Henry Cavendish) 在1798年首次进行。

\begin{Itemize}
    \item 使用一个水平的扭摆装置,悬挂两个小质量 \( m \) 的球体。
    \item 通过引力将球体放置在较大质量 \( M \) 的球体附近。
    \item 测量扭摆装置的角位移 \( \theta \)。
    \item 根据扭摆的转动方程 \( \vec{\tau} = -k \vec{\theta} \) 计算扭矩,其中 \( k \) 是扭摆常数。
\end{Itemize}

根据平衡条件,可以得到:

\[
G = \frac{4 \pi^2 k r^2}{m_1 m_2 \sin\theta}
\]
\begin{singlefigure}[扭摆实验]{chapter2_cavendish.png}[0.6]
\end{singlefigure}
现代测量方法$^*$涉及更精确的仪器和技术:

\begin{Itemize}
    \item 使用高精度振动天平来测量小质量的引力。
    \item 利用激光干涉仪技术提高测量精度。
\end{Itemize}

通过多次实验,重力常数的国际标准值为:

\[
G \approx 6.674 \times 10^{-11} \, \text{m}^3 \, \text{kg}^{-1} \, \text{s}^{-2}
\]

\subsection[自由落体加速度]{\itr{Free-fall acceleration}{自由落体加速度}}
在更加大的尺度上,自由落体的加速度大小不能用$g$来一言以蔽之。设靠近地球表面的物体的重力加速度大小为$g_0$,地球质量为$M_E$,半径为$R_E$,则对于离地球表面高度为$h$的物体,其严格的重力加速度为
\[g'=\frac{GM_{E}}{(R_{E}+h)^{2}}\approx g_{0}-2g_{0}\frac{h}{R_{E}}\]
\subsection[卫星运动]{\itr{Satellite motion}{卫星运动}}
卫星的总机械能(不考虑推进器以及旋转的动能)为$\displaystyle E=\frac{1}{2}mv^{2}-\frac{GMm}{r}$。

\section*{第一宇宙速度}
卫星要作环绕地球的匀速圆周运动,于是有
\[\frac{GMm}{r^{2}}=ma=\frac{mv^{2}}{r}\]
所需机械能则可化简为
\[E=\frac{1}{2}mv^{2}-\frac{GMm}{r}=-\frac{GMm}{2r}\]

考虑最小速度,那么$r$近似为地球半径,故有
\[\frac{1}{2}mv^{2}-\frac{GMm}{R_{E}}=-\frac{GMm}{2R_{E}}\]
可得
\[v_{1}=\sqrt{\frac{GM}{R_{E}}}\approx7.8km/s\]
\section*{第二宇宙速度}
第二宇宙速度是指物体能够克服地球引力的发射速度。

为了克服引力,物体需达到的总能量为零\mgnote{设物体恰好能够到达无穷远处,此时速度几乎为$0$,地球对物体的引力势能也几乎为$0$。}:

\[
E = K + U = 0
\]
也即
\[
\frac{1}{2}mv^2 - \frac{GMm}{R} = 0
\]
因此,第二宇宙速度 \( v_2 \) 为
\[
v_2 = \sqrt{\frac{2GM}{R}}\approx11.2\mathrm{km/s}
\]

\section*{第三宇宙速度$^*$}
第三宇宙速度$v_3$是指物体能够挣脱太阳引力的在地球上的发射速度。

注意,这里不能仅仅将推导第二宇宙速度的公式中的$M$换成太阳的质量。在这种情况下,航天器需要同时挣脱地球的引力和太阳的引力;同时,地球绕太阳公转也为航天器提供了一部分初速度。
\[
v = \sqrt{\frac{2GM}{R}}\approx42.2\mathrm{km/s}
\]
其中$M$为太阳的质量,$R$为地球与太阳的距离。

减去地球公转速度,为
\[v'=v-v_{0}=42.2-29.8=12.4\mathrm{km/s}\]

请注意,这里的$v'$是以地球为参考系时,物体达到无穷远时需拥有的速度。

于是,物体在地球上发射时,还应拥有用于克服地球引力的动能,即第二宇宙速度对应的动能,故有
\[\frac{1}{2}mv_{3}^{2}=\frac{1}{2}mv_{2}^{2}+\frac{1}{2}m(v')^{2}\]

可解得
\[v_{3}=16.7\mathrm{km/s}\]
\subsection[火箭推进]{\itr{Rocket propulsion}{火箭推进}}
从动量角度研究火箭推进是很多题目喜欢考察的:
\begin{singlefigure}[火箭推进]{chapter2_rocket.png}[0.8]
\end{singlefigure}  
由动量定理有
\[
(M+\Delta m)v=M(v+\Delta v)+\Delta m(v-v_{e})\mgnote{$v_{e}$中$e$表示eject,$v_{e}$就是气体喷射速度。气体喷射速度是一个相对速度。}
\]
整理得
\[
M\Delta v=v_{e}\Delta m\Rightarrow  M\dif v=v_{e}\dif m=-v_{e}\dif M
\]
由此可以得到
\[
  \int_{v_{i}}^{v_{f}}v=-v_{e}\int_{M_{i}}^{M_{f}}\frac{\dif  M}{M}
\]
即
\[
v_{f}-v_{i}=v_{e}\ln{\frac{M_{i}}{M_{f}}}
\]

这也告诉我们,火箭不带燃料的部分越轻,火箭所带燃料越多,火箭能达到的速度越大。

其实广义来讲,火箭推进也算一类非弹性碰撞。整个过程中动量守恒,但机械能是增加的。