\chapter[转动动力学]{\itr{Rotation Dynamics}{转动动力学}\mgnote*{\raggedleft 作者:23级-陈锦浩}}
在质点运动学中,我们讨论的都是质点。然而,物体往往是占据着一定的空间,有着自己的质量分布特性的。那么,只靠质点运动学,还能够解决物体的运动问题吗?答案是否定的。为了解决这个问题\mgnote{更严谨地说,刚体的运动问题},物理学家们建立了转动动力学。
\section[刚体]{\itr{Rigid Body}{刚体}}
想象一个物体,它受到力产生加速度的同时,各个部位还会发生形变,这样的运动显然是非常复杂的。我们简化了研究对象,选择那些具有体积,且形状固定不变的物体。这样的物体就叫做刚体。
\begin{Itemize}
    \item \itr{rigid body}{刚体}:\En{An object that is
        \itr{nondeformable}{不可形变的}, or in other words, an object in which the
        separations between all parts of particles remain
        constant.}
\end{Itemize}
\section[转动]{\itr{Rotation}{转动}}
我们已经非常了解平动了。鉴于刚体的运动可以分为平动和转动,我们不妨先放弃平动,单独对转动做出分析。

\subsection[极坐标系]{\itr{Polar Coordinate \& Cylindrical Coordinate}{极坐标系与柱坐标系}\mgnote*[30pt]{作者:23级-陈锦浩}}
转动显然涉及旋转,旋转又显然涉及角度,而角度大小的确定需要一个基准,这就给了我们“选定一个旋转轴”分析转动的思想。

为了描述角度,我们引入极坐标系这一数学工具。

极坐标系\footnote{你可能会注意到,在\refleaftext{chapter2_polar_coordinate} 中,我们已经介绍过了极坐标系。然而,这里将引入的极坐标系,其中的一些约定将与之前有所不同。传统上,我们对极坐标系的基底的定义与之前的定义相同,这是因为,我们只考虑了二维的情况,无法再添加一个维度来描述角度矢量。而在这里,为了更好地研究旋转,我们在对角度矢量的定义中引入了垂直于平面的方向。在本章及其证明与习题中所提到的极坐标系,都按照本章的约定方式。}的基本要素是极点和极轴,其中极轴是一条从极点引出的有向轴。
\ctikzfig{chapter3_polar_coordinate}
\begin{center}
    \em 这是一个极坐标系,$O$为极点,$x$轴方向为极轴方向
\end{center}
在极坐标系中,我们使用坐标$(r,\theta)$或者$(|\vec{r}|,|\vec{\theta}|)$描述点的位置。

\begin{Itemize}
    \item $\vec{r}$即起点为极点,终点为被描述点的向量,称为极径矢量。
    \item $\vec{\theta}$即角度矢量,方向\labelroot{chapter3_direction}垂直于坐标平面向纸面外。我们规定,从极轴出发旋转到极径矢量的旋转角,\itr{counterclockwise}{逆时针} 时方向为正,\itr{clockwise}{顺时针} 时方向为负\footnotemark。
\end{Itemize}
\footnotetext{当然,对于一个第四象限的点,你可以认为它是逆时针旋转得到的优角(就是大于180度的角),也可以认为它是顺时针旋转得到的锐角。在本书中,由于叉乘的考量,我们往往取锐角考虑。}

然而,仅仅介绍极坐标系是不足够的,因为极坐标系仅有二维,但一个刚体的运动却极有可能是三维的。因此,我们还需要进一步介绍柱坐标系。

柱坐标系是极坐标系往$z$轴平面的延伸。柱坐标相比极坐标的$r,\theta$还多了一个$z$,用于指示点在$z$轴上的投影位置。

\hspace*{-0.5em}\begin{minipage}{0.6\textwidth}
    \centering
    \begin{tikzpicture}[scale=1]
        \coordinate (O) at (0,0);
        \coordinate (Ox) at (-3,-3);
        \coordinate (Oy) at (4.243,0);  % sqrt{18}
        \coordinate (Oz) at (0, 6);

        % draw axis
        \draw[-latex, line width=1] (O)-- (Ox) node[below] {$x$};
        \draw[-latex, line width=1] (O)-- (Oy) node[right] {$y$};
        \draw[-latex, line width=1] (O)-- (Oz) node[above] {$z$};


        % draw arcs
        \draw[thick] ($(0, 0) + (236:3cm and 2cm)$(P) arc
        (236:360:3cm and 2cm);
        \draw[thick] ($(0, 0) + (236:3cm and 2cm)$(P) arc
        (236:360:3cm and 2cm);

        \draw[thick] ($(0, 5) + (236:3cm and 2cm)$(P) arc
        (236:360:3cm and 2cm);

        \draw[thick, -latex] ($(0, 0) + (236:1.5cm and 1cm)$(P) arc
        (236:310:1.5cm and 1cm);

        \coordinate (Phi) at (0,-1) ;
        \node[below] at (Phi) {$\theta$};


        \coordinate (A1) at (0, 5);
        \coordinate (B) at (3, 5);
        \coordinate (C) at (-1.7, 3.3);
        \draw[thick] (A1)--(B);
        \draw[thick] (A1)--(C);



        % radius
        \coordinate (D) at (1.9,-1.5);
        \coordinate (P) at (1.9,3.5);
        \draw[thick] (O)--(D);
        \draw[thick, dashed] (A1)--(P) node[right, yshift=-1mm] {$P$};
        \draw[thick] (D)--(P);
        \fill[black] (P) circle (3pt);


        \coordinate (A) at (2.6, 4.0);
        \draw[thick, dashed] (A1)--(A) node[right, yshift=-1mm, xshift=-1mm] {$A$};


        % arcs
        \draw[thick] ($(0, 5) + (310:1.8cm and 1.2cm)$(P) arc
        (310:330:1.8cm and 1.2cm);

        \draw[thick] ($(0, 3.5) + (310:1.8cm and 1.2cm)$(P) arc
        (310:330:1.8cm and 1.2cm);

        \draw[thick] ($(0, 3.5) + (310:3cm and 2cm)$(P) arc
        (310:330:3cm and 2cm);

        \coordinate (Q) at (1.9,1.97);
        \node[below,xshift=2mm] at (Q) {$Q$};
        \node[xshift=-3mm] at (O){$O$};
        % \fill[black] (Q) circle (3pt);


        \coordinate (B) at (2.6, 2.5);
        \node[below,xshift=1mm] at (B) {$B$};
        % \fill[black] (B) circle (3pt);
        \draw[thick] (A) --(B);

        \coordinate (S) at (1.15, 4.1);
        \node[below, xshift=-2mm] at (S) {$S$};
        % \fill[black] (S) circle (3pt);

        \coordinate (R) at (1.15, 2.6);
        \node[below, xshift=-2mm] at (R) {$R$};
        %\fill[black] (R) circle (3pt);


        \coordinate (D) at (1.52, 4.42);
        \node[above] at (D) {$D$};
        % \fill[black] (D) circle (3pt);

        \coordinate (C) at (1.54, 2.86);
        \node[below] at (C) {$C$};
        %\fill[black] (C) circle (3pt);

        \draw[thick] (S) --(R);
        \draw[thick] (D) --(C);
        \draw[thick] (R) --(Q);
        \draw[thick] (C) --(B);

        % verticals on the planes
        \coordinate (H) at (-1.65,-1.65);
        %\fill[black] (H) circle (3pt);
        %
        \coordinate (I) at (-1.65,3.35);
        %\fill[black] (I) circle (3pt);
        \draw[thick] (H) --(I);

        \coordinate (J) at (3,0);
        %\fill[black] (J) circle (3pt);
        \coordinate (K) at (3,5);
        %\fill[black] (K) circle (3pt);
        \draw[thick] (J) --(K);

        % filling
        \filldraw[opacity=0.2]
        (D)--(A) arc (325:306:3cm and 2.2cm)--(S)
        arc (305:325:1.8cm and 1.2cm)--cycle;

        \filldraw[opacity=0.2]
        (P) arc (306:325:3cm and 2.2cm)--(B)
        arc (325:306:3.0cm and 2.2cm)--cycle;

        \filldraw[opacity=0.2]
        (P)--(Q)--(R)--(S)--cycle;

        % differential labels
        \node[right, yshift=1mm,xshift=2mm, rotate=-20] at (Q) {$r \dif \theta$};
        \node[right, yshift=6mm, xshift=-1mm ] at (B) {$\dif z$};
        \node[right,xshift=3mm, yshift=2mm, rotate=-20] at (D) {$\dif r$};

    \end{tikzpicture}
    \refstepcounter{figure}

    图\thefigure: 柱坐标系\footnote{本图来自 \url{https://www.latexstudio.net/index/details/index/mid/813.html}}
\end{minipage}
\begin{minipage}{0.35\textwidth}
    如图,这是一个以$x$轴为极轴的柱坐标系,$P$的坐标为$(r,\theta,z)$。
    \[\vec{OP}=\vec{r}+\vec{z}\]
    阴影部分是一个极小的体积元(这里放大绘制)。
\end{minipage}
\vspace{2ex}

在对刚体运动进行描述时,我们往往选取刚体的旋转轴作为$z$轴建立柱坐标系,然后借助柱坐标系说明其运动的相关参数。


\subsection[叉乘]{\itr{Cross Product}{叉乘}\labelroot[16pt]{chapter3_cross_product}}
在介绍转动的有关概念前,我们还需引入叉乘这一数学工具。

由于叉乘的具体内容介绍属于线性代数与数学分析的任务,这里并不会展开讲。

叉乘可以看成这样一个运算:按顺序输入两个向量$\vec{A},\vec{B}$,输出一个特定长度且与它们都正交的向量$\vec{C}$。当然,这样的描述是不精确的,一是长度需要确定的值,二是与$\vec{A},\vec{B}$都正交的向量方向不唯一\mgnote{除了$\vec{0}$,还有正反两个方向}。因此,接下来,我们要对长度和方向作出具体描述。

\begin{Itemize}
    \item 长度: $C=AB\sin\theta$,其中$\theta$是由$\vec{A}$旋转至$\vec{B}$所成的小于$\dfrac{\pi}{2}$的角\footnotemark 。在几何意义上,$|\vec{C}|$即是$\vec{A},\vec{B}$构成的平行四边形的面积。
    \item 方向: $\vec{C}$的方向是介绍长度时所述的$\vec{\theta}$的方向\mgnote{关于角度矢量的方向,可回见\\\refleaftext{chapter3_direction}。}。在几何意义上,$\vec{C}$与$\vec{A},\vec{B}$所张成平面的法向量平行。
\end{Itemize}
\footnotetext{事实上,这里不规定旋转,对于长度是没有影响的。这里这么说,主要是希望在介绍长度和方向时,这个角度可以统一。回想一下,在介绍极坐标系时,我们就规定逆时针是正方向了,所以这么规定可以合理地引出正负。}

如果这么说不够直观,我们可以参考右手定则:

\begin{singlefigure}[\En{Right-Hand Rule}]{chapter3_right_hand_rule}[0.6]
    以$\vec{A}$为起始,右手手指向$\vec{B}$弯曲(取$<\frac{\pi}{2}$方向),大拇指方向为$\vec{C}$的方向\\
    我们称$\vec{A}$叉乘$\vec{B}$等于$\vec{C}$,记作$\vec{A}\times\vec{B}=\vec{C}$
\end{singlefigure}

下面不加证明地给出叉乘的有关性质和定理:

\begin{Itemize}
    \item \itr{Anti-Commutative Law}{反交换律} $\vec{A}\times\vec{B}=-\vec{B}\times\vec{A}$,即交换两向量顺序,它们的夹角方向会取反。
    \item \itr{Distributive Law}{分配律} $\vec{A}\times(\vec{B}+\vec{C})=\vec{A}\times\vec{B}+\vec{A}\times\vec{C}$
    \item \itr{Derivative}{导数} $\dfrac{\dif}{\dif t}(\vec{A}\times\vec{B})=\dfrac{\dif\vec{A}}{\dif t}\times \vec{B}+\vec{A}\times\dfrac{\dif\vec{B}}{\dif t}$\labelroot{chapter3_derivation}\\
    \En{It is important to preserve the
        multiplicative order of A and B.}\\
    鉴于反交换率,保证叉乘的顺序正确是十分重要的。
    \item \itr{Lagrange's Identity}{拉格朗日恒等式} $(\vec{A}\times\vec{B})\cdot(\vec{C}\times\vec{D})=\left|\begin{array}{cc}
            \vec{A}\cdot\vec{C} & \vec{A}\cdot\vec{D} \\
            \vec{B}\cdot\vec{C} & \vec{B}\cdot\vec{D}
        \end{array}\right|$
    \item \itr{Double Cross Product}{二重叉乘}\labelroot{chapter3_double_cross_product}\begin{align*}
        (\vec{A}\times\vec{B})\times\vec{C} & =(\vec{A}\cdot\vec{C})\vec{B}-(\vec{B}\cdot\vec{C})\vec{A} \\
        \vec{A}\times(\vec{B}\times\vec{C}) & =(\vec{A}\cdot\vec{C})\vec{B}-(\vec{A}\cdot\vec{B})\vec{C}
    \end{align*}
    这里也可以发现,叉乘不满足结合律。
\end{Itemize}

有些同学初次看见那么多公式难免慌张。事实上,普通物理\Romannumeral{1}对于叉乘的考察非常有限,基本只要清楚定义和基本的运算律即可。
\subsection[运动学概念]{\itr{Concepts in Kinematics}{运动学概念}}
%\textit{为了保证分析的简单,除非特别说明,我们默认,选择的旋转轴方向总是与角速度方向平行}。

在定好一个柱坐标系后之后,确定刚体的角度矢量及其与时间的关系,就能确定刚体的转动。

\begin{Itemize}
    \item \itr{Angular displacement}{角位移}\mgnote{$\vec{\theta}$的单位rad属于数学单位,它是一个无量纲量。} $\Delta\vec{\theta}=\vec{\theta}_f-\vec{\theta}_i$
    \item \itr{Instantaneous angular velocity}{瞬时角速度} $\vec{\omega} = \dfrac{\dif\vec{\theta}}{\dif t}$
    \item \itr{Instantaneous angular acceleration}{瞬时角加速度}  $\vec{\alpha}=\dfrac{\dif\vec{\omega}}{\dif t}$
\end{Itemize}

\En{When rotating about
    a fixed axis, every
    particle on a rigid
    object rotates
    through the same
    angle and has the
    same angular speed
    and the same angular
    acceleration.}

平动与转动有如下关系\labelroot{chapter3_connection}:
\begin{Itemize}
    \item  \itr{arc length}{弧长} $s=r\theta$,此指圆周运动。
    \item $\vec{v}_t=\vec{\omega}\times\vec{r}$ \& $v_t=\omega r$,其中t下标指\itr{tangential}{切向},$\vec{v}_t$即同时垂直于$z$轴\footnotemark 和$\vec{r}$的速度分量。
    \item $\vec{a}_t=\vec{\alpha}\times\vec{r}$ \& $a_t=\alpha r$,$\vec{a}_t$即同时垂直于$z$轴和$\vec{r}$的加速度分量。
\end{Itemize}
\footnotetext{注意这里垂直于$z$轴的条件,我们考虑的是绕轴旋转,所以任何轴向的速度分量也不在我们研究转动的考虑之中。}

这里叉乘的先后顺序可能并不那么好记,推荐读者自己画一个圆周运动的图,用右手定则对应验证一遍如上关系,以加深理解与记忆。

\subsection[动力学概念]{\itr{Concepts in Dynamics}{动力学概念}}
在引入了叉乘(见\refleaftext{chapter3_cross_product})这一数学工具后,我们得以较为轻松地描述转动的动力学概念。

以下默认已经选定转轴\labelroot{chapter3_axis}。

\begin{Itemize}
    \item \itr{The Moment of Inertia}{转动惯量} $\displaystyle I=\sum_{i}m_ir_i^2$,其中把刚体看成由多个质点组合而成,$r_i$即是质点$m_i$的极径矢量大小。

    \En{The moment of inertia is a measure of the
        resistance of an object to changes in its rotational
        motion, just as mass is a measure of the tendency
        of an object to resist changes in its linear motion.}
    \item \itr{Torque}{力矩} $\vec{\tau}=\vec{r}\times\vec{F}_{/z}$,其中$\vec{r}$是力$\vec{F}$的作用点对应的极径矢量,$\vec{F}_{/z}=\vec{F}-\vec{F}_z$,$\vec{F}_z$即$\vec{F}$在$z$轴上的分矢量\labelroot{chapter3_lower/z}\footnotemark。
    \item \itr{Net Torque}{合外力矩} $\displaystyle\vec{\tau}_{net}=\sum_{i}\vec{\tau}_i$
    \item 类似牛顿第二定律,有:在惯性系中,$\vec{\tau}_{net}=I\vec{\alpha}$,即合外力矩等于刚体的转动惯量乘以角加速度。
\end{Itemize}
\footnotetext{事实上,力矩拥有两种定义,一种是对点定义的,为$\vec{r}\times\vec{F}$;另一种是对轴定义的,为$\vec{r}\times\vec{F}_{/z}$。鉴于我们使用了柱坐标系建立这些概念,这里选择对轴定义的力矩。}
\begin{singlefigure}[力矩示意图]{chapter3_torque}[0.5]
    在本图中,由于$\vec{F}$没有$z$轴上的分量,所以$\vec{F}=\vec{F}_{/z}$。\\
    事实上,普物\Romannumeral{1}中基本没有出现过$\vec{F}$有$z$轴分量的情况。
\end{singlefigure}
关于以上内容的证明,见\refleaftext{prove3.1}。

其中,为了便于计算力矩,我们可以引入 \itr{moment arm/force arm}{力臂} 的概念,其大小即为转轴到力的作用线的距离。有了这一点,力矩的大小就可以通过力臂乘以力得到。
\ctikzfig{chapter3_moment_arm}
\begin{center}
    \em $l_1$是$\vec{F}_1$的力臂,$l_2$是$\vec{F}_2$的力臂。
\end{center}

对于常见的物体,其转动惯量是应当记住的:
\begin{singlefigure}{chapter3_moment_inertia1}
\end{singlefigure}
\vspace*{-2ex}
\begin{singlefigure}[常见物体的转动惯量]{chapter3_moment_inertia2}
\end{singlefigure}
对于更一般的物体,则往往通过积分的方式求取它们的转动惯量。
\section[平行轴定理/施泰纳定理]{\itr{Parallel Axis Theorem/Steiner Theorem}{平行轴定理/施泰纳定理}}
在之前对转动的动力学概念的讨论中,我们默认转轴是选定的\mgnote{见\refleaftext{chapter3_axis}},这是因为,当转轴不同时,转动惯量与合力矩都会发生变化。平行轴定理是解决转轴变化过程中转动惯量变化的利器,其表述如下:

\begin{law}[\itr{Parallel Axis Theorem/Steiner Theorem}{平行轴定理/施泰纳定理}---\refleaftext{prove3.2}]
    记刚体$M$以过质心的轴(简称质心轴)为转动轴时的转动惯量为$I_{CM}$,以任意平行于质心轴的轴为转动轴时的转动惯量为$I$,这两条轴之间的距离为$h$,则有
    \[I=I_{CM}+Mh^2\]
\end{law}

幸于拥有平行轴公式,我们在背诵常见刚体的转动惯量时,一般只要记下它以质心轴为转动轴时的转动惯量即可。

\section[刚体中的能量]{\itr{Energy in a Rigid Body}{刚体中的能量}}
好比从牛顿三律进发到能量定律,我们将开始研究刚体中的能量。
\subsection[纯转动中的能量]{\itr{Energy in Rotation ONLY}{纯转动中的能量}}
首先,让我们思考一个只发生转动的刚体所带的能量(\refleaftext{prove3.3})。
\begin{Itemize}
    \item 动能定理 $K_R=\dfrac{1}{2}I\omega^2$
    \item 做功 $\dif W=\tau\dif\theta$
    \item 功率 $P=\tau\omega$
    \item 旋转势能 $U=\dfrac{1}{2}\kappa\theta^2$
\end{Itemize}
其中,旋转势能是指由弹性旋转体扭转回原位的趋势产生的,只依赖于其状态的能量,在数值上等于将旋
转体自原位扭转至当前位置所做的功。$\kappa$称为扭转常数,单位为$\mathrm{N}\cdot\mathrm{m}$。

当旋转体扭转了角度$\vec{\theta}$时,就会产生反力矩$\vec{\tau}=-\kappa\vec{\theta}$。
\subsection[同时发生转动与平动时的能量]{\itr{Energy in Rotation \& Translation}{同时发生转动与平动时的能量}}
现在,让我们思考一个问题:
\begin{center}
    \itshape 一个刚体在质心系中能够发生平动吗?
\end{center}

答案是否定的。如果刚体在质心系中发生平动,那么质心就会拥有平动速度,然而,在质心系中,质心又一直处于原点,这与质心拥有平动速度相矛盾,因此刚体在质心系中只能发生转动\labelroot{chapter3_rotation_CM}。

既然如此,我们便会有这样的想法:利用质心系,把同时发生平动和转动的物体转化成只发生转动的物体来研究。如此,便产生了柯尼希定理:

\begin{law}[\itr{Konig's Theorem}{柯尼希定理}---\refleaftext{prove3.4}]
    记一个系统$M$的质心速度为$v_{_{CM}}$,在质心系中的动能为$K^{CM}$,在静止参考系中的动能为$K$,并记$K_{CM}=\dfrac{1}{2}Mv_{_{CM}}{}^2$,则有
    \[K=K_{CM}+K^{CM}\]
    如果考虑对一个刚体使用柯尼希定理,则有:
    \[K=\dfrac{1}{2}Mv_{_{CM}}{}^2+\dfrac{1}{2}I_{CM}\omega^2\]
\end{law}

在研究一个同时发生平动和转动的物体的动能时,我们往往使用柯尼希定理,分别求解其质心的平动动能和在质心系中的转动动能。
\section[角动量]{\itr{Angular Momentum}{角动量}}
与平动时的动量类似,我们可以定义刚体的角动量。
\subsection[定义]{Definition}
\begin{Itemize}
    \item \itr{Angular Momentum}{角动量} $\displaystyle\vec{L}=\sum_i\vec{r}_i\times\vec{p}_{i/z}=I\vec{\omega}$,其中$\vec{p}_{i/z}=\vec{p}_i-\vec{p}_{iz}$,$ \vec{r}_i\times\vec{p}_{i/z}$应视作质点$m_i$的角动量。角动量的方向总是与角速度的方向一致\footnotemark。
    \item 类似动量,我们有$\dfrac{\dif \vec{L}}{\dif t}=\vec{\tau}_{net}$。
\end{Itemize}
以上内容证明见\refleaftext{prove3.5}。

\footnotetext{对于这句话需持谨慎态度。在本章所定义的概念体系中,这句话是成立的,但由于大多数对角动量的定义都是对点定义,即$\displaystyle\sum_i\vec{r}_i\times\vec{p}_i$,就出现了角动量方向与角速度方向不一致的说法。\textbf{当出现这类判断时,请读者优先考虑不一致的说法}。}
\subsection[角动量守恒]{\itr{Conservation of Angular Momentum}{角动量守恒}}
\begin{Itemize}
    \item \itr{Conservation of Angular Momentum}{角动量守恒}由角动量的性质,我们可以发现,当合外力矩的值为$\vec{0}$时,系统的角动量保持守恒。
    \[\vec{\tau}_{net}=0\Leftrightarrow\vec{L}=Constant\]
\end{Itemize}

利用角动量守恒,可以解释开普勒第二定律。
\begin{singlefigure}[\En{Kepler's Second Law}]{chapter3_kepler's_second_law}
    对$M_p$分析,有$\vec{\tau}_{net}=\vec{r}\times\vec{F}_g=\vec{0}$,因此角动量守恒\\[1ex]
    $\dfrac{\dif A}{\dif t}=\dfrac{1}{2}|\vec{r}\times\vec{v}|=\dfrac{L}{M_p}\Rightarrow$单位时间扫过面积相等
\end{singlefigure}
\En{When a force is directed toward or away from a
    fixed point and is function of $\vec{r}$ only, it is called a
    \itr{Central Force}{中心力}.}

总是可以选择合适的轴,使得中心力产生的力矩为$\vec{0}$。

花样滑冰时选手收紧胳膊旋转较快,而张开双臂后旋转变
慢,这也是角动量守恒的功劳。你可能会发现,收紧胳膊时,人的动能变大了,这是生物能的功劳。
\subsection[质心系转化]{\itr{Center of Frame Translation}{质心系转化}}
角动量在静止参考系和质心系间的转化与柯尼希定理类似。
\begin{law}[\itr{Angular Momentum Translation Using C.M. Frame}{利用质心系的角动量变换}\footnotemark---\refleaftext{prove3.6}]
    记质心角动量为$\vec{L}_{CM}=M\vec{r}_{_{CM}}\times\vec{v}_{_{CM}}$,刚体在质心系中的角动量为$\vec{L}^{CM}=\sum_im_i\vec{r}_i^{^{CM}}\times\vec{v}^{^{CM}}_i$,刚体在静止参考系中的角动量为$\vec{L}$,且保证两个参考系中选取的轴平行,则有
    \[\vec{L}=\vec{L}_{CM}+\vec{L}^{CM}\]
\end{law}
\footnotetext{对于对点定义的角动量,也有类似的结论,证明方法也基本相同。}
\section[重心]{\itr{Center of Gravity}{重心}}
我们会注意到,在分析重力产生的合力矩时,依据定义,应该对每一个质元求分力矩,然后进行求和。这似乎是一件复杂的事情。有没有办法寻找一个点,使得刚体的合重力在此作用时,得到的力矩恰等于重力的合力矩呢?事实上,我们把这样的点称作重心。

一个重要的结论是:
\begin{center}
    {\itshape 对于均匀的重力场,刚体的重心与质心重合(\refleaftext{prove3.7})}。
\end{center}

至于不均匀的情况,就需要依据定义进行计算了。
\section[非惯性系情形]{\itr{Non-Inertial Frame Situation}{非惯性系情形}}
\subsection[非惯性力]{\itr{Inertial Force}{惯性力}}
由于转动动力学的所有定律都是基于牛顿运动定律,结合数学方法推出的,因此,对于一个拥有加速度$\vec{a}_{frame}$的平动非惯性参考系\labelroot{chapter3_translation},我们需要在分析其中的物体时添上惯性力$\vec{F}_{fictious}=-m\vec{a}_{frame}$,其中$m$为物体的质量。此时,牛顿运动定律保持成立,转动动力学中的定律也保持成立。

\subsection[科里奥利力*]{\itr{Coriolis Force}{科里奥利力}*}
你可能注意到了,之前的表述中,我们讲的是“拥有加速度$\vec{a}_{frame}$的平动非惯性参考系”\mgnote{见\refleaftext{chapter3_translation}}。这暗示着,如果参考系还发生着转动,情况又会有所不同。

对于一个仅发生转动的参考系,有如下定理:
\begin{law}[\itr{Frame Translation with Rotation}{考虑转动的参考系变换}$^*$---\refleaftext{prove3.8}]
    如果希望牛顿运动定律在一个转动参考系$\mathcal{F}'$中依然成立,我们需要对$\mathcal{F}'$中的物体添加三个假想力,分别是$-2m\vec{\omega}\times\vec{v}',-m\vec{\alpha}\times\vec{r}',m\omega^2\vec{r}'$。

    其中,$\vec{\omega}$是$\mathcal{F}'$的角速度,$\vec{\alpha}$是$\mathcal{F}'$的角加速度,$\vec{v}'$是物体在$\mathcal{F}'$中的速度,$\vec{r}'$是物体在$\mathcal{F}'$中的极径矢量。

    如果我们将条件简化,认为$\mathcal{F}'$没有角加速度,则无需考虑$-m\vec{\alpha}\times\vec{r}'$。

    我们称$-2m\vec{\omega}\times\vec{v}'$为 \itr{Coriolis Force}{科里奥利力},$m\omega^2\vec{r}'$为 \itr{Centrifugal Force}{离心力}。
\end{law}
\begin{singlefigure}[地球的科里奥利力]{chapter3_earth}[0.42]
    如果考虑地球的科里奥利力,我们会发现,在北半球上,\\科里奥利
    力永远指向物体运动方向的右侧,而在南半球正好相反。
\end{singlefigure}
台风的旋转方向在北半球是逆时针,在南半球是顺时针,这也是科里奥利力的影响。

{\bfseries 科里奥利力是一种假想力}\footnote{我们有时会看见类似“科里奥利力是由于地球自转而施加给地球上的物体的力”的说法,这么讲可能是因为我们往往以地球为参考系分析问题,此时要想保证牛顿第二定律成立,就要认为物体受到科里奥利力。不过,一般而言,科里奥利力较小,在日常生活的分析中大多可忽略不计。}。
\section[回顾与总结]{Review and Summary}
在我们的推导中,我们发现,转动力学的结论与平动力学有着惊人的相似性。关注这份相似性既能体会物理的美感,又能加深对结论的记忆。
\[\ \ \begin{aligned}
         & \mathrm{Angular~speed~}\vec{\omega}=\dif\vec{\theta}/\dif t                                                                                                          &   & \mathrm{Linear~speed~}\vec{v}=\dif \vec{x}/\dif t         \\
         & \mathrm{Angular~acceleration~}\vec{\alpha}=\dif\vec{\omega}/\dif t                                                                                                   &   & \mathrm{Linear~acceleration~}\vec{a}=\dif \vec{v}/\dif t  \\
         & \mathrm{Resultant~torque~}\sum\vec{\tau}=I\vec{\alpha}                                                                                                               &   & \mathrm{Resultant~force~}\sum \vec{F}=m\vec{a}            \\
         & \begin{array}{l}{{\mathrm{If}}}\\{\vec{\alpha}=\mathrm{constant}}\left\{\begin{array}{l}{{\vec{\omega}_{f}=\vec{\omega}_{i}+\vec{\alpha} t}}\\{{\vec{\theta}_{f}-\vec{\theta}_{i}=\vec{\omega}_{i}t+\frac{1}{2}\vec{\alpha} t^{2}}}\\{{\omega_{f}{}^{2}=\omega_{i}{}^{2}+2\alpha(\theta_{f}-\theta_{i})}}\end{array}\right.\end{array}
         &                                                                                                                                                                      &
        \begin{array}{l}{{\mathrm{If}}}\\{\vec{a}=\mathrm{constant}}\left\{\begin{array}{l}{{\vec{v}_{f}=\vec{v}_{i}+\vec{a}t}}\\{{\vec{x}_{f}-\vec{x}_{i}=\vec{v}_{i}t+\frac{1}{2}\vec{a}t^{2}}}\\{{v_{f}{}^{2}=v_{i}{}^{2}+2a(x_{f}-x_{i})}}\end{array}\right.\end{array}                                                                         \\
         & \mathrm{Work}\ W=\int_{\theta_{i}}^{\theta_{f}}\tau \dif\theta                                                                                                       &   & \mathrm{Work}\ W=\int_{x_{i}}^{x_{f}}F_{x}\dif x          \\
         & \mathrm{Rotational~kinetic~energy~}K_{\mathrm{R}}=\frac{1}{2}I\omega^{2}                                                                                             &   & \mathrm{Kinetic~energy~}K=\frac{1}{2}mv^{2}               \\
         & \mathrm{Power}\ P=\vec{\tau}\cdot\vec{\omega}                                                                                                                        &   & \mathrm{Power}\ P=\vec{F}\cdot\vec{v}                     \\
         & \mathrm{Angular~momentum~}\vec{L}=I\vec{\omega}                                                                                                                      &   & \mathrm{Linear~momentum~}\vec{p}=m\vec{v}                 \\
         & \mathrm{Resultant~torque~}\sum\vec{\tau}=\dif \vec{L}/\dif t                                                                                                         &   & \mathrm{Resultant~force~}\sum \vec{F}=\dif \vec{p}/\dif t
    \end{aligned}\]











