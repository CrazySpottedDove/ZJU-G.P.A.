\chapter[转动动力学]{\itr{Rotation Dynamics}{转动动力学}}
\begin{solution}[{Determine the Moment of Inertia of an Irregularly Shaped Object}]
    This problem describes one experimental method of
    determining the moment of inertia of an irregularly
    shaped object such as the \itr{payload}{有效负载} for a satellite.
    \begin{center}
        \thisincludegraphics[width=0.6\textwidth]{chapter3_example_1}
    \end{center}

    The figure shows a mass $m$ \itr{suspended}{悬挂} by a \itr{cord}{绳子} \itr{wound}{被缠绕}
    around a spool of radius $r$, forming part of a \itr{turntable}{转台}
    supporting the object. When the mass is released from
    rest, it \itr{descends}{下降} through a distance $h$, acquiring a speed
    $\vec{v}$. Show that the moment of inertia $I$ of the equipment
    (including the turntable) is\quad$mr^2(\dfrac{2gh}{v^2}-1)$.

    \tcbrule

    这是一道基础的转动力学题目,一般的求解思路为:列力学方程---列运动方程---列关联方程---求解。

    设绳子的张力为$\vec{T}$,物体的转动惯量为$I$,有:
    \[\left\{
        \begin{array}{cc}
            \left.\begin{array}{c}
                      Tr=I\alpha \\
                      mg-T=ma
                  \end{array}\right\} & \cdots\text{力学方程} \\
            v^2=2ah                & \cdots\text{运动方程}    \\
            a=r\alpha              & \cdots\text{关联方程}
        \end{array}
        \right.\]
    联立求解即得\[I=mr^2(\dfrac{2gh}{v^2}-1)\]
\end{solution}
\begin{solution}[{Rolling Items}]
    Three objects of \itr{uniform density}{均匀的密度} --- a \itr{solid sphere}{实心球}, a \itr{solid
        cylinder}{实心圆柱}, and a \itr{hollow cylinder}{空心圆柱} --- are placed at the top of
    an \itr{incline}{斜坡}.
    \begin{center}
        \thisincludegraphics[width=0.6\textwidth]{chapter3_example_2}
    \end{center}
    If they all are released from rest
    at the same \itr{elevation}{高度} and roll without \itr{slipping}{滑动}, which object reaches the bottom first?

    \tcbrule

    本题是经典的纯滚动问题。所谓纯滚动,就是滚动体与接触面接触的点速度为$\vec{0}$(即旋转速度和质心速度相抵消)。

    我们往往选择过质心的轴作为旋转轴,建立参考系。这是因为,即使这样建立的参考系是一个非惯性系,只要它不发生转动,那么,使用证明“均匀重力场中重力可以等效作用在质心”中用到的方法(\refleaftext{prove3.7}),就可以证明惯性力可以等效作用在质心。由于我们选择的是质心轴,惯性力产生的力矩恒为$0$,因此也就不会对转动的分析产生影响\footnote{很多解析都直接选择了一个非惯性质心系来分析转动,却并没有讲解惯性力可以忽略的原因。}。

    现在,让我们选择质心系分析问题\footnote{当然,过质心的轴有很多,但是大家应该能意会到选择了怎样的轴(对称性好的轴),就不再描述了}。

    若记球体$m_1$的转动惯量为$I_1$,实心圆柱$m_2$的转动惯量为$I_2$,空心圆柱$m_3$的转动惯量为$I_3$,斜面倾角为$\theta$,则有:
    \[\left\{
        \begin{array}{l}
            m_1a_1=m_1g\sin\theta-f_1 \\
            m_2a_2=m_2g\sin\theta-f_2 \\
            m_3a_3=m_3g\sin\theta-f_3 \\
            f_1r_1=I_1\alpha_1        \\
            f_2r_2=I_2\alpha_2        \\
            f_3r_3=I_3\alpha_3        \\
            a_1=r_1\alpha_1           \\
            a_2=r_2\alpha_2           \\
            a_3=r_3\alpha_3
        \end{array}
        \right.\]
    于是解得
    \[\left\{
        \begin{array}{l}
            a_1=\dfrac{m_1r_1{}^2}{I_1+m_1r_1{}^2}g\sin\theta \\[2ex]
            a_2=\dfrac{m_2r_2{}^2}{I_2+m_2r_2{}^2}g\sin\theta \\[2ex]
            a_3=\dfrac{m_3r_3{}^2}{I_3+m_3r_3{}^2}g\sin\theta
        \end{array}
        \right.\]
    由常见物体的转动惯量\mgnote{见\refleaftext{chapter3_moment_inertia2}}知
    \begin{align*}
        I_1 & =\dfrac{2}{5}m_1r_1{}^2                    \\
        I_2 & =\dfrac{1}{2}m_2r_2{}^2                    \\
        I_3 & =\dfrac{1}{2}m_3(r_{3(inner)}{}^2+r_3{}^2)
    \end{align*}
    易知
    \[a_1>a_2>a_3\]
    故实心球快于实心圆柱快于空心圆柱。
\end{solution}

\begin{solution}[Calculation of the Moment of Inertia]
    A rod's \itr{linear density}{线密度} is given by $\lambda=kx$, where $x$ represents the distance from the point to the rod's center.
    \ctikzfig{chapter3_example_3_3}
    Given the length of the rod $L$, try to calculate the moment of inertia of the rod, given the rotation axis at:

    (1) Center $O$ as the $y$ axis shows.

    (2) One end as the $y'$ axis shows.

    \tcbrule

    (1)由于$O$也是质心位置,不妨先求$I_{CM}$,再利用平行轴定理求解$I_{end}$。
    注意到
    \[\dif m=\lambda\dif x=k|x|\dif x\]
    于是
    \begin{align*}
        \int_{-\frac{L}{2}}^{\frac{L}{2}}(x^2)(k|x|)\dif x & =2\int_0^{\frac{L}{2}}kx^3\dif x                  \\
                                                           & =2\left.(\dfrac{1}{4}kx^4)\right|_0^{\frac{L}{2}} \\
                                                           & =\dfrac{1}{32}kL^4
    \end{align*}

    (2)欲用平行轴定理(\refleaftext{law3.1}),则需知晓棍子的质量。
    \begin{align*}
        m & =\int_{-\frac{L}{2}}^{\frac{L}{2}}k|x|\dif x      \\
          & =2\int_0^{\frac{L}{2}}kx\dif x                    \\
          & =2\left.(\dfrac{1}{2}kx^2)\right|_0^{\frac{L}{2}} \\
          & =\dfrac{1}{4}kL^2
    \end{align*}
    故
    \[I_{end}=I_{CM}+m(\dfrac{L}{2})^2=\dfrac{3}{32}kL^4\]
\end{solution}
\begin{solution}[\En{ Massive \itr{Pulley}{滑轮}}]
    Consider two \itr{cylinders}{圆柱} having masses $m_1$
    and $m_2$, where $m_1 < m_2$, connected by a
    string passing over a pulley. The pulley
    has a radius $R$ and moment of inertia $I$
    about its axis of rotation.
    \begin{singlefigure}{chapter3_massive_pulley}[0.3]
        The string does
        not \itr{slip}{滑动} on the pulley, and the system
        is released from rest.
    \end{singlefigure}
    Find the linear
    speeds of the cylinders after
    cylinder 2 \itr{descends}{下降} through a
    distance $h$, and the angular
    speed $\omega$ of the pulley at this time.

    \tcbrule

    本题可从运动角度或能量角度考虑。\\
    法一:运动分析\\
    设左绳的张力为$T_1$,右绳的张力为$T_2$,则有
    \[\left\{\begin{array}{c}
            T_1-m_1g=m_1a     \\
            T_2R-T_1R=I\alpha \\
            m_2g-T_2=m_2a     \\
            a=R\alpha
        \end{array}\right.\]
    于是解得
    \[a=\dfrac{m_2-m_1}{m_1+m_2+\frac{I}{R^2}}g\]
    则
    \[v=\sqrt{2ah}=\sqrt{\dfrac{m_2-m_1}{m_1+m_2+\frac{I}{R^2}}2gh},\quad\omega=\sqrt{\dfrac{m_2-m_1}{(m_1+m_2)R^2+I}2gh}\]
    法二:能量分析\\
    系统机械能守恒,于是有
    \[\left\{\begin{array}{c}
            m_2gh=m_1gh+\dfrac{1}{2}m_1v^2+\dfrac{1}{2}m_2v^2+\dfrac{1}{2}I\omega^2 \\
            v=R\omega
        \end{array}\right.\]
    亦可解得
    \[v=\sqrt{2ah}=\sqrt{\dfrac{m_2-m_1}{m_1+m_2+\frac{I}{R^2}}2gh},\quad\omega=\sqrt{\dfrac{m_2-m_1}{(m_1+m_2)R^2+I}2gh}\]
\end{solution}
\begin{solution}[\En{ Object rotating on a string
            of changing length}]
    Initially, the mass \itr{revolves}{转动} with a speed $v_1$ = 2.4 m/s in
    a circle of radius $R_1$ = 0.80 m.
    The string is then pulled slowly through the hole so
    that the radius is reduced to $R_2$ = 0.48 m. What is the
    speed, $v_2$, of the mass now?
    \begin{singlefigure}{chapter3_string}[0.7]
    \end{singlefigure}

    \tcbrule

    本题考察角动量守恒。可以注意到,绳子对物块的力始终是径向的,对应的力矩始终为$\vec{0}$,因此,物块的角动量守恒。不妨就以洞为轴,有
    \[I_1\omega_1=I_2\omega_2\Rightarrow R_1{}^2\omega_1=R_2{}^2\omega_2\]
    代入$\omega_1=\dfrac{v_1}{R_1},\quad\omega_2=\dfrac{v_2}{R_2}$,即得
    \[v_1R_1=v_2R_2\]
    代入数据即得$v_2=4.0$m/s.
\end{solution}

\begin{solution}[\En{ Rotation of a sliding rigid rod}]
    Consider a rod with mass m and length L standing straight on the friction-less ground. When we
    release the rod, it will fall from the unstable \itr{equilibrium position}{平衡位置}.
    \begin{center}
        \tikzfig{chapter3_rotating_rod_1}
        \quad
        \tikzfig{chapter3_rotating_rod_2}
    \end{center}
    (a) Calculate the angular velocity of the rod, when it has an angle of $\theta$ with respect to the ground
    as illustrated in Figure 1.\\
    (b) What is the final angular velocity $\omega_1$ of the rod before it hits the ground?\\
    (c) If the same rod is leaning to a frictionless wall with an initial angle of α to the frictionless
    ground (see Figure 2), what is the final angular velocity $\omega_2$ of the rod before it hits the ground?\\
    {\em Note that there is a possibility that the right end of the rod leaves from the wall before the rod hits the ground.}

    \tcbrule

    本题主要考察转动中的能量守恒,以及对平动速度和角速度关系的分析。

    (a) 首先,由题意知不存在摩擦力,而支持力做功始终为零,所以以棍子为研究对象,有机械能守恒。又注意到,棍子在水平方向始终不受力,因此质心是在垂直下降。于是有
    \begin{equation}
        mg\dfrac{L}{2}-mg\dfrac{L}{2}\sin\theta=\dfrac{1}{2}mv_{_{CM}}{}^2+\dfrac{1}{2}I\omega^2
    \end{equation}
    研究质心的运动,有
    \begin{align}
        v_{_{CM}} & =\dfrac{\dif (\dfrac{L}{2}\sin\theta)}{\dif t}                            \\[1ex]
                  & =\dfrac{L}{2}\dfrac{\dif\sin\theta}{\dif\theta}\dfrac{\dif\theta}{\dif t} \\[1ex]
                  & =\dfrac{L}{2}\cos\theta\omega
    \end{align}
    将(3.4)代入(3.1)即得
    \begin{equation}
        \omega=2\sqrt{\dfrac{3g}{L}\dfrac{1-\sin\theta}{1+3\cos^2\theta}}
    \end{equation}
    (b) 即考虑(a)中的极限情况,将$\theta=0$代入(3.5)中即得
    \begin{equation}
        \omega_1=\sqrt{\dfrac{3g}{L}}
    \end{equation}
    (c) \begin{center}
        \begin{tikzpicture}[scale=2]
            \coordinate (O) at (0,0);
            \coordinate (A) at (-1,0);
            \coordinate (B) at (0,1.732);
            \coordinate (D) at (-1.5,0);
            \coordinate (E) at (0,2.5);
            \coordinate (BB) at (0,1.414);
            \coordinate (AA) at (-1.414,0);

            \draw [red](A) -- (B) ;
            \draw [thick=3pt](D)--(O);
            \draw [thick=3pt](O)--(E);
            \draw [red,dashed](AA)--(BB);

            \coordinate (M) at ($(A)!0.5!(B)$);
            \coordinate (N) at ($(AA)!0.5!(BB)$);

            \draw[dashed,plainred] (O) -- (M);
            \draw[dashed,plainred] (O)--(N);
            \draw[dashed,yellow5] (0,0) circle[radius=1];

            \node[below left] at (A) {$A$};
            \node[above right] at (B) {$B$};
            \node[above right] at (BB) {$B'$};
            \node[below left] at(AA) {$A'$};
            \node[above left] at (M) {$C$};
            \node[below left] at (O) {$O$};
            \node[above right]at (A) {$\ \ \ \alpha$};
            \node[above left] at (N) {$C'$};
            \draw (-0.8,0) arc(0:60:0.2);
            \draw (-1.214,0) arc (0:45:0.2);
            \node[above right] at (AA) {$\ \ \ \,\theta$};
        \end{tikzpicture}
    \end{center}

    如图,先考虑棍未与墙壁脱离情形,注意到质心$C$到$O$的距离始终为$\dfrac{L}{2}$,因此确定质心的运动轨迹是一个圆。由$\angle COA=\angle CAO$,知$v_{_{CM}}=\omega\dfrac{L}{2}$,即关系式。再由\\[1ex]“无摩擦力”,“支持力不做功”知棍子机械能守恒,于是可以列出守恒式:
    \begin{equation}
        mg(\dfrac{L}{2}\sin\alpha-\dfrac{L}{2}\sin\theta)=\dfrac{1}{2}mv_{_{CM}}{}^2+\dfrac{1}{2}I\omega^2
    \end{equation}
    由(3.7)可解得
    \begin{equation}
        \omega=\sqrt{\dfrac{3g(\sin\alpha-\sin\theta)}{L}}
    \end{equation}
    接下来,我们考虑临界条件。当棍子脱离墙时,来自墙的支持力消失,也就是说,\textbf{质心在水平方向不再拥有加速度}。于是,$v_{_{CM}x}$最大时,棍子将脱离墙。
    \begin{align}
        v_{_{CM}x} & =v_{_{CM}}\sin\theta                                                                                           \\
                   & =\omega\dfrac{L}{2}\sin\theta                                                                                  \\
                   & =\dfrac{\sin\theta}{2}\sqrt{3gL(\sin\alpha-\sin\theta)}                                                        \\
                   & =\sqrt{3gL}\cdot\sqrt{\dfrac{\sin\theta}{2}}\cdot\sqrt{\dfrac{\sin\theta}{2}}\cdot\sqrt{\sin\alpha-\sin\theta} \\
                   & \le\dfrac{1}{3}\sin\alpha\sqrt{gL\sin\alpha}\quad(\text{当且仅当}\sin\theta=\dfrac{2}{3}\sin\alpha)
    \end{align}
    之后,在水平方向,质心的运动保持不变。由(3.4)知,当棍子即将落地时,有\begin{equation}
        v_{_{CM}y}=\dfrac{L}{2}\omega_2
    \end{equation}于是可以列守恒式
    \begin{equation}
        mg\dfrac{L}{2}\sin\alpha=\dfrac{1}{2}m(v_{_{CM}x}{}^2+v_{_{CM}y}{}^2)+\dfrac{1}{2}I\omega_2{}^2
    \end{equation}
    将(3.14)代入(3.15)即得
    \begin{equation}
        \omega_2=\sqrt{(9\sin\alpha-\sin^3\alpha)\dfrac{g}{3L}}
    \end{equation}
    \dove\ PS:这大概是本章考察的天花板了。
\end{solution}
