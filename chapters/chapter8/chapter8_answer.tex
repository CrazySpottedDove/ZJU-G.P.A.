\chapter[电磁学]{\itr{Electromagnetism}{电磁学}}
\begin{solution}[Polarization]
    A positive point charge $Q$ is located at the center of a spherical shell dielectric with an inner radius of $R_i$ and an outer radius of $R_o$, 
    where the dielectric constant is $\varepsilon_r$. Determine the functions $\vec{E}$, $V$, $\vec{D}$, and $\vec{P}$ as a function of the radial distance $R$.
    \tcbrule

    本题目是高斯定律的简单应用。取以该点电荷为中心的一系列同心球面作为高斯面,无限远处电势为0。

    (1)$R > R_o$时,我们易得:
    \[E_{R1} = \dfrac{Q}{4\pi \varepsilon_0 R^2}\]
    \[V_1 = \dfrac{Q}{4\pi \varepsilon_0 R}\]
    \[D_{R1} = \varepsilon_0 E = \dfrac{Q}{4\pi R^2}\]
    
    在真空中,极化矢量$\vec{P} = 0$
    
    (2)$R_i < R < R_o$时:
    \[E_{R2} = \dfrac{Q}{4\pi \varepsilon_0\varepsilon_r R^2}\]
    \[D_{R2} = \dfrac{Q}{4\pi R^2}\]
    
    由关系式$\vec{D} = \varepsilon_0\vec{E} + \vec{P}$可得$\vec{P} = \varepsilon_0(\varepsilon_r - 1)\vec{E}$,即:
    \[P = (1-\dfrac{1}{\varepsilon_r})\dfrac{Q}{4\pi R^2}\]

    在与外部的交界处电势应当连续,取积分可得:
    \[V_2 = V_1|_{R=R_o} - \int_{R_o}^{R}\dfrac{Q}{4\pi \varepsilon_0\varepsilon_r R^2}\dif R = \dfrac{Q}{4\pi \varepsilon_0}[(1-\dfrac{1}{\varepsilon_r})\dfrac{1}{R_o} + \dfrac{1}{\varepsilon_r R}]\]
    (3)$R < R_i$时:
    \[E_{R3} = \dfrac{Q}{4\pi \varepsilon_0 R^2}\]
    \[D_{R3} = \varepsilon_0 E = \dfrac{Q}{4\pi R^2}\]
    \[\vec{P} = 0\]

    采用类似方法,可得:
    \[V_3 = V_2|_{R = R_i} - \int_{R_i}^{R}\dfrac{Q}{4\pi \varepsilon_0 R^2}\dif R = \dfrac{Q}{4\pi \varepsilon_0}[(1-\dfrac{1}{\varepsilon_r})\dfrac{1}{R_o} - (1-\dfrac{1}{\varepsilon_r})\dfrac{1}{R_i} + \dfrac{1}{R}]\]

    综上图像如下:
    \begin{singlefigure}{chapter8_answer1.png}[0.7]
    \end{singlefigure}
\end{solution}

\begin{solution}[Current Density/Boundary Condition]
    As shown in Figure bellow. An electric field is applied between the plates of a parallel plate capacitor with an area \( S \). The space between the two metal plates 
    is filled with two dielectric materials of different thicknesses \( d_1 \) and \( d_2 \), with dielectric constants \( \varepsilon_1 \) and \( \varepsilon_2 \), 
    and conductivities \( \sigma_1 \) and \( \sigma_2 \). Determine the following:
    \begin{singlefigure}{chapter8_exercise电流密度边界条件.png}[0.7]
    \end{singlefigure}
    
    (a) The current density between the two parallel plates.\\
    (b) The electric field strength in the two dielectric materials.\\
    (c) The surface charge density on the two parallel plates and at the interface.
    \tcbrule

    (a)忽略边界效应,考虑稳恒电流。稳态下平行板间电流密度是不变的包括介质交界面处,因此电流也是不变的。因此有:
    \[V = (R_1 + R_2)I\]
    \[R_1 = \dfrac{d_1}{\sigma_1 S}\qquad R_2 = \dfrac{d_2}{\sigma_2 S}\]
    \[J = \dfrac{I}{S}\]
    
    可解得:
    \[J = \dfrac{I}{S} = \dfrac{V}{d_1/\sigma_1 + d_2/\sigma_2} = \dfrac{\sigma_1\sigma_2 V}{\sigma_2 d_1 + \sigma_1 d_2}\]
    
    (b)由$\vec{J}=\sigma\vec{E}$可得:
    \[E_1 = \dfrac{\sigma_2 V}{\sigma_2 d_1 + \sigma_1 d_2}\]
    \[E_2 = \dfrac{\sigma_1 V}{\sigma_2 d_1 + \sigma_1 d_2}\]

    当然,该结果亦可由$V = E_1 d_1 + E_2 d_2$和边界条件$\sigma_1 E_1 = \sigma_2 E_2$联立解得

    (c)取上极板、下极板、界面处三个平面,将空间分为四个区域三个交界面,分别使用边界条件$\varepsilon_2 E_2 - \varepsilon_1 E_1 = \rho_s$

    对于上下极板,其另一侧的电场为0,因此:
    \[\rho_{s1} = \varepsilon_1 E_1 = \dfrac{\varepsilon_1\sigma_2 V}{\sigma_2 d_1 + \sigma_1 d_2}\]
    \[\rho_{s2} = -\varepsilon_2 E_2 = -\dfrac{\varepsilon_2\sigma_1 V}{\sigma_2 d_1 + \sigma_1 d_2}\]

    交界面处:
    \[\rho_{si} = \dfrac{(\varepsilon_2\sigma_1 - \varepsilon_1\sigma_2)V}{\sigma_2 d_1 + \sigma_1 d_2}\]
\end{solution}

\begin{solution}[Magnetic Field]
    As shown in Fig.a. A toroidal iron core with a relative magnetic permeability $\mu_r = 3000$, an average radius \( R = 80 \) mm, and a circular cross-sectional 
    radius \( b = 25 \) mm. The length of the air gap \( l_g = 3 \) mm. A winding of $N=500$ turns produces a magnetic flux \(\Phi =  10^{-5} \) Wb. 
    Neglecting magnetic leakage and using the average path length, calculate:
    \begin{minipage}{0.45\textwidth}
        \centering
        \begin{singlefigure}[a]{chapter8_exercise磁场.png}[0.99]
        \end{singlefigure}
    \end{minipage}
    \begin{minipage}{0.45\textwidth}
        \centering
        \begin{singlefigure}[b]{chapter8_exercise磁阻.png}[0.99]
        \end{singlefigure}
    \end{minipage}
    
    (a) The magnetic induction intensity \( B_g \) and magnetic field strength \( H_g \) in the air gap, and \( B_c \) and \( H_c \) in the iron core\\
    (b) The required current \( I \)\\
    (c) Similar to circuits and resistors, there are also concepts of magnetic circuits and \itr{magnetic resistance}{magnetic resistance} in magnetism. The latter is a constant 
    independent of the number of turns in a coil and is only related to the object itself. Magnetic resistance characterizes the hindrance to the magnetic flux passing through the magnetic 
    circuit, and its unit is $H^{-1}$. Given that the scaling coefficient in its expression is 1(with no additional constant factor).Taking Fig.a as an example, calculate the magnetic 
    resistance of the iron core \( R_c \) and air gap \( R_g \) respectively, and denote the magnetic flux by $N$, $I$, $R_g$ and $R_c$
    (hint: The unit of inductance is $H$)\\
    (d) Consider the magnetic circuit in Fig.b. Two windings with turns $N_1$ and $N_2$ are wound on the two side limbs of the ferromagnetic core. The cross-sectional area of the 
    core is $S$, and the permeability is $\mu$. The length of the left, center and right limb is \(l_1, l_2, l_3\) respectively. Determine the magnetic flux in the center limb.
    (hint: Observe the expression of magnetic flux in (c) and attempt to extend Kirchhoff's law to magnetic circuits)
    \tcbrule
    注意:本题目(c)(d)小问涉及新定义,旨在运用和探究已知内容,不涉及为具体考点。

    (a)由磁通量定义式易得:
    \[B_g = B_c = \dfrac{\Phi}{S} = \dfrac{10^{-5}}{\pi \times 0.025^2} = 5.09\times 10^{-3} (T)\]

    由磁感应强度与磁场强度的关系,可得:
    \[H_g = \dfrac{B_g}{\mu_0} = 4.057 \times 10^{3} (A/m)\]
    \[H_c = \dfrac{B_c}{\mu_0\mu_r} = 1.35 (A/m)\]

    (b)以完整的一圈作为积分路径,可得:
    \[\oint H\dif l = \sum I\quad\Rightarrow\quad H_g l_g + H_c l_c = NI\]
    \[l_c = 2\pi \times 0.08 - 0.003\]

    解得:
    \[I = 25.6 mA\]

    (c)电感的定义式为$L = \dfrac{N\Phi}{I}$,磁阻的物理量纲为其倒数,且与匝数无关,考虑量纲$\dfrac{NI}{\Phi}$。比例系数为1,不考虑其他的常数系数。

    进一步拆分,该物理量与电阻类似,仅与物体本身有关,已知$\Phi = BS$,考虑通过$\oint H\dif l = \sum I$将分子进行拆分以抵消$B$,可得:
    \[\dfrac{NI}{BS} = \dfrac{H_g l_g + H_c l_c}{BS} =\dfrac{H_g l_g}{BS} + \dfrac{H_c l_c}{BS} = \dfrac{l_g}{\mu_0 S} + \dfrac{l_c}{\mu_0\mu_r S}\]

    可得表达式$R = \dfrac{l}{\mu S}$,这是与电阻表达式$R = \dfrac{l}{\sigma S}$形式一致的物理量。代入数据求得:
    \[R_g = 1.21\times 10^{6} H^{-1}\]
    \[R_c = 6.75\times 10^{4} H^{-1}\]

    结合上面求得的结果$\dfrac{NI}{\Phi} = R_g + R_c$可得关系:
    \[\Phi = \dfrac{NI}{R_g + R_c}\]
    
    (d)考虑与电路的对应,电路中关注电阻电压电流,根据关系$NI = \Phi R$,$R$对应电阻,则$NI$对应电压,$\Phi$对应电流。

    简化模型为磁路,如下:
    \begin{singlefigure}{chapter8_answer3.png}[0.7]
    \end{singlefigure}

    三条路径的磁阻分别为:
    \[R_1 = \dfrac{l_1}{\mu S}\]
    \[R_2 = \dfrac{l_2}{\mu S}\]
    \[R_3 = \dfrac{l_3}{\mu S}\]

    仿照KVL定律列写回路方程:
    \[N_1I_1 = (R_1 + R_3)\Phi_1 + R_1\Phi_2\]
    \[N_1 I_1 - N_2 I_2 = R_1\Phi_1 + (R_1 + R_2)\Phi_2\]

    解得:
    \[\Phi_1 = \dfrac{R_2 N_1 I_1 + R_1 N_2 I_2}{R_1 R_2 + R_2 R_3 + R_1 R_3}\]
\end{solution}

\begin{solution}[Poynting Vector]
    A long and straight cylindrical wire with a radius of \(b \) and an electrical conductivity of \(\sigma \) carries a direct current of \(I \), and the current is uniformly distributed 
    in the wire, as shown in the following figure. Find the Poynting vector on the surface of a cylindrical wire and prove the Poynting theorem: The surface integral of the Poynting vector 
    on a closed surface is equal to the power radiated by the volume enclosed by this closed surface.

    \begin{minipage}{0.45\textwidth}
        \centering
        \begin{singlefigure}[a]{chapter8_exercise坡应廷矢量1.png}[0.99]
        \end{singlefigure}
    \end{minipage}
    \begin{minipage}{0.45\textwidth}
        \centering  
        \begin{singlefigure}[b]{chapter8_exercise坡应廷矢量2.png}[0.99]
        \end{singlefigure}
    \end{minipage}
    \tcbrule

    由关系$\vec{J} = \sigma \vec{E}$以及定义式$J = \dfrac{I}{S}$可得:
    \[E = \dfrac{I}{\sigma S} = \dfrac{I}{\sigma\pi b^2}\]

    在导体表面取平行于截面的一圈作为环路:
    \[H = \dfrac{I}{2\pi b}\]

    由图示,$\vec{E}$和$\vec{H}$垂直:
    \[P = EH = \dfrac{I^2}{2\sigma\pi^2 b^3}\]

    我们对此结果进行面积分。图中坡印亭矢量没有垂直于截面的分量,因此只需积分侧表面:
    \[\oint \vec{P} \cdot d\vec{A} = \dfrac{I^2}{2\sigma\pi^2 b^3} 2\pi bl\]

    由于电阻的决定式$R = \dfrac{l}{\sigma S}$:
    \[\oint \vec{P} \cdot d\vec{A} = \dfrac{I^2 l}{\sigma\pi b^2} = I^2 R\]

    后者为通电导线消耗的功率,因此坡印亭定理得证。
\end{solution}

\begin{solution}[Polarization]
    There is a permanently polarized insulating sphere of radius \( R \) with the ploarization \( \vec{P} = P_0 \dfrac{r}{R}\hat{r} \).
    Find the electric field \( E_{in} \) inside sphere and \( E_{out} \) outside sphere as functions of \( r \) respectively.
    \tcbrule

    在极化矢量部分我们已经提到$\sigma' = \vec{P}\cdot\vec{n}$。题干描述为只存在极化矢量没有外加电场的情形。我们可以得到它的电荷分布:
    \[q' = \oint \vec{P}\cdot \dif \vec{A} = 4\pi P_0 \dfrac{r^3}{R}\]

    对于一个球体,利用高斯公式可以求解电场强度:
    \[\oint \vec{E}\cdot\dif \vec{A} = 4\pi r^2\vec{E} = \dfrac{q'}{\varepsilon_0}\]

    考虑方向,极化矢量由负电荷指向正电荷,球体内部的电荷为负电荷,因此:
    \[E_{in} = -\dfrac{P_0 r}{\varepsilon_0 R}\qquad E_{out} = -\dfrac{P_0 R^2}{\varepsilon_0 r^2}\]

    事实上,类似于高斯定理,结合球坐标系的偏导以及极化矢量是与半径唯一有关的函数的特点,体电荷密度可以表示为:
    \[\rho' = -\nabla\cdot \vec{P} = \dfrac{1}{r^2}\dfrac{\partial (r^2\vec{P})}{\partial r} = -\dfrac{3P_0}{R}\]
\end{solution}

\begin{solution}[Mutual Induction]
    Please prove: In any case, for mutual inductance \(M_{12} \) and \(M_{21}\), the following equation exists: 
    \[M_{12} = M_{21}\]
    \tcbrule

    我们考虑能量。在电感中能量表示为$W = \dfrac{1}{2}LI^2$。
    
    现在考虑两个电感线圈1和2。首先闭合线圈1所在回路,系统能量为
    \[W_1 = \dfrac{1}{2}L_1 I_1^2\]

    再闭合线圈2,首先线圈2自身电流从0到$I_2$,其自身能量为
    \[W_2 = \dfrac{1}{2}L_2 I_2^2\]

    在电流$I_2$的影响下,对线圈1产生互感电动势,为了保持电流$I_1$不变,需要克服互感电动势做功:
    \[W_{21} = -\int_0^{I_2} \varepsilon I_1 \dif{i} = \dfrac{1}{2}M_{21}I_{1}I_{2}\]
    
    这部分能量储存于磁场中,因此系统磁场的能量为
    \[W = W_1 + W_2 + W_{21} = \dfrac{1}{2}L_1 I_1^2 + \dfrac{1}{2}L_2 I_2^2 + \dfrac{1}{2}M_{21}I_{1}I_{2}\]

    当然,如果先接通线圈2所在回路再接通线圈1所在回路,这个系统的能量应当$W' = W$,由此:
    \[M_{12} = M_{21}\]
    \tcbrule
    当然,除了能量的角度,我们给出场波教材中的另一种解释:\footnote{本来想把这个思路作为题目的,但是不仅数理要求很高而且难以以题目形式给出,因此仅供欣赏。}

    在麦克斯韦方程组中我们已知磁感应强度的散度为0:
    \[\nabla \cdot \vec{B} = 0\]

    由数学结论,旋度的散度为0。我们必然能够找到一个场矢量$\vec{A}$使得其旋度为$\vec{B}$,即:
    \[\vec{B} = \nabla \times \vec{A}\]

    将这一等式带入静磁场旋度公式,可得:
    \[\nabla \times \vec{B} = \nabla \times (\nabla \times \vec{A}) = \nabla(\nabla\cdot\vec{A}) - \nabla^2 \vec{A} = \mu_0 J\]

    由亥姆霍兹定理,空间中任意确定的场是一个无散的场与无旋的场的叠加。我们不妨令$\nabla\cdot\vec{A} = 0$(这也被称为库仑规范),那么:
    \[\nabla^2 \vec{A} = -\mu_0 J\]

    这是一个矢量泊松方程,对照我们很熟悉的公式及其方程的解$\nabla\cdot\vec{E} = -\nabla^2 V = \dfrac{\rho}{\varepsilon_0}\qquad V = \dfrac{1}{4\pi\varepsilon_0}\int_{V}\dfrac{\rho}{R}\dif{V}$,我们可知:
    \[A = \dfrac{\mu_0}{4\pi}\int_{V} \dfrac{J}{R}\dif{V}\]

    由此定义的$A$被称为磁矢势,对标于电场中的电势。

    我们先求解互感$M_{12}$,由其定义式为基础,带入磁矢势的定义:
    \[M_{12} = \dfrac{N_2}{I_1}\int_{S_2} \vec{B_1}\cdot\dif{\vec{S_2}} = \dfrac{N_2}{I_1}\int_{S_2} (\nabla\times\vec{A_1})\cdot\dif{\vec{S_2}}\]

    由Stokes公式,面积分转换为线积分:
    \[M_{12} = \dfrac{N_2}{I_1}\oint_{C_2} \vec{A_1}\cdot\dif{\vec{l_2}}\]

    由电流密度的定义:
    \[\vec{J}\dif{V} = JS\dif{l} = I\dif{l}\]

    代入可得:
    \[A_1 = \dfrac{\mu_0 N_1 I_1}{4\pi}\oint_{C} \dfrac{\dif{l_1}}{R}\]
    
    原本的公式只有一圈积分,这里将多匝的影响也考虑在内。

    进一步代入可得:
    \[M_{12} =\dfrac{\mu_0 N_1 N_2}{4\pi}\oint_{C_1}\oint_{C_2} \dfrac{\dif{l_1}\cdot\dif{l_2}}{R}\]

    其中$R$是微分长度$\dif{l_1}$和$\dif{l_2}$之间的距离。上式习惯将匝数归并到沿电路$C_1$和$C_2$的积分中,因此也写作:
    \[M_{12} =\dfrac{\mu_0}{4\pi}\oint_{C_1}\oint_{C_2} \dfrac{\dif{l_1}\cdot\dif{l_2}}{R}\]

    上式被称为计算互感的纽曼公式。观察可见角标1和2可互换,因此原命题得证。
\end{solution}

\begin{solution}[Point charge]
    As shown in figuer below, an electron is constrained to move along the axis of the ring with a charge \( q \). If the 
    electron can perform small oscillations through the center of the ring, calculate its oscillation frequency.
    \begin{singlefigure}{chapter8_exercise虎哥1.jpg}[0.7]
    \end{singlefigure}
    \tcbrule

    带电圆环在其圆心中心线上形成的电场强度表达式为(假设中心线为z轴):
    \[E = \dfrac{qz}{4\pi\varepsilon_0(z^2 + R^2)^{3/2}}\]

    由于微小振动,$R\gg z$,因此:
    \[E = \dfrac{qz}{4\pi\varepsilon_0R^3}\]

    构建振动方程:
    \[m\dfrac{\dif^2 z}{\dif t^2} = -eE = -\dfrac{eqz}{4\pi\varepsilon_0R^3}\]

    解得:
    \[\omega = \sqrt{\dfrac{eq}{4m\pi\varepsilon_0 R^3}}\]
    \[f = \dfrac{1}{2\pi}\sqrt{\dfrac{eq}{4\pi\varepsilon_0R^3}}\]
\end{solution}

\begin{solution}[Magnetic Field and Current]
    As shown in figuer below, there is a \itr{coaxial cable}{同轴电缆} made of superconducting material (\(\sigma\rightarrow\infty\)),
    and having short \itr{circuited end}{短路端} free to move along the $x$ axis.The radius of its central rod is \( a \) and its outer diameter is \( b \).
    \begin{singlefigure}{chapter8_exercise虎哥2.jpg}[0.7]
    \end{singlefigure}

    (a)What is the inductance of the cable as a function of $x$?\\
    (b)What is the force on the end? If the magnitude of the current it carries is \( i \).
    \tcbrule

    (a)按照固定步骤,假设中心杆中通有电流$i$,则有:
    \[B = \dfrac{\mu_0 i}{2\pi r}\quad\Rightarrow\quad \Phi = \int_a^b Bx\dif r = \dfrac{\mu_0 i x}{2\pi}\ln(\dfrac{b}{a})\]

    (注意这里的磁通量是螺旋环绕的磁场的磁通量,$x\dif r$表示了面积)

    因此自感系数为:
    \[L = \dfrac{\phi}{i} = \dfrac{\mu_0 x}{2\pi}\ln(\dfrac{b}{a})\]
    
    (b)在电感中储存的系统能量为$U = \dfrac{1}{2}Li^2$,因此在本题中:
    \[U = \dfrac{\mu_0 x i^2}{4\pi}\ln(\dfrac{b}{a})\]

    接下来应用虚功原理,可得:
    \[F = -\dfrac{\partial U}{\partial x} = -\dfrac{\mu_0 i^2}{4\pi}\ln(\dfrac{b}{a})\]
\end{solution}

\begin{solution}[Electromagnetic field]
    As shown in figuer below, a thin block with conductivity \(\sigma\) and thickness \( \delta \) moves with constant velocity $v_{ix}$
    between short citcuited superconducting parallel plates.An initial surface current \( K_0 \)(the current per width) is imposed
    at \( t = 0 \) when \( x = x_0 \), but the source is then removed.
    \begin{singlefigure}{chapter8_exercise虎哥3.jpg}[0.7]
    \end{singlefigure}

    (a)The surface current on the plates \( K(t) \) will vary with time. What is the magnetic field in term of \( K(t) \)? Neglect 
    fringing effects.\\
    (b)Because the moving block is so thin, the current is uniformly distributed over the thickness \( \delta \).
    Please find \( K(t) \) as a function of time.\\
    (c)What value of velocity will just keep the magnetic field constant with time until the moving block reaches the end?
    \tcbrule

    (a)由安培环路定理:
    \[\oint B\cdot\dif l = B\cdot b = \mu_0 K(t)\cdot b\]
    \[B = \mu_0 K(t)\]

    (b)薄物块在运动,运动产生磁通量的变化,产生感生电动势,感生电动势维持了电流。

    假设y轴方向物块高度为a,z轴方向长度为b,则电阻满足:
    \[R = \dfrac{1}{\sigma}\dfrac{l}{S} = \dfrac{a}{\sigma b\delta}\]

    假设$x$轴方向运动的函数为$x(t)$,则磁场通量为:
    \[\Phi = Bx(t)a = \mu_0 a K(t)x(t)\]

    感生电动势为:
    \begin{align*}
        \varepsilon &= -\dfrac{\dif \Phi}{\dif t} = -\mu_0 a \dfrac{\dif}{\dif t}[K(t)x(t)]\\
                    &= -\mu_0 a [\dfrac{\dif K(t)}{\dif t}x(t) + K(t)\dfrac{\dif x(t)}{\dif t}]\\
                    &= -\mu_0 a [\dfrac{\dif K(t)}{\dif t}x(t) - v K(t)]\\
                    &= IR = K(t) b \dfrac{a}{\sigma b\delta}\\
                    &= \dfrac{aK(t)}{\sigma \delta}
    \end{align*}

    其中$x(t) = x_0 - vt$,代入上式整理,得到:
    \[\mu_0(x_0 - vt) \dfrac{\dif K(t)}{\dif t} = (\mu_0 v - \dfrac{1}{\sigma \delta})K(t)\]

    分离微分方程的变量,得到:
    \[\dfrac{\dif K(t)}{K(t)} = \dfrac{\mu_0 v - \dfrac{1}{\sigma \delta}}{\mu_0v(\dfrac{x_0}{v} - t)}\dif t\]

    求解一阶微分方程,可得:
    \[K(t) = K'(\dfrac{x_0}{v} - t)^{1/\mu_0 v \sigma\delta -1}\]

    代入初值条件求解系数,得:
    \[K_0 = K'(\dfrac{x_0}{v})^{1/\mu_0 v \sigma\delta -1}\]

    反解出$K'$并最终代入结果,得:
    \[K(t) = K_0(1-\dfrac{v}{x_0}t)^{1/\mu_0 v \sigma\delta -1}\]

    (c)由上一问求得的结果可知,若$B = constant$,则$K(t) = constant$,因此:
    \[\dfrac{1}{\mu_0 v \sigma\delta} - 1 = 0\]

    所以:
    \[v = \dfrac{1}{\mu_0 \sigma\delta}\]
\end{solution}