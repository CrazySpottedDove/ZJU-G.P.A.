\chapter[电磁学]{\itr{Electromagnetism}{电磁学}}
\begin{prove}[Gauss's Law]
    我们考虑直接通过麦克斯韦方程组证明。首先已知$\nabla \cdot \vec{E} = \dfrac{\rho}{\varepsilon_0}$,
    现在假设一个很小的立方体包围带点电荷空间,沿着$x,y,z$方向,我们有关于电场通量的表达式:
    \begin{align*}
        \Phi &= \oint_S \vec{E} \cdot \dif \vec{S} \\
        &= \left[E_x(x + \dif x) - E_x(x)\right] \dif y \dif z \\
        &+ \left[E_y(y + \dif y) - E_y(y)\right] \dif x \dif z \\
        &+ \left[E_z(z + \dif z) - E_z(z)\right] \dif x \dif y
    \end{align*}
    使用微积分中的高斯定理,可以得到:
    \begin{align*}
        \Phi &= (\dfrac{\partial E_x}{\partial x} + \dfrac{\partial E_y}{\partial y} + \dfrac{\partial E_z}{\partial z})\dif{x}\dif{y}\dif{z} \\
        &= \nabla \cdot \vec{E} \dif{V}\\
        &= \dfrac{\rho}{\varepsilon_0} \dif{V}\\
        &= \dfrac{1}{\varepsilon_0}(\rho \dif{V})\\
        &= \dfrac{Q_{in}}{\varepsilon_0}
    \end{align*}
    当然,我们也理解,对于多个这样的微元立方体,其叠加的时候,总的电场通量是各个小的电通量的累加。此时考虑两个微元立方体的接触面,对于一个立方体进入的通量等于另一个立方体离开的通量,这一部分是可以相互抵消的,因此:
    \[\oint\vec{E}\cdot\dif{S} = \sum_{i = 1}^{N} \oint_{S_i}\vec{E}\cdot\dif{S_i} = \sum_{i=1}^{N}V_i \dfrac{\oiint_{S_i}\vec{E}\cdot\dif{\vec{S_i}}}{V_i}\]
    再结合积分与求和的关系,可以最终得到:
    \begin{align*}
        \oint_S\vec{E}\cdot\dif{S} &= \iiint_V\lim_{\Delta V\rightarrow 0}\dfrac{\oiint_{S_i}\vec{E}\cdot\dif{S_i}}{\Delta V}\dif{V}\\
        &= \iiint_V (\nabla\cdot\vec{E})\dif{V}\\
        &= \iiint_V \dfrac{\rho}{\varepsilon_0}\dif{V}\\
        &= \dfrac{Q_{in}}{\varepsilon_0}
    \end{align*}
\end{prove}

\begin{prove}[Properties of electric dipoles]
    对于如下的电偶极子系统:
    \begin{singlefigure}[电偶极子系统示意图]{chapter8_电偶极子的电势与电势能}[0.5]
    \end{singlefigure}
    % 下面的推导可能与图不搭,建议后续修改
    (1)关于电偶极子系统周围的电场,我们主要利用库仑定律推导,根据近似关系,我们有:
    \[\vec{r}_{\pm}=(rcos\theta\mp\dfrac{d}{2})\hat{k}+rsin\theta\hat{j}\]
    \[r_{\pm}^2 \approx r^2\mp rdcos\theta\qquad(d\ll r)\]
    其中第二个公式利用了余弦定理,且我们在这个问题的思考中默认$d$远小于$r$。然后我们列写库仑定律表达式(这里我们利用矢量计算,因此对库仑定律公式做出些许改变,以符合单位制的要求):
    \begin{align*}
        \vec{E} &= \dfrac{1}{4\pi\varepsilon_0}[\dfrac{(+q)\vec{r}_{+}}{r_{+}^3}+\dfrac{(-q)\vec{r}_{-}}{r_{-}^3}]\qquad\text{(库仑定律公式)}\\
            &\approx \dfrac{q}{4\pi\varepsilon_0}[\dfrac{(rcos\theta-\dfrac{d}{2})\hat{k}+rsin\theta\hat{j}}{(r^2 -rdcos\theta)^{3/2}}-\dfrac{(rcos\theta+\dfrac{d}{2})\hat{k}+rsin\theta\hat{j}}{(r^2 +rdcos\theta)^{3/2}}]\qquad\text{$r^3=(r^2)^{3/2}$}
    \end{align*}
    由于以下近似关系:
    \[\dfrac{1}{(r^2\mp rdcos\theta)^{3/2}}=\dfrac{1}{r^3}(1\mp \dfrac{d}{r}cos\theta)^{-3/2}\approx\dfrac{1}{r^3}(1\pm \dfrac{3}{2}\dfrac{d}{r}cos\theta)\]
    代入化简上式可得:
    \begin{align*}
        \vec{E} &\approx \dfrac{q}{4\pi\varepsilon_0 r^3}[-d\hat{k}+\dfrac{3dcos\theta}{r}(rcos\theta\hat{k}+rsin\theta\hat{j})] \\
            &= \dfrac{1}{4\pi\varepsilon_0 r^3}[(-qd\hat{k})+\dfrac{3(qd)r cos\theta}{r^2}(rcos\theta\hat{k}+rsin\theta\hat{j})]\\
            &= \dfrac{1}{4\pi\varepsilon_0 r^3}[-\vec{p}+3\dfrac{(\vec{r}\cdot\vec{p})\vec{r}}{r^2}]
    \end{align*}
    (2)关于电偶极子系统周围的电势,我们也利用近似处理与基础公式结合的方式推导。我们默认无限远处为0电势。近似关系包括:
    \[r_{-}-r_{+} \approx dcos\theta \qquad r_{-}r_{+}\approx r^2\]
    因此利用点电荷的电势表达式计算可得:
    \begin{align*}
        V &= \dfrac{1}{4\pi\varepsilon_0}(\dfrac{q}{r_{+}}+\dfrac{-q}{r_{-}})=\dfrac{qdcos\theta}{4\pi\varepsilon_0 r^2}\\
            &= \dfrac{pcos\theta}{4\pi\varepsilon_0 r^2}
    \end{align*}
\end{prove}

\begin{prove}[导体表面电场与曲率半径的关系]
    我们假想两个球形带电体,他们具有不同的半径。现在使用一个导线连接两个球体表面,电荷将会在两个带电导体上出现移动并最终实现平衡。很明显,这时候两个球体表面是等电势的。我们有电势的推导公式:
    \[V=\dfrac{1}{4\pi\varepsilon}\dfrac{q}{R}=\dfrac{\sigma R}{\varepsilon_0}\qquad E=\dfrac{\sigma}{\varepsilon_0}\]
    所以:
    \[V_1 =V_2\quad\Rightarrow\quad \dfrac{E_1}{E_2}=\dfrac{\sigma_1}{\sigma_2}=\dfrac{R_2}{R_1}\]
    同时按照推论,我们也有$E\propto\sigma$,因此:
    \[E\propto\dfrac{1}{R}\quad\sigma\propto\dfrac{1}{R}\]
\end{prove}

\begin{prove}[一阶电路的结论]
    \begin{singlefigure}[简单一阶RC电路]{chapter8_RC}[0.6]
    \end{singlefigure}
    首先考虑当电路从没有接通电源到接通电源后的变化过程。
    
    根据电容特性以及基尔霍夫定律,假设变量为回路电流和电容两端的电压,可知:
    \begin{align*}
        i=C\dfrac{\dif{u}}{\dif{t}}\\
        \varepsilon =iR+u\\
        \text{初值条件:}u=0
    \end{align*}

    整理可得:
    \[\varepsilon =u+RC\dfrac{\dif{u}}{\dif{t}}\]

    根据微分方程的结果可知,该方程的解为:
    \[u=\varepsilon(1-e^{-t/RC})\]

    当电容完全充电后,将开关切换到b点,此时也可以列写出相应的微分方程:
    \begin{align*}
        RC\dfrac{\dif{u}}{\dif{t}}+u=0\\
        \text{初值条件:}u=\varepsilon
    \end{align*}

    可解得:
    \[u=\varepsilon e^{-t/RC}\]

    对于RL电路读者可采用类似方法证明,此处不多赘述。
\end{prove}

\begin{prove}[Y-$\Delta$等效]
    \begin{singlefigure}[Y-$\Delta$等效电路]{chapter8_电路等效变换.png}[0.7]
    \end{singlefigure}

    等效要求不改变电路中其他部分的电压和电流等各项电路性质,因此我们可以假设三个端口的电势和电流,这在两种连接中应该是分别对应相等的。考虑到用电流表示电压时“$\Delta$”
    型连接需要表示六个电流参数,而电压表示电流时“Y”型连接只需要四个电势参数,所以我们采用电压表示电流。

    注意到,“Y”型连接有一个公共点O,如果希望表示电压电流的话,知道这一点处的信息将能够极大地简化相关物理量的表示。另外,电压表示电流我们可以采用电导率来降低公式的复杂度:
    \[ I_1 = G_1U_{10} = (\varphi_1 - \varphi_0) \times G_1 \]
    \[ I_2 = G_2U_{20} = (\varphi_2 - \varphi_0) \times G_2 \]
    \[ I_3 = G_3U_{30} = (\varphi_3 - \varphi_0) \times G_3 \]

    由KCL定律:
    \[ I_1 + I_2 + I_3 = 0\quad\Rightarrow\quad \varphi_1G_1 + \varphi_2G_2 + \varphi_3G_3 - \varphi_0(G_1 + G_2 + G_3) = 0 \]

    因此:
    \[ \varphi_0 = \dfrac{\varphi_1G_1 + \varphi_2G_2 + \varphi_3G_3}{G_1 + G_2 + G_3} \]

    由Y电路将电势$\varphi_0$代入最初的三个表达式可得:
    \[ I_1 = \dfrac{G_1G_2}{G_1 + G_2 + G_3} (\varphi_1 - \varphi_2) + \dfrac{G_1G_3}{G_1 + G_2 + G_3} (\varphi_1 - \varphi_3) \]
    \[ I_2 = \dfrac{G_2G_1}{G_1 + G_2 + G_3} (\varphi_2 - \varphi_1) + \dfrac{G_2G_3}{G_1 + G_2 + G_3} (\varphi_2 - \varphi_3) \]
    \[ I_3 = \dfrac{G_3G_1}{G_1 + G_2 + G_3} (\varphi_3 - \varphi_1) + \dfrac{G_3G_2}{G_1 + G_2 + G_3} (\varphi_3 - \varphi_2) \]

    由$\Delta$电路:
    \[ I_1 = G_{12}U_{12} + G_{31}U_{13} = G_{12}(\varphi_1 - \varphi_2) + G_{31}(\varphi_1 - \varphi_3) \]
    \[ I_2 = G_{12}U_{21} + G_{23}U_{23} = G_{12}(\varphi_2 - \varphi_1) + G_{23}(\varphi_2 - \varphi_3) \]
    \[ I_3 = G_{31}U_{31} + G_{23}U_{32} = G_{31}(\varphi_3 - \varphi_1) + G_{23}(\varphi_3 - \varphi_2) \]
    
    根据系数相等可得:
    \[ G_{12} = \dfrac{G_1G_2}{G_1 + G_2 + G_3} \]
    \[ G_{23} = \dfrac{G_2G_3}{G_1 + G_2 + G_3} \]
    \[ G_{31} = \dfrac{G_3G_1}{G_1 + G_2 + G_3} \]

    根据电导率反解电阻表达式:
    \[ R_{12} = \dfrac{R_1R_2 + R_2R_3 + R_1R_3}{R_3} \]
    \[ R_{23} = \dfrac{R_1R_2 + R_2R_3 + R_1R_3}{R_1} \]
    \[ R_{31} = \dfrac{R_1R_2 + R_2R_3 + R_1R_3}{R_2} \]

    再次反解表达式,可以得到逆变换的结果:
    \begin{align*}
        \begin{cases}
            \vspace{3mm}
            R_1 = \dfrac{R_{31}R_{12}}{R_{12} + R_{23} + R_{31}}\\
            \vspace{3mm}
            R_2 = \dfrac{R_{12}R_{23}}{R_{12} + R_{23} + R_{31}}\\
            R_3 = \dfrac{R_{23}R_{31}}{R_{12} + R_{23} + R_{31}}\\
        \end{cases}
    \end{align*}
    
\end{prove}

\begin{prove}[\itr{Ampere's Law}{安培定理}]
    \begin{singlefigure}[安培定理的证明]{chapter8_证明安培定理}[0.6]
    \end{singlefigure}
    如上图所示,我们取$xOy$平面内很小的一个方形区域,应用麦克斯韦方程$\nabla\times\vec{B}=\mu_0\vec{J}$,有:
    \begin{align*}
        \oint \vec{B} \cdot \dif{\vec{l}} &= \left[ B_x(y) - B_x(y + \dif{y}) \right] \dif{x} + \left[ B_y(x + \dif{x}) - B_y(x) \right] \dif{y}\\
                                        &= \left( \dfrac{\partial B_y}{\partial x} - \dfrac{\partial B_x}{\partial y} \right) \dif{x} \dif{y} \\
                                        &= \left( \nabla \times \vec{B} \right)_z \dif{x} \dif{y} = \left( \nabla \times \vec{B} \right) \cdot \dif{\vec{S}} \\
                                        &= \mu_0 \vec{J} \cdot \dif{\vec{S}} = \mu_0 I
    \end{align*}
\end{prove}

\begin{prove}[外加磁场对磁偶极子的影响]
    \begin{singlefigure}[磁场对磁偶极子的影响]{chapter8_磁场对磁偶极子的影响}[0.6]
    \end{singlefigure}
    我们使用经典力学分析,根据圆周运动易得初始平衡状态时$\dfrac{Ze^2}{4\pi\varepsilon_0 r^2}=m_e\omega_0^2 r$,此时外加一个磁场,其中的电子会多受到一个洛伦茨力的影响。

    以$\vec{\Delta \omega}\parallel\vec{B}_0$为例,则:
    \[\dfrac{Ze^2}{4\pi\varepsilon_0 r^2} + e\omega r B = m_e\omega^2 r\]

    此处有$\omega = \omega_0 + \Delta \omega$,因此:
    \[\dfrac{Ze^2}{4\pi\varepsilon_0 r^2} + e\omega_0 r B + e\Delta\omega r B = m_e\omega_0^2 r + 2\omega_0\Delta\omega m_e r + m_e\Delta\omega^2 r\]

    由于$\omega\gg\Delta\omega$以及最初平衡的等式,我们在该展开式中删除原等式的部分以及仅含有$\Delta\omega$的部分,可得:
    \[e\omega_0 r B = 2\omega_0\Delta\omega m_e r\quad\Rightarrow\quad \Delta\omega = \dfrac{eB}{2m_e}\]

    代入磁偶极矩与角速度的关系式即得最终结论。当$\vec{\Delta \omega}\parallel -\vec{B}_0$的推导也类似,此处不再赘述。
\end{prove}

\begin{prove}[电磁波的性质]
    在自由空间(电荷密度$\rho = 0$,电流密度$J = 0$的空间)中,电磁波满足:
    \begin{Itemize}
        \item 麦克斯韦方程组满足狭义相对论。
        \item 电与磁作为波的存在性:麦克斯韦方程组证明了电磁波的存在,且电磁波是横波且是偏振的。$\vec{E}$、$\vec{B}$、传播方向两两相互正交。
        \item 电磁波中电场和磁场始终是共面的,二者相位差为0。
        \item 光是一种电磁波,且光速满足$c=\dfrac{1}{\sqrt{\mu_0\varepsilon_0}}$
        \item 守恒律:电荷总量、动量、能量三者均守恒。
        \item $E = cB$,因此大多数仪器包括人眼对于电场的变化更为敏感。
    \end{Itemize}
    \tcbrule
    首先引入平面波的概念。在传播过程中,单点产生的电磁波将会扩散到无穷远的地方。半径越大,对于其中有限大的局部电磁波,就越接近于平面。因此电磁波的性质将在平面传播的基础上讨论。

    根据麦克斯韦方程,在自由空间中,我们有:
    \begin{align*}
        \nabla\cdot\vec{E} = 0 \quad&\Rightarrow\quad \dfrac{\partial E_x}{\partial x} + \dfrac{\partial E_y}{\partial y} + \dfrac{\partial E_z}{\partial z} = 0\\[1.5ex]
        \nabla\times\vec{E} = -\dfrac{\partial \vec{B}}{\partial t}\quad&\Rightarrow\quad 
        \begin{vmatrix}
            \hat{i} & \hat{j} & \hat{k}\\
            \dfrac{\partial}{\partial x} & \dfrac{\partial}{\partial y} & \dfrac{\partial}{\partial z}\\
            E_x & E_y & E_z
        \end{vmatrix}
        = -\mu_r\mu_0(\dfrac{\partial H_x}{\partial t}\hat{i} + \dfrac{\partial H_y}{\partial t}\hat{j} + \dfrac{\partial H_z}{\partial t}\hat{k})\\[1.5ex]
        \nabla\cdot\vec{H} = 0\quad&\Rightarrow\quad \dfrac{\partial H_x}{\partial x} + \dfrac{\partial H_y}{\partial y} + \dfrac{\partial H_z}{\partial z} = 0\\[1.5ex]
        \nabla\times\vec{H} = \dfrac{\partial\vec{D}}{\partial t}\quad&\Rightarrow\quad
        \begin{vmatrix}
            \hat{i} & \hat{j} & \hat{k}\\
            \dfrac{\partial}{\partial x} & \dfrac{\partial}{\partial y} & \dfrac{\partial}{\partial z}\\
            H_x & H_y & H_z
        \end{vmatrix}
        = \varepsilon_r\varepsilon_0(\dfrac{\partial E_x}{\partial t}\hat{i} + \dfrac{\partial E_y}{\partial t}\hat{j} + \dfrac{\partial E_z}{\partial t}\hat{k})
    \end{align*}
    进一步地,我们展开所有行列式,得到下面8个等式:
    \begin{align*}
        &\dfrac{\partial E_x}{\partial x} + \dfrac{\partial E_y}{\partial y} + \dfrac{\partial E_z}{\partial z} = 0 \tag{1} \\[1.5ex]
        &\dfrac{\partial E_z}{\partial y} - \dfrac{\partial E_y}{\partial z} = -\mu_r \mu_0 \dfrac{\partial H_x}{\partial t} \tag{2-1} \\[1.5ex]
        &\dfrac{\partial E_x}{\partial z} - \dfrac{\partial E_z}{\partial x} = -\mu_r \mu_0 \dfrac{\partial H_y}{\partial t} \tag{2-2} \\[1.5ex]
        \displaybreak
        &\dfrac{\partial E_y}{\partial x} - \dfrac{\partial E_x}{\partial y} = -\mu_r \mu_0 \dfrac{\partial H_z}{\partial t} \tag{2-3} \\[1.5ex]
        &\dfrac{\partial H_x}{\partial x} + \dfrac{\partial H_y}{\partial y} + \dfrac{\partial H_z}{\partial z} = 0 \tag{3} \\[1.5ex]
        &\dfrac{\partial H_z}{\partial y} - \dfrac{\partial H_y}{\partial z} = \varepsilon_r \varepsilon_0 \dfrac{\partial E_x}{\partial t} \tag{4-1} \\[1.5ex]
        &\dfrac{\partial H_x}{\partial z} - \dfrac{\partial H_z}{\partial x} = \varepsilon_r \varepsilon_0 \dfrac{\partial E_y}{\partial t} \tag{4-2} \\[1.5ex]
        &\dfrac{\partial H_y}{\partial x} - \dfrac{\partial H_x}{\partial y} = \varepsilon_r \varepsilon_0 \dfrac{\partial E_z}{\partial t} \tag{4-3}
    \end{align*}

    电磁波是以一点向外传播的,在包络的球面上每一点的电场和磁场强度都相等。对于平面波而言(假设平面是$xOy$平面),就是在这一个面上的电场和磁场与$x$和$y$轴都无关。因此有:
    \[\dfrac{\partial E_x}{\partial x} = \dfrac{\partial E_y}{\partial y} = \dfrac{\partial E_x}{\partial y} = \dfrac{\partial E_y}{\partial x} = 0\]
    \[\dfrac{\partial H_x}{\partial x} = \dfrac{\partial H_y}{\partial y} = \dfrac{\partial H_x}{\partial y} = \dfrac{\partial H_y}{\partial x} = 0\]

    根据式(1)(2-3)(3)(4-3)可得:
    \[\dfrac{\partial E_z}{\partial z} = 0\qquad \dfrac{\partial H_z}{\partial t} = 0\]
    \[\dfrac{\partial H_z}{\partial z} = 0 \qquad \dfrac{\partial E_z}{\partial t} = 0\]

    因此可得到$H_z$和$E_z$与$z$轴以及时间无关,即满足平面波:
    \[E_z(z,t) = constant\]
    \[H_z(z,t) = constant\]

    进一步地,$z$轴方向分量与电磁波无关:
    \[E_z = H_z = 0\]

    由于分量为0,我们可以处理上一阶段剩余的四个方程,并得到如下的结果:
    \begin{align*}
        &\dfrac{\partial E_y}{\partial z} = \mu_r \mu_0 \dfrac{\partial H_x}{\partial t} \tag{2-1'} \\[1.5ex]
        &\dfrac{\partial E_x}{\partial z} = -\mu_r \mu_0 \dfrac{\partial H_y}{\partial t} \tag{2-2'} \\[1.5ex]
        \displaybreak
        &\dfrac{\partial H_y}{\partial z} = -\varepsilon_r \varepsilon_0 \dfrac{\partial E_x}{\partial t} \tag{4-1'} \\[1.5ex]
        &\dfrac{\partial H_x}{\partial z} = \varepsilon_r \varepsilon_0 \dfrac{\partial E_y}{\partial t} \tag{4-2'}
    \end{align*}

    我们不妨假设电场强度沿着$x$轴方向(即$E_y = 0$),则进一步简化得到:
    \[\dfrac{\partial H_x}{\partial t} = \dfrac{\partial H_x}{\partial z} = 0\]

    也就是说磁场强度没有$x$轴分量,它应当是在$y$轴。故电场方向和磁场方向垂直,并与电磁波传播方向分别垂直。

    根据方程(2-2')(4-1'),分别对其两边对$z$偏导,得到:
    \[\dfrac{\partial^2 E_x}{\partial z^2} = -\mu_r\mu_0\dfrac{\partial}{\partial t}\dfrac{\partial H_y}{\partial z} = \mu_r\mu_0\varepsilon_r\varepsilon_0\dfrac{\partial^2 E_x}{\partial t^2}\]
    \[\dfrac{\partial^2 H_y}{\partial z^2} = -\varepsilon_r\varepsilon_0\dfrac{\partial}{\partial t}\dfrac{\partial E_x}{\partial z} = \mu_r\mu_0\varepsilon_r\varepsilon_0\dfrac{\partial^2 H_y}{\partial t^2}\]

    很明显,这是一个波动方程,波速:
    \[v = \dfrac{1}{\sqrt{\mu_r\mu_0\varepsilon_r\varepsilon_0}}\]

    波速是与参考系无关的(计算所需的数据均为与介质唯一有关的常数)。因此麦克斯韦方程组的存在不依赖于参考系,且证明了光速的唯一性。

    对于波,可以用复数表示为$E_x = E_0 e^{i\varphi_E}e^{i(kz - \omega t)}$、$H_y = H_0 e^{i\varphi_H}e^{i(kz - \omega t)}$,这样就包含了相位以及幅值等参数。利用方程(2-2')(4-2')可得:
    \[ik E_0e^{i\varphi_E}e^{i(kz - \omega t)} = i\mu_r\mu_0\omega H_0e^{i\varphi_H}e^{i(kz - \omega t)}\]

    由于$\dfrac{\omega}{k} = v$,所以:
    \[E_0e^{i\varphi_E} = \mu_r\mu_0\dfrac{1}{\sqrt{\mu_r\mu_0\varepsilon_r\varepsilon_0}}H_0e^{i\varphi_H} = v B_0 e^{i\varphi_H}\]

    除了复指数外所有变量均为时数。因此从上面的公式可以得到以下结论:
    \[\sqrt{\varepsilon_r\varepsilon_0}E_0 = \sqrt{\mu_r\mu_0}H_0\]
    \[\varphi_E = \varphi_H\]
    \[E_0 = cB_0\qquad\text{(真空中)}\]
\end{prove}