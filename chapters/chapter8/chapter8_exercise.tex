\section{课后习题:电磁学}
\begin{example}[Polarization ---\refleaftext{solution8.1}]
    A positive point charge $Q$ is located at the center of a spherical shell dielectric with an inner radius of $R_i$ and an outer radius of $R_o$, 
    where the dielectric constant is $\varepsilon_r$. Determine the functions $\vec{E}$, $V$, $\vec{D}$, and $\vec{P}$ as a function of the radial distance $R$.
\end{example}
\begin{example}[Current Density/Boundary Condition ---\refleaftext{solution8.2}]
    As shown in Figure bellow. An electric field is applied between the plates of a parallel plate capacitor with an area \( S \). The space between the two metal plates 
    is filled with two dielectric materials of different thicknesses \( d_1 \) and \( d_2 \), with dielectric constants \( \varepsilon_1 \) and \( \varepsilon_2 \), 
    and conductivities \( \sigma_1 \) and \( \sigma_2 \). Determine the following:
    \begin{singlefigure}{chapter8_exercise电流密度边界条件.png}[0.7]
    \end{singlefigure}

    (a) The current density between the two parallel plates.\\
    (b) The electric field strength in the two dielectric materials.\\
    (c) The surface charge density on the two parallel plates and at the interface.
\end{example}
\begin{example}[Magnetic Field ---\refleaftext{solution8.3}]
    As shown in Fig.a. There is a toroidal iron core with a relative magnetic permeability $\mu_r = 3000$, an average radius \( R = 80 \) mm, and a circular cross-sectional 
    radius \( b = 25 \) mm. The length of the air gap is \( l_g = 3 \) mm. A winding of $N=500$ turns produces a magnetic flux \(\Phi =  10^{-5} \) Wb. 
    Neglecting magnetic leakage and using the average path length, calculate:
    \begin{minipage}{0.45\textwidth}
        \centering
        \begin{singlefigure}[a]{chapter8_exercise磁场.png}[0.99]
        \end{singlefigure}
    \end{minipage}
    \begin{minipage}{0.45\textwidth}
        \centering
        \begin{singlefigure}[b]{chapter8_exercise磁阻.png}[0.99]
        \end{singlefigure}
    \end{minipage}
    
    (a) The magnetic induction intensity \( B_g \) and magnetic field strength \( H_g \) in the air gap, and \( B_c \) and \( H_c \) in the iron core\\
    (b) The required current \( I \)\\
    (c) Similar to circuits and resistors, there are also concepts of magnetic circuits and \itr{magnetic resistance}{magnetic resistance} in magnetism. The latter is a constant 
    independent of the number of turns in a coil and is only related to the object itself. Magnetic resistance characterizes the hindrance to the magnetic flux passing through the magnetic 
    circuit, and its unit is $H^{-1}$. Given that the scaling coefficient in its expression is 1(with no additional constant factor).Taking Fig.a as an example, calculate the magnetic 
    resistance of the iron core \( R_c \) and air gap \( R_g \) respectively, and denote the magnetic flux by $N$, $I$, $R_g$ and $R_c$
    
    (Hint: The unit of inductance is $H$)\\
    (d) Consider the magnetic circuit in Fig.b. Two windings with turns $N_1$ and $N_2$ are wound on the two side limbs of the ferromagnetic core. The cross-sectional area of the 
    core is $S$, and the permeability is $\mu$. The length of the left, center and right limb is \(l_1, l_2, l_3\) respectively. Determine the magnetic flux in the center limb.
    
    (Hint: Observe the expression of magnetic flux in (c) and attempt to extend Kirchhoff's law to magnetic circuits)
\end{example}

\begin{example}[Poynting's Vector ---\refleaftext{solution8.4}]
    A long and straight cylindrical wire with a radius of \(b \) and an electrical conductivity of \(\sigma \) carries a direct current of \(I \), and the current is uniformly distributed 
    in the wire, as shown in the following figure. Find the Poynting vector on the surface of a cylindrical wire and prove the Poynting theorem: The surface integral of the Poynting vector 
    on a closed surface is equal to the power radiated by the volume enclosed by this closed surface.
    
    \begin{minipage}{0.45\textwidth}
        \centering
        \begin{singlefigure}[a]{chapter8_exercise坡应廷矢量1.png}[0.99]
        \end{singlefigure}
    \end{minipage}
    \begin{minipage}{0.45\textwidth}
        \centering  
        \begin{singlefigure}[b]{chapter8_exercise坡应廷矢量2.png}[0.99]
        \end{singlefigure}
    \end{minipage}
\end{example}

\begin{example}[Polarization ---\refleaftext{solution8.5}]
    There is a permanently polarized insulating sphere of radius \( R \) with the ploarization \( \vec{P} = P_0 \dfrac{r}{R}\hat{r} \).
    Find the electric field \( E_{in} \) inside sphere and \( E_{out} \) outside sphere as functions of \( r \) respectively.
\end{example}

\begin{example}[Mutual Induction ---\refleaftext{solution8.6}]
    Please prove: In any case, for mutual inductance \(M_{12} \) and \(M_{21}\), the following equation exists: 
    \[M_{12} = M_{21}\]
\end{example}

\begin{example}[Point charge ---\refleaftext{solution8.7}]
    As shown in figure below, an electron is constrained to move along the axis of the ring with a charge \( q \). If the 
    electron can perform small oscillations through the center of the ring, calculate its oscillation frequency.
    \begin{singlefigure}{chapter8_exercise虎哥1.jpg}[0.7]
    \end{singlefigure}
\end{example}

\begin{example}[Magnetic Field and Current ---\refleaftext{solution8.8}]
    As shown in figure below, there is a \itr{coaxial cable}{同轴电缆} made of superconducting material (\(\sigma\rightarrow\infty\)),
    and having short \itr{circuited end}{短路端} free to move along the $x$ axis.The radius of its central rod is \( a \) and its outer diameter is \( b \).
    \begin{singlefigure}{chapter8_exercise虎哥2.jpg}[0.7]
    \end{singlefigure}

    (a)What is the inductance of the cable as a function of $x$?\\
    (b)What is the force on the end? If the magnitude of the current it carries is \( i \).
\end{example}

\begin{example}[Electromagnetic field ---\refleaftext{solution8.9}]
    As shown in figure below, a thin block with conductivity \(\sigma\) and thickness \( \delta \) moves with constant velocity $v_{ix}$
    between short citcuited superconducting parallel plates.An initial surface current \( K_0 \)(the current per width) is imposed
    at \( t = 0 \) when \( x = x_0 \), but the source is then removed.
    \begin{singlefigure}{chapter8_exercise虎哥3.jpg}[0.7]
    \end{singlefigure}

    (a)The surface current on the plates \( K(t) \) will vary with time. What is the magnetic field in term of \( K(t) \)? Neglect 
    fringing effects.\\
    (b)Because the moving block is so thin, the current is uniformly distributed over the thickness \( \delta \).
    Please find \( K(t) \) as a function of time.\\
    (c)What value of velocity will just keep the magnetic field constant with time until the moving block reaches the end?
\end{example}