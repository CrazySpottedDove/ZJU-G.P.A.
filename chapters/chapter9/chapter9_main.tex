\chapter[光学]{\itr{Optics}{光学}}
% 世界因为光而五彩斑斓,万物因为光而更有生机。中学我们已经学过一些光的基础知识,例如反射、折射以及全反射,以及干涉和衍射等。在这一章节,我们将
% 从中学的基础出发,继续深入探究光的性质及其背后的机理。并且我们将会发现更多光的美妙。

% 这一章节,将从几何光学以及波动光学两个部分进行光的讨论,基础要求较低,读者可以不必太过担心。
% \section[几何光学]{\itr{Geometrical optics}{几何光学}}
% \subsection[导论]{\itr{Introduction}{导论}}
% 中学我们学过一些反射和折射的性质,例如平面镜反射中入射角等于反射角,而在不同介质交界面发生的折射则满足$n=\dfrac{\sin\theta_1}{\sin\theta_2}$,其中$n$为折射率等等。

% 最初,物理学界对光的认知分为以牛顿为代表的“光是一种粒子”以及以惠更斯为代表的“光是一种波”。对于折射现象,牛顿的理论在光子不受力的前提下,其平行于交界面的动量应该不变,但角度的变化导致推导出了超越
% 光速的更快光速,这是不合常理的。而惠更斯认为光是在以太中的纵波,很明显,波峰折射后依然是波峰,波谷亦然。我们可以通过假定在两种介质中的波长不同,并以交界面处列写等式
% $\dfrac{\lambda_i}{\cos\theta_i}=\dfrac{\lambda_j}{\cos\theta_j}$,顺利推导出了满足麦克斯韦方程组的光速在介质中的速度较真空小的正确结论。

% 1850年,福柯测量到水中的光速确实小于真空光速,这一决定性的实验因此扼杀了牛顿的光粒子理论。虽然惠更斯的以太假设并没有确切的根据,但确实他发现了一个正确的结论,并且凑巧的是,这一结论与
% 波粒二象性结论结合动量守恒假设推导的结果一致,均有:
% \[n=\dfrac{c}{v}\geq 1\]

% 对于白光,其融合了多种不同波长的波。我们知道光速是恒定的,但光的波长以及频率则会有多种值,我们日常可见光仅是其中很小的一段范围内的光:
% \begin{singlefigure}[光]{chapter9_光.png}[0.6]
% \end{singlefigure}

% 对于不同波长的光,在同一种折射中折射率是不同的,由此得到了色散现象,使得出现了彩虹、日晕等自然现象。
% \subsection[反射和折射]{\itr{Reflection  and  Refraction}{反射和折射}}
% 反射和折射相信读者已经很熟悉了,并且对这两种现象都有一定了解。那么为什么会这样?我们需要首先介绍\itr{Fermat’s Principle}{费马原理}:

% 费马原理认为,从一个定点传播到另一个定点的光线遵循一条路径,与附近的路径相比,所需的时间要么是最小的,要么是最大的,或者保持不变(即静止)。

% 从费马原理出发,我们可以证明以下的反射和折射的性质:
% \begin{law}
%     反射中,入射角$\theta_1$和反射角$\theta_2$相等关系,即:
%     \[\theta_1=\theta_2\]
% \end{law}
{\LARGE \centering \color{red}\itshape
  BLANK\\
  期待你的建设

}