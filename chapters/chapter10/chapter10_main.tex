\chapter[量子物理]{\itr{Quantum Physics}{量子物理}}
首先恭喜各位马上将会结束普物的考验。最初作者本人本想以“近代物理”作为章节标题,但考虑到具体内容和难度,还是以“量子物理”作为标题。

我们需要预先告诉各位的是,本章节的内容相较于前面的内容更为抽象化。但面向课程受众与期末考核的话,课程依然仅仅停留在量子力学的表面,并不需要各位很强的数理功底,因此大家需要尤其记忆重点公式以学会套用结论即可(偶尔需要适当举一反三doge)。
\begin{singlefigure}[人类群星闪耀时]{chapter10_人类群星闪耀时}[0.6]
    1927年第五届索维尔会议,人类群星闪耀的时刻。\\
    愿各位读者在本章节能够大神附体、所向披靡!
\end{singlefigure}
\section[一切从光开始]{\itr{Everything starts with light}{一切从光开始}}
中学物理已经告诉我们,量子力学的故事要从光开始说起。在前面一个章节中我们已经了解到光的波动性,但学界争论许久的光的问题并未完全解决。在著名的黑体辐射问题上科学家发现了一些端倪······

黑体模型是一个理想模型,我们预先给出一些概念:
\begin{Itemize}
    \item 黑体能够完全吸收所有入射辐射
    \item 黑体能够发出所有波长的辐射,但在某些波长上出现峰值,该波长仅与黑体的温度有关
    \item 许多高温物体与黑体有类似的辐射。
    \item \itr{emissivity}{发射率}是物体发射的单色辐射强度与相应黑体辐射的比值,即:
    \[\varepsilon_{\lambda} = \dfrac{I_{\lambda}(emitted)}{B_{\lambda}(T)}\]
    \item \itr{absorptivity}{吸收率}是入射单色强度被吸收的分数,即:
    \[A_{\lambda}=\dfrac{I_{\lambda}(absorbed)}{I_{\lambda}(incident)}\]
    \item 基尔霍夫定律指出,在热力学平衡条件下,$\varepsilon_{\lambda}=A_{\lambda}$,这暗示了在特定波长下能很好地吸收能量的物体在该波长下也是很好的发射器。
    \item \itr{Spectral emittance}{单色辐射出射度}:单位面积单位时间中每个波长间隔之间的强度,即
    \[R_\lambda(T)=\dfrac{\dif{R_\lambda}}{\dif{\lambda}}\quad \text{单位$\mathrm{W\cdot m^{-3}}$}\]
    \item \itr{total intensity}{辐射出射度/总辐射强度}:单位面积单位时间,在所有波长上分布函数的积分,即
    \[R(T)=\int_0^{\infty} R_{\lambda}(T)\dif{\lambda}\quad\text{单位$\mathrm{W\cdot m^{-2}}$}\]
\end{Itemize}
\begin{law}[\itr{Stefan-Boltzmann law}{斯特藩-玻尔兹曼定律}]
    一个黑体表面单位面积辐射出的总辐射强度与黑体本身的热力学温度$T$(又称绝对温度)的四次方成正比,即:
    \[R(T)=\sigma T^4\]
    其中常数$\sigma=5.67\times10^{-8}\mathrm{W\cdot m^{-2}\cdot K^{-4}}$。
\end{law}
\begin{law}[\itr{Wien displacement law}{维恩位移定律}]
    黑体辐射过程中,随着黑体温度的不同,其发射的最大波长也不同,满足以下等式:
    \[T\lambda_{max}=b\]
    其中常数$b=2.898\times 10^{-3}\mathrm{mK}$。
\end{law}

在探索黑体辐射的完整公式的过程中,诸多物理学家提出多个公式,但无一例外地,都无法正确符合最终的实验结论。最终普朗克通过量子的观点提出了普朗克公式:
\begin{law}[\itr{Planck’s radiation law}{普朗克辐射定律}]
    \[R(\lambda ,T)=\dfrac{2\pi hc^2}{\lambda^5(e^{hc/\lambda kT}-1)}\]
    其中$h$为普朗克常数,$c$为光速,$k$为玻尔兹曼常数。同时普朗克常数给出粒子的能量为$E=h\nu$。
\end{law}
\begin{Itemize}
    \item 当$\lambda$足够小时即为维恩公式$R(\lambda ,T)=\dfrac{C_1}{\lambda^5}e^{-C_2/\lambda T}$,根据普朗克的公式可见$C_1=2\pi hc^2$,$C_2=hc/k$。
    \item 当$\lambda$足够大时即为瑞利-琼斯定律,$R(\lambda ,T)=\dfrac{2\pi ckT}{\lambda^4}$。
    \item 对该公式在0到正无穷积分,可得斯特藩-玻尔兹曼定律。
    \item 对该公式微分求最大值,可得维恩位移定律。
\end{Itemize}

在对于光电效应的研究中,物理学家也发现一些问题,光电效应存在最大电流密度、截止频率,并且其从初态到产生光电效应的时间间隔非常小。最终爱因斯坦通过普朗克的量子理论成功解释了光电效应。
在光电效应中,光被视为粒子作用于金属表面激发出光电子。具体结论中学已经学过:
\begin{law}[\itr{Photoelectric effect}{光电效应}]
    \[h\nu=K+A\]
    其中$K$表示逸出的电子的动能,$A$表示逸出功。由此可以解释电流密度不同为光子的数量导致电子逸出的数量不同,进而表征了光辐射强度$I=Nh\nu$,并且可以计算出截止频率。
\end{law}
\section[康普顿效应]{\itr{The Compton Effect}{康普顿效应}}
康普顿在通过光击打特定物质的时候发现,散射出的光中存在一些新的波长,波长差与散射角度以及粒子质量有关。同样的,他使用普朗克的量子理论推论得到康普顿散射公式\footnote{温馨提示:该公式的推导曾经作为部分老师的期末考试大题出现}。

\begin{law}[\itr{The Compton Effect}{康普顿效应}---\refleaftext{prove10.1}]
    入射和出射光的波长差为:
    \[\Delta\lambda = \dfrac{h}{mc}(1-\cos\phi)\]

    其中$m$为粒子的质量,康普顿散射考虑了相对论效应。
\end{law}

进一步地,我们可以得到光的波粒二象性。同时二象性能够推广到所有物质,称为物质波(或者叫德布罗意波)。德布罗意在其博士论文中预言了物质波的存在,并在不久的将来,电子的波动性被成功观测,德布罗意因此凭借博士学位论文获得诺贝尔奖。

下面我们给出结论,这些都是量子理论的核心思想:
\begin{law}[\itr{Wave-Particle duality}{波粒二象性}]
    经过量子观念的思考并推广,我们得到波动与粒子的相互关系:
    \begin{Itemize}
        \item 光子的能量$E=h\nu=h\dfrac{c}{\lambda}$
        \item 光子的动量$p=\dfrac{h\nu}{c}=\dfrac{h}{\lambda}$
        \item 光辐射强度$I=Nh\nu$
    \end{Itemize}
    需要注意的是,这里的公式还需要考虑相对论对于粒子质量的影响,若没有特殊提示,则不需要参数修正。
\end{law}
\section[关于原子稳定性的讨论]{\itr{Discussion on Atoms Stability}{关于原子稳定性的讨论}}
“原子”之义,在于世界本源的基本粒子。但随着科学的进展,越来越多的实验发现了原子的内部组成,对于原子结构的理解也在不断地发展。

汤姆孙于1897年发现电子的存在,并提出一种布丁模型——正电荷均匀分布在原子内部,带负电的电子则被放置在适当的平衡位置;但同时也出现一些问题,例如该理论认为受到扰动后电子振动并发出特定频率的光,但实际的氢原子光谱却异常复杂。
卢瑟福基于$\alpha$粒子散射实验提出了一种新的模型——正电荷在原子中心形成微小的原子核并聚集了原子的绝大部分质量,电子绕着原子核运动,原子远大于原子核的大小;但这一理论违背了麦克斯韦方程,移动电荷意味着光的辐射,危害了原子的稳定性。

对于最简单的氢原子,物理学家开始关注于他的原子光谱,并致力于得到能够适配光谱实验数据的理论体系。其中巴耳末系具有深远影响,该理论认为氢原子的光谱波长满足:
\[\dfrac{1}{\lambda}=R_H(\dfrac{1}{2^2}-\dfrac{1}{n^2})\quad n=3,4,5...\quad R_H=1.0967758\times10^7 \mathrm{m^{-1}}\]

更一般地,我们可以替换其中的2为$k$,即为$\dfrac{1}{\lambda}=R_H(\dfrac{1}{k^2}-\dfrac{1}{n^2})\quad n=k+1,k+2...$,根据$k$值的不同有不同的名字,从1到4,依次为莱曼系(紫外波段)、巴耳末系(可见光波段)、帕申系(红外波段)、布拉开系(远红外波段)。

这时,一个叫做波尔的物理学家想到了普朗克的天才发现。不加证明地(实际上波尔本人也没有给出证明),他给出了一个新的氢原子模型:
\begin{Itemize}
    \item 原子中的电子能够维持在稳定的状态而不产生辐射。
    \item 原子吸收或发射特定波长的辐射后将会从一个稳定的状态转移到另一个稳定的状态。
    \item 原子外部运动的电子的角动量是约化普朗克常数的整数倍,即$L=n\hbar$\footnote{由于普朗克常数与$\pi$经常同时出现,因此定义约化普朗克常数$\hbar = \dfrac{h}{2\pi}$}
\end{Itemize}

这可以概括为三个假设:能级假设、跃迁假设以及轨道量子化假设。

在波尔的天才设想之下,我们利用经典观念中的原子模型,以氢原子为例,获得了一些关于量化轨道与能量的结论:
\begin{law}[\itr{Quantized Orbits and Energy}{量化轨道与能量}---\refleaftext{prove10.2}]
    我们利用波尔的原子模型中的正整数$n$作为各个物理量的下角标,可得:
    \begin{Itemize}
        \item 波尔半径$r_1=\dfrac{\varepsilon_0 h^2}{\pi me^2}=5.29\times 10^{-11}\mathrm{m}$
        \item 氢原子的\itr{Ground State}{基态}能量$E_1=-\dfrac{me^4}{8\varepsilon_0^2 h^2}=-13.6eV$
        \item 多个能级之间的能量关系$E_n=\dfrac{E_1}{n^2}$
        \item 从高能级$i$到低能级$j$的过程中辐射的光的频率为:
        \[\nu_{ij}=\dfrac{E_i -E_j}{h}=-\dfrac{me^4}{8\varepsilon_0^2 h^2}(\dfrac{1}{n_i^2}-\dfrac{1}{n_j^2})\]
        通过对比巴耳末系等相关的光谱理论,我们可以推导出其中相应的常数。
    \end{Itemize}
\end{law}

后续,波尔的理论也进一步被弗兰克-赫兹实验证实,他们测出了汞原子存在4.9eV的基态,与波尔的预测相符。

在这样一群人类脑力斗智斗勇的领域,量子力学四个最重要的思想产生了:
\begin{Itemize}
    \item 波尔的互补原理/海森堡测不准原理
    \item 波粒二象性
    \item 用概率解释波函数
    \item 对应原理——在更大的量子数下量子力学与经典力学的结合。
\end{Itemize}

在德布罗意将光的波粒二象性拓展到所有物质的时候且有诸多实验证实之后,有人会思考,既然粒子具有波的特性,那么波上最稳定的驻波位置是否能够保证原子的稳定性?答案是否定的。
驻波存在的情况下我们依然会发现有震荡的产生,这并不利于保证稳定性。
\section[不确定性原理]{\itr{Uncertainty principle}{不确定性原理}}
这时,一个叫做海森堡的人站了出来,他提出了利用矩阵描述量子力学的方法,并进一步提出了不确定性原理。海森堡认为,与其考虑那些无法被测量的神奇的原子轨道,不如考虑其他的由这些状态表现的物理量。他认为所有物理量都应该使用矩阵表示,并提出:
\begin{law}[\itr{Uncertainty principle}{不确定性原理}]
    \[xp-px=i\hbar\]
    \[\Delta x\cdot\Delta p\geq\hbar/2\quad\text{(注)}\footnote{事实上,不确定性原理展示的所谓误差应当为测量的标准差,即$\sigma_x\cdot\sigma_p\geq\hbar/2$}\]
    
    根据海森堡的不确定性原理,我们进一步可以得到:
    \[\Delta x\cdot\Delta p_x\geq\hbar/2\quad\Delta y\cdot\Delta p_y\geq\hbar/2\quad \Delta z\cdot\Delta p_z\geq\hbar/2\]
    以及能量-时间不确定性公式:
    \[\Delta E\cdot\Delta t\geq\hbar/2\]
\end{law}

不确定性原理的含义有两个层面,一是测量某个属性时存在不确定性,二是对于一组有不确定性关系的属性,其标准差的乘积应当不小于某个大于0的常数值。

原子的稳定性通过舍弃轨道的思想,转而使用波的概念进行了解释。为了解释这些定态的存在,我们需要找到具有静态概率密度的时变波函数$\psi(x,t)$。

波通常以三角函数表示,我们有必要对相关思考进行一些了解。我们首先需要了解复平面以及欧拉公式:
\[e^{i\theta} = \cos\theta + i\sin\theta\]

因此,三角函数(及其四则运算)均能够转化为复数形式的运算。这也是薛定谔方程推导的第一个思考:我们已经明白波粒二象性。在“波”这一概念下,我们写出其一般方程:
\[y(x,t) = cos(\omega t - kx)\]

因此我们得到波函数:
\[\Psi(x,t) = \psi_0 e^{i(kx-\omega t)}\]

展开后取实部就是我们一般认为的采用三角函数表示的波的形式。

其共轭表示为:
\[\Psi^* = \psi_0 e^{-i(kx+\omega t)}\]

波函数及其共轭的乘积为实数,这是一个概率密度函数(概率与概率密度是两个不同的概念),它表征了量子世界中粒子出现的概率:
\[P(x) = \Psi\Psi^{*} = |\psi_0|^2\]

波函数具有一些性质:
\begin{Itemize}
    \item 波函数$\Psi$是单值的、连续的;
    \item 波函数利用薛定谔方程,可以简化能量的计算;
    \item 如果粒子存在,那么探寻所有的位置,找到粒子的概率为1。波函数存在以下性质:
    \[\int_{0}^{\infty}P(x)\dif{x} = 1\]
\end{Itemize}

在波函数的概率诠释中,我们无法找到一个粒子在某一时刻的准确量,我们只能够找到其统计量,比如某个物理量的期望值。也就是说能够表述这个物理量的平均值,
以及其方差等在统计学上的概念。类似下面这个例子:
\[\text{位置的期望\mgnote{在专业书籍中,通常用符号$\langle x\rangle$表示}}\quad\bar{x} = \int xP(x)\dif x\]

那么,我们需要找到这么一个可以应用波函数的方程,学界需要一股新的思想潮流。
\section[薛定谔方程]{\itr{Schrödinger Equation and Its Applications}{薛定谔方程及其应用}}
\subsection[薛定谔方程]{\itr{Schrödinger Equation}{薛定谔方程}}
物理学界天才横空出世,提出\itr{Schrödinger Equation}{薛定谔方程}。该方程成为波动力学的重要方程之一,并经受住了时间的考验。我们给出薛定谔方程的形式如下\footnote{少数教师的期末会考到该方程的默写,请注意记忆}:
\begin{law}[\itr{Schrödinger Equation}{薛定谔方程}---\refleaftext{prove10.3}]
    低速自由粒子的一维薛定谔方程:
    \[i\hbar\dfrac{\partial\Psi(x,t)}{\partial t}=[-\dfrac{\hbar^2}{2m}\dfrac{\partial^2}{\partial x^2}+U(x,t)]\Psi(x,t)\qquad\text{(一维)}\]
    或者:
    \[i\hbar\dfrac{\partial}{\partial t}\Psi(x,t)=-\dfrac{\hbar^2}{2m}\nabla^2\Psi(x,t)+U(x,t)\Psi(x,t)\qquad\text{(三维)}\]
    其中$\psi(x,t)$为波函数,$U(x,t)$为势函数。

    当参数不随时间变化的时候,为定态薛定谔方程,定态可以理解为不随时间变化,例如静止的电场。方程简化如下:
    \[\dfrac{\dif^2 \psi(x)}{\dif x^2}+\dfrac{2m}{\hbar^2}[E-U(x)]\psi(x)=0\qquad\text{(一维)}\]
    或者:
    \[-\dfrac{\hbar^2}{2m}\nabla^2\psi(x)+U(x)\psi(x)=E\psi(x)\qquad\text{(三维)}\]

    如果我们定义一些算符,可以将这个方程进行简化的表述。我们定义四个算符:
    \begin{Itemize}
        \item 能量算符:$i\hbar \dfrac{\partial}{\partial t} \cong E$
        \item 动量算符\mgnote{事实上,这一算符是基于波函数概率诠释的基础,求解动量的期望值得到的。光的波动方程形式与波函数一致,亦可类比。}:$-i\hbar \dfrac{\partial}{\partial x} \cong p$
        \item 动能算符:$-\dfrac{\hbar^2}{2m} \dfrac{\partial^2}{\partial x^2} \cong \dfrac{p^2}{2m}$
        \item 哈密顿算符:$-\dfrac{\hbar^2}{2m} \dfrac{\partial^2}{\partial x^2} + U \cong H$
    \end{Itemize}

    因此简化表达为:$E\Psi = H\Psi$,其中满足方程的函数$\Psi$为本征函数,$E$为对应本征值。
\end{law}

后续,狄拉克等人对薛定谔方程进行了相对论下的修正。但本章节,我们更多对于定态,尤其是一维非相对论情况下的定态进行分析。

对于一维自由粒子,其势函数为0;且$\Psi(x,t) = \psi(x) e^{- iEt/\hbar}$\mgnote{后面的指数项被称为时间因子},$E$为能量。此时的薛定谔方程将简化为二阶微分方程:
\[-\dfrac{\hbar^2}{2m}\dfrac{\dif ^2}{\dif{x^2}}\Psi(x) = E\Psi(x)\]

解得最终结果:
\[\Psi(x) = \psi_0 e^{ikx}\qquad k^2 = \dfrac{2mE}{\hbar^2}\]

采用这一理论解决驻波问题有了眉目。在一维空间中,宽度为$L$的无限深势阱前提下的波函数满足:\mgnote{这一结论将会在下一部分详细讲解求解过程}
\[\Psi(x,t) = \sqrt{\dfrac{2}{L}}sin(kx)e^{i\omega t}\]

根据能量和动量的关系:
\[E = \dfrac{p^2}{2m}\quad\Rightarrow\quad \hbar\omega = \dfrac{\hbar^2k^2}{2m}\]

此时电子的概率密度表示为:
\[P(x) = \dfrac{2}{L}\sin^2(kx)\]

明显,这是一个与时间无关的不小于0的常量\mgnote{对于定态的波函数而言,时间因子会在概率的表达式中被消去,这与之定态的情况相符}。其再乘上电子电荷$-e$即为电荷密度(统计学意义上)。
不随时间而改变则意味着原子是稳定的,且电荷密度是稳定的。电荷密度的稳定带来的是静电场,是没有辐射的,因此能量也不会被耗散。综上,困扰科学家的原子稳定性之谜有了合理的解释。

\begin{law}[\itr{Principle of superposition of states}{态叠加原理}]
    对于满足薛定谔方程的解$\Psi$,称为粒子的一个状态。如果多个状态$\Psi_i$都满足是粒子的状态,则其线性组合也是粒子的一个可能的状态,称为\itr{Principle of superposition of states}{态叠加原理}:
    \[\Psi = \sum_i c_i\Psi_i\qquad c_i\text{为复常数}\]
\end{law}

态叠加原理是“波的叠加性”和“波函数完全描述一个体系的量子态”的概括。假设有这样一个粒子,在本征态$\Psi_1$下测量一个力学量可以得到一个确切的测量结果$a_1$,在本征态$\Psi_2$下测量可以得到另一个确切的结果$a_2$。
那么在定态下测量,可以得到一个概率为1的确切结果,且测量后状态不变。而在非定态时,由于叠加态的存在,多个本征态测量结果中的任意一个本征值都可能出现,它们出现的概率由权重$c_i$确定,分别为$\left|c_i^2\right|$($\sum |c_i|^2=1$)。
粒子有多种可能的状态,而测量时单次只能观测到一种状态的结果,这被称为量子态坍缩。事实上,薛定谔的猫阐述的就是这样的理论。

态叠加原理表明,测量一个物理量时,微观世界的行为会由于某种不知名的影响而产生概率化的表现。一些观念认为,某一刻时我们测量的一个值,在波的行为上呈现为一个尖锐的峰,而测量之后它会立刻以波的形式弥散开来。不过无论是否这个解释合理,我们不得不承认,在大多数情况下,
我们对同一个粒子的同一个力学量测量许多次,每一次的测量都是不可预知的,测量的结果种类也是多种多样的。
\subsection[一维无限深势阱]{\itr{Infinitely Deep Potential Well in 1D}{一维无限深势阱}}
我们以一维为例探索薛定谔方程的应用。所谓无限深势阱,是针对势函数$U$而言的。以金属原子为例,其势函数相当复杂,我们希望简化波函数的计算就需要近似。首先,粒子除非受到非常大的外部作用,否则只能在原子内部有限体积内运动,因此外部可以视为势函数无穷大。
而原子内部的势函数可以近似为定值,使用平均势能代替晶格势能,也就是电子是一定区域内的自由电子,不考虑电子间以及电子与晶格离子间的作用。零点势能的选择是任意的,因此可以得到无限深势阱。求解薛定谔方程,重点在于找边界条件。

对于下面所示的势函数(请注意,势函数的不同将导致波函数不同,此处仅为示例):
\begin{singlefigure}[一种一维无限深势阱]{chapter10_一维无限深势力井.png}[0.45]
\end{singlefigure}

我们对于该势阱的势函数的定义为:
\begin{align*}
    U(x) = 
    \begin{cases}
        0 & x\in(0,a)\\
        \infty & x\notin(0,a)
    \end{cases}
\end{align*}
\begin{law}[\itr{An Infinitely Deep Potential Well in 1D}{一维无限深势阱}---\refleaftext{prove10.4}]
    对于上面所述的一维定态无限深势阱,通过薛定谔方程能够解得其波函数为:
    \begin{align*}
        \Psi(x) = 
        \begin{cases}
            \sqrt{\dfrac{2}{a}}\sin(\dfrac{n\pi}{a}x) & x\in(0,a)\\
            0 & x\notin(0,a)
        \end{cases}
    \end{align*}
\end{law}
\subsection[量化能量与检测概率]{\itr{The Quantized Energy and Probability of Detection}{量化能量与检测概率}}
根据之前的结论(部分结论为上述公式推导中得到,若不理解可以参考推导过程),我们现在得到了以下等式:
\begin{align*}
    k = \dfrac{n\pi}{a}\\[1.5em]
    k = \sqrt{\dfrac{2mE}{\hbar^2}}
\end{align*}
我们因此可以得到量化的能量表达式:
\[E = n^2\dfrac{\pi^2\hbar^2}{2ma^2}\]

$n= 1$时为零点能,我们发现等式并不为0。这可以用不确定性原理解释。当能量趋近于0时,则动量(及其变化率)趋近于0,
由此得到位置的测量误差$\Delta x$区域无穷大。但是我们又已知粒子只可能在势函数为0的位置出现,这是一个有限值,由此导出矛盾,因此能量有最低的非0值。

进一步地,我们可以得到概率密度分布,这可以有效说明粒子的稳定性等现象,当然我们知道,概率密度较高处发现粒子的概率较大:
\begin{singlefigure}[波函数与概率分布]{chapter10_检测概率.png}[0.95]
\end{singlefigure}

当然,粒子大多数情况下是振动的,我们可以将其运动简化为一个谐振子。采用经典力学中的机械波,我们有:
\[U(x) = \dfrac{1}{2}kx^2 = \dfrac{1}{2}m\omega^2x^2\qquad \omega = \sqrt{\dfrac{k}{m}}\]

继续使用一维定态薛定谔方程,解得:
\[E_n = (\dfrac{1}{2}+n)\hbar\omega\]

$n=0$时也是零点能。
\subsection[隧穿效应]{\itr{Tunnelling Effect}{隧穿效应}}
之前我们讨论了一维无限深势阱,那么如果势函数(或者说势垒)不是无限大会如何呢?我们考虑一个粒子的波动在沿着一维运动,恰好前方遇到一堵高墙,单凭借粒子自身的动能无法达到墙的最高点,那么所有粒子都会被反射吗?
答案是否定的,根据隧穿效应,这种情况下依然有可能使得部分粒子越过这堵高墙跨到另一侧,就好比在你马上撞到墙的时候你发现身体穿过了墙,这就是隧穿效应的直观表现。隧穿效应就是微观版的崂山道士。

考虑如下的势垒以及其势函数:
\begin{singlefigure}[隧穿效应示意图]{chapter10_隧穿效应.png}[0.7]
\end{singlefigure}

\begin{align*}
        U(x) = 
        \begin{cases}
            U_0 & x\in[0,a]\\
            0 & x\notin[0,a]
        \end{cases}
    \end{align*}
\begin{law}[\itr{Tunnelling Effect}{隧穿效应}---\refleaftext{prove10.5}]
    对于上述的隧穿效应,其隧穿概率(穿过势垒的粒子与入射的粒子的比值)为:
    \[T = e^{-2ka} \qquad k = \dfrac{\sqrt{2m(U_0-E)}}{\hbar}\]
\end{law}

类似隧穿效应,如果粒子在运动的时候遇到一个势垒较低的坑,粒子能够完全跳过去吗?答案也是否定的。粒子完全穿透的情况只有势垒水平恒定时才会出现。

因为隧穿效应的伟大发现,我们能够进一步探微,量子隧道显微镜因此得以发明和应用,同时隧穿效应也成功解释了在原子核巨大吸引力的情况下放射性元素依然能够$\alpha$衰变释放核子的原因。
\section[氢原子]{\itr{Hydrogen Atom}{氢原子}}
由于氢原子假想为一个球体,我们可以用球坐标轴替换其薛定谔方程,其中三个参数可分离。其中涉及的计算过于复杂,有兴趣的同学可以参考详细的量子力学的书籍。我们这里只描述得到的一些概念:
\begin{law}[\itr{Three quantum numbers}{三个量子数}]
    我们给出一些符号及其概念,他们与普通化学课程的定义一致:
    \begin{Itemize}
        \item \itr{Principle quantum number}{主量子数}:$n\quad n = 1,2,3,...$,表征能量量子化,这里的符号与前面的公式中的$n$相同。
        
        $E_n = -\dfrac{1}{n^2}\dfrac{me^4}{8\varepsilon_0^2h^2} = -\dfrac{13.6eV}{n^2}\text{(数值仅对于氢原子)}$。
        \item \itr{Orbital Angular momentum quantum number}{轨道角动量量子数}:$l\quad l = 0,1,2,...,n-1 \quad\text{or}\quad s,p,d,f,...$,它表征了电子云的空间形态。
        
        轨道角动量$L = \sqrt{l(l+1)}\hbar$
        \item \itr{Magnetic quantum number}{磁量子数}:$m_l\quad m_l = 0,\pm 1,\pm 2,...,\pm l$。
        
        外加磁场时角动量在磁场方向的分量$L_z = m_l\hbar$,它也表明轨道角动量的方向是量子化的。
        \item \itr{Spin quantum number}{自旋量子数}:$s=\dfrac{1}{2}$
        
        \itr{spin angular momentum}{自旋角动量}$S=\sqrt{s(s+1)}\hbar$
        \item \itr{Spin magnetic quantum number}{自旋磁量子数}:根据自旋与外加磁场的关系得到$m_s = \pm \dfrac{1}{2}$
        
        如果外加磁场方向与自旋角动量方向平行,则有$L_{sz}=S_z = m_s\hbar =\pm \dfrac{1}{2}\hbar$
    \end{Itemize}
    在上述几个定义中,$L_z$是$L$在$z$轴上的分量。轨道角动量与其沿$z$轴旋转的角度依然符合不确定性原理。

    另外,在球坐标求解氢原子波函数的过程中,我们定义径向概率密度:
    \[P(r) =r^2|R_{nl}(r)|^2\]
    其中$R_{nl}(r)$是球坐标下三个变量中关于半径距离的函数,径向概率密度函数与波函数的概率密度函数含义一致,区别在于径向概率是在半径方向上积分得到,即$P(r)\dif r$
\end{law}

主量子数标定能级;轨道量子数标定电子在哪个轨道上;磁量子数是指在外加磁场的情况下,原子光谱出现的单个条带分裂成多个条带(也就是磁场作用下出现同一能级下的电子能量变化,
由单个$E_m$能量产生$E_m+\Delta E$、$E_m-\Delta E$等多种能量的条带)。这被称为\itr{Zeeman effect}{塞曼效应}。例如下图所示的原子光谱图:
\begin{singlefigure}[塞曼效应]{chapter10_条带分裂.png}[0.95]
\end{singlefigure}

而自旋角动量与自旋量子数并非由薛定谔方程直接分析氢原子得到。当时物理学家推测的理论认为,没有磁场作用下金属发射的电子经过单狭缝后会呈现单一条带,而加上磁场后应该是奇数个条带(因为磁量子数一共奇数个)。
但1921年,德国施特恩(Otto Stern,1888—1969)和格拉赫(Walter Gerlach,1889—1979)在实验中将碱金属原子束经过一不均匀磁场射到屏幕上时,
发现射线束分裂成两束,并向不同方向偏转。这暗示人们,电子除了有轨道运动外,还有自旋运动,是自旋磁矩顺着或逆着磁场方向取向的结果\footnote{在量子力学的解释中,电子的自旋是一个完全随机的方向,它只会在我们测量时在我们所测量的方向上进行选择性表现}。

上面的推导与发现,也为元素周期表的排布带来一些帮助。
\section[X射线、激光与能带]{\itr{X-rays, Lasers, and Bandgap}{X射线、激光与能带}}
若精力有限,下面的内容仅满足了解性质。观察近些年的考题较少涉及该部分知识。

谈了这么多理论,量子物理的应用有哪些呢?我们下面主要讲解三个应用。
\subsection[X射线]{\itr{X-rays}{X射线}}
1895年,伦琴发现了X射线(又被称为阴极射线,因为与电子有关)。这是一种高频的电磁波。

关于X射线的产生,常用的方法是利用高速运动的电子击打金属靶材,通过碰撞产生的能量损耗释放电磁波。这通常在一个真空环境中进行。X射线可以分为\itr{The continuous X-Ray spectrum}{连续X射线光谱}以及\itr{The characteristic X-Ray spectrum}{特征X射线光谱}:

\begin{singlefigure}[连续X射线光谱]{chapter10_连续X射线光谱.png}[0.5]
    由电子碰撞原子后的能量损失形成。这一能量变化是允许连续的。
\end{singlefigure}
\begin{singlefigure}[特征X射线光谱]{chapter10_特征X射线光谱.png}[0.9]
    由电子碰撞原子后原子内部结构变化形成。这一能量变化释放的电磁波长是特定的。
\end{singlefigure}

我们假设经过电场加速后的电子能量为$K_0$。对于连续光谱,由于这一碰撞是可以造成不大于电子能量的任意能量损耗,而能量损耗以电磁波的形式释放,由此可见光谱是连续的。

而如果能量足够大,以至于部分电子直接将原子内的电子击出原子范围内,会导致一瞬间内部缺少一个电子,这时处于较高能级的电子将会跃迁到较低能级,并释放特定波长的电磁波。这是则为特征光谱。跃迁的电子可能来自于不同的能级,因此辐射的电磁波也会产生不同波长。

以上的内容我们可以用下图所示的图像表述:
\begin{singlefigure}[X射线光谱示意图]{chapter10_X射线光谱.png}[0.54]
\end{singlefigure}

对于不同的金属靶材,其图像会有具体数值上的差异。观察图像我们可以清楚地找到连续光谱和特征光谱。特征光谱的产生需要加速电压足够大才能够呈现。另外我们也发现这里的X射线光谱存在最小的临界值,且这个值与加速电压有关,这是为什么?
很容易理解,我们的电子的能量是有上限的。也就是说电子的能量损耗不会超过其本身能量的上限,从而存在最小波长的X射线:
\[\lambda_{min}=\dfrac{hc}{K_0}\]
\subsection[激光]{\itr{Lasers}{激光}}
激光,全称受激辐射光放大。顾名思义,激光的产生需要受到外部激励,同时还需要进行一定的光放大。1917年,爱因斯坦从理论上指出:除自发辐射外,处于高能级上的粒子还可以另一方式跃迁到较低能级。他指出当频率为$\nu=(E_2-E_1)/h$的光子入射时,也会引发粒子以一定的概率,
迅速地从能级$E_2$跃迁到能级$E_1$,同时辐射两个与外来光子频率、相位、偏振态以及传播方向都相同的光子,这个过程称为受激辐射。
\begin{singlefigure}[激光的产生]{chapter10_受激辐射.png}[0.55]
\end{singlefigure}

受激辐射的存在一定产生激光吗?答案是否定的。很明显,我们的世界中没有自然存在的激光。在具体解释激光的产生机制之前,我们需要首先了解到辐射的发射和吸收是与处于相应能级的粒子数有关的。

现在假定两个能级$E_1<E_2$,两个能级在某一时刻都会有一定的处于相应能级的粒子数,设为$N_1$和$N_2$。对于自发辐射过程(上图第二个过程),我们有自发辐射常数$A_{21}$,表示从能级$E_2$跃迁到能级$E_1$的概率,或者称为单位时间内跃迁的粒子数占高能级粒子数的比例。
因此我们有:
\[(\dfrac{\dif N_{21}}{\dif t})_{SA} = A_{21}N_2\]

而对于受激吸收以及受激辐射过程,也是类似地存在上述表达式。但同时,这一概率系数是与外来光场的单色能量密度$\rho(v)$成正比。因此我们有:
\[(\dfrac{\dif N_{12}}{\dif t})_{SE} = B_{12}\rho(v)N_1\qquad (\dfrac{\dif N_{21}}{\dif t})_{STE} = B_{21}\rho(v)N_2\]
\[B_{21}=B_{12}\]

其中$B_{12}$、$B_{21}$称为受激辐射跃迁爱因斯坦系数。

对于一般的粒子,它倾向于能量较低的能级。热力学平衡状态下遵循玻尔兹曼定律,有$N_i=Ce^{-E_i/kT}$。我们可以明显地发现,能量越高的能级其粒子数是越少的,也就会导致受激辐射的概率远小于受激吸收,很容易导致光子大多被吸收而很少有释放。为了产生激光,我们必须要让三个同时进行的过程中的受激辐射过程占据主要优势。
怎么办?我们应该想到要尽量使得高能级的粒子数更多,这便是激光产生过程中的重要步骤——粒子数反转(也称为布居数反转/集居数反转)。
\begin{singlefigure}[粒子数反转]{chapter10_粒子数反转.png}[0.55]
\end{singlefigure}

粒子数反转的主要过程如上图所示。粒子数反转首先不能是二能级系统。在三能级或者四能级系统中,一般需要通过泵浦将低能级的粒子提升到更高的能级,在高能级的状态下是不稳定的,会很快自发辐射并跃迁到较低的能级。
而此时的中间能级处于亚稳态,粒子保持稳定的时间会比高能级长很多(当然不可否认为了热平衡这些粒子还是要回到基态)。我们只要能够给予系统足够的能量,使得泵浦到高能级的粒子速率超过粒子跃迁回基态的速率即可。

三能级系统的基态可能有些时候基态粒子数还是太多了,一些激光发生器采用了四能级系统,将$E_2$、$E_3$作为激光产生的上下能级,通过将基态的粒子泵浦到$E_4$并自发辐射跃迁到$E_3$。由于$E_2$粒子数比基态少很多,因此这种粒子数反转更容易实现。

粒子数反转使得受激辐射成为主导过程,多次的受激辐射产生了越来越多的频率相位等完全一致的光子。但这些光子并不够,我们希望他们能够维持一个稳定的光强并发射出来。这就需要使用到\itr{Optical  Resonator}{光学谐振器}。
\begin{singlefigure}[光学谐振器]{chapter10_光学谐振腔.png}[0.85]
\end{singlefigure}

外部的能量输入到光学谐振腔中,引发谐振腔内部的物质的受激辐射。在光学谐振器中,一侧为全反射镜,一侧为部分反射镜。多次的反射提供了足够的受激辐射时间和距离,产生了更多我们想要的光子,同时谐振腔的长度要求满足光的驻波条件,以免相互抵消,即:
\[2nL = q\lambda\]

在这个过程中,方向平行于轴线的光被放大增强,偏离轴线的则通过四周被释放出去。多次反射积累的足够光子将通过部分反射镜发射。同时,光学谐振腔也通过布儒斯特窗这一元件,通过设定入射光角度为布儒斯特角,实现对光的偏振性选择。

激光具有以下特征:
\begin{Itemize}
    \item 高度的单色性:激光的波长范围极小,从而呈现单色激光。
    \item 高度的相干性:易于实现干涉。时间上的相干性:相干长度很大,可达$2\times 10^7$km;空间的相干性:发射角度极小,约为毫弧度量级。
    \item 高度的方向性:散射角度极小,约为毫弧度。
    \item 极高的光亮度:可以清晰聚焦,实现激光的小范围内的极高能量密度。
\end{Itemize}
\subsection[能带]{\itr{Bandgap}{能带}}
我们知道,对于单个独立的原子,其核外电子通常处于不同的能级。单个能级的能量是一个定值,画图表示为一条线的范围。但是当多个原子相互靠近形成金属固体时,单一的能级的范围将会扩大,称为一段连续的能量范围,称为能带。
\begin{singlefigure}[能带]{chapter10_能带示意图.png}[0.85]
\end{singlefigure}

为什么单一的能级会变成能带?我们可以用不确定性原理解释。首先介绍紧束缚近似:从孤立原子的核外电子状态出发,将晶体视为原子间相互靠近的结果。根据量子理论,此时核外电子云会靠近并连接在一起,也就是说电子是所有原子核所共有的,
可以在不同原子核外运动。因此,单个原子所拥有该电子的时间是有限的。根据不确定性原理,$\Delta t$是一个有限值,因此$\Delta E$不能为0,故而能级成为条带的样子。根据图像可见,距离足够远时能量范围很小,而距离越近能量范围越大,
甚至于部分不同能级的条带出现交叉现象。

在能带之间的不可能出现电子的能量范围,称为禁带(或者称为带隙)。禁带宽度$E_g$与原子有关,对于导体而言,禁带一般较小甚至为0(能带之间重叠),使得电子能够轻易跃迁,从而易于导电;而对于绝缘体,禁带宽度一般较大,电子难以跃迁到不同的能级对应的
能带,从而难以导电;而对于半导体,则介于两者之间。

能带概念充分应用于半导体元件中,我们常见的半导体是硅等元素。中学我们已经学过,价电子是表征原子稳定性的重要依据。单质硅中最外层电子形成稳定的共价键,纯度极高的硅单质称为本征硅,它具有绝缘体的导电特性。一般情况下,极少数的电子会挣脱
束缚成为自由电子,并留下空穴(可视为带正电的粒子),即电子-空穴对。

单质中电子-空穴对很少,浓度极低,因此难以导电,从微观能级的角度理解为低能级的价带被填充满,但高能级的导带是空的没有电子。为了增强半导体材料的导电性,我们选择在大量硅原子中掺杂少量价电子数不为4的其他原子,具体分为两种情况:
\begin{Itemize}
    \item N型半导体:掺杂诸如磷、砷等+5价元素,使得材料中出现额外的无法被固定的共价键束缚的电子;
    \item P型半导体:掺杂诸如硼、镓等+3价元素,使得材料中部分共价电子对中缺少电子,产生空穴。
\end{Itemize}

掺杂粒子如何影响导电性呢?这是由于新的原子带来了杂质能级所致。对于N型半导体,额外的电子被离子吸引不能随意移动,能量低于导带,但又没有像共价键那么强,因此很容易就可以跃迁到导带。这一处于带隙中间的能级为施主能级。而P型半导体正好相反,
拥有处于带隙中且离价带很近的能级,被称为供主能级。激发的电子可以很容易进入供主能级中从而导电。当然,不同的原子导致的杂质能级是不同的,带来的导电性能也是不同的。同时这些能级会由于温度变化有较为明显的变化,这也是半导体材料的性质对于温度敏感的原因之一。
\begin{singlefigure}[施主能级和供主能级]{chapter10_供主能级和施主能级.png}[0.85]
    (a)为施主能级,(b)为供主能级\footnote{图片截取自知乎:\url{https://zhuanlan.zhihu.com/p/584057872}}
\end{singlefigure}

在N型和P型半导体中,掺杂的原子是少数,它们提高了单一硅单质的导电性。此时,若将P型和N型半导体接触,由于电子和空穴的浓度差异,接触面附近的电子将会扩散到空穴中,使得中间一段区域内的原子共价键为完整的4个共价键,形成一定的区域称为PN结。
PN结的产生导致了整体物质中间形成了一个内建电场,方向由N型半导体指向P型半导体。

\begin{minipage}{0.45\textwidth}
    \begin{figure}[H]
        \centering
        \includegraphics[width = 0.8\textwidth]{../figures/figure10/chapter10_PN结正偏.png}
        \caption{PN结正偏}
    \end{figure}
\end{minipage}
\begin{minipage}{0.45\textwidth}
    \begin{figure}[H]
        \centering
        \includegraphics[width = 0.8\textwidth]{../figures/figure10/chapter10_PN结反偏.png}
        \caption{PN结反偏}
    \end{figure}
\end{minipage}
\vspace{2mm}

PN结有正反偏,这与二极管的正向导通反向截止、导通电压等密切相关。P型半导体连接电源正极时,空穴和电子继续向对侧的方向扩散,使得PN结宽度变窄,达到一定程度后顺利导通;而反偏则相反,空穴和电子相互远离
使得PN结宽度变大,电子更加难以跃迁,从而无法导电。当反向电压不断增大,会最终产生击穿。