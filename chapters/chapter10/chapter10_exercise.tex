\section{课后习题:量子物理}
我认为有必要声明,量子物理的应试应当以熟悉各种概念以及薛定谔方程及其一维定态的两种模型的求解为主,基本与作业题匹配。
但由于个别班级的极个别期末关于波函数的题目难度较大,因此本章节习题多来自专业书籍且集中于波函数的知识点,难度把控较为困难,仅供参考。

关于推导和计算,请多参考相关定理的证明。关于光电效应、原子结构等的题目,请参考教师布置的课后作业题。
\begin{example}[Probability of Detection ---\refleaftext{solution10.1}]
    A particle is confined in a one-dimensional infinite square well potential of width $a$. Its wave function is given by:
    \[\psi_n(x) = \sqrt{\dfrac{2}{a}} \sin(\dfrac{n\pi x}{a}) (0 < x < a)\]
    
    If the particle is in the $n = 1$ state (ground state), what is the probability of finding it in the region between $x = 0$ and $x = a/4$?

    (Hint: $\int \sin^2(x) \dif x = \dfrac{x}{2} - \dfrac{1}{4}\sin(2x) + C$)
\end{example}

\begin{example}[An Infinitely Deep Potential Well in 1D ---\refleaftext{solution10.2}]
    A particle is in a one-dimensional infinite deep potential well with a width of $a$. At a certain moment, 
    the two walls of the well suddenly move in opposite directions, causing the width of the well to change to $2a$.
    The potential Wells before and after the change are symmetrical about $x=0$
    The particle wave function has no time to change. What is the probability that the particle remains in the 
    ground state for the widened infinite deep potential well?
    
    (Hint 1: Principle of superposition of states $\psi_{1,1} = \sum c_i\psi_{2,i}\quad P_1 = |c_1|^2$)
    
    (Hint 2: $\int \cos(2x)\cos(x)\dif x = \dfrac{1}{6}(3\sin (x) + \sin (3x)) + C$ )
\end{example}

\begin{example}[Tunnelling Effect ---\refleaftext{solution10.3}]
    The \(\delta(x)\)  is a very important function in engineering technology, and it is defined as:
    \begin{align*}
        \delta(x) =
        \begin{cases}
           \infty \quad &x = 0\\
           0 \quad &x \neq 0 
        \end{cases}
        \qquad \int_{-\infty}^{\infty} \delta(x) \dif x = \int_{0^-}^{0^+}\delta(x) \dif x= 1
    \end{align*}

    There is a one-dimensional \(\delta\) barrier at $x=0$:
    \[U(x) = A \delta(x) \qquad (A>0)\]
    
    When a particle with energy $E$ is incident from the left, find the transmission coefficient.
    (Hint: In this question,\(\dfrac{\dif ^2\psi(x)}{\dif x^2}\) is divergent. Consider integrating it in $\left[-\varepsilon,\varepsilon\right]\quad \varepsilon\rightarrow 0$)
\end{example}

\begin{example}[Schrödinger Equation ---\refleaftext{solution10.3}]
    Consider a particle of mass $m$ confined in a 2D rectangular infinite square well.The potential energy function is:
    \[
    U(x,y) = \begin{cases} 
        0 & 0 < x < L_x , 0 < y < L_y \\
        \infty & \text{boundary and exterior}
        \end{cases}
    \]
    The wave function is separable: $\psi(x,y) = X(x)Y(y)$, with total energy $E = E_x + E_y$.

    (a) Calculate the normalized wave function $\psi_{n_x,n_y}(x,y)$, and prove the particle energy is:
    \[E_{n_x,n_y} = \dfrac{\pi^2 \hbar^2}{2m} \left( \dfrac{n_x^2}{L_x^2} + \dfrac{n_y^2}{L_y^2} \right)\]

    (b)Given $L_y = 2L_x = 2L$, calculate the second excited state (provide quantum numbers $(n_x, n_y)$ and energy value).

    (c)For the Schrödinger Equation $\hat{H}\psi=E\psi$, the set of all eigenvalues $E$ of operator $\hat{H}$ is called the \itr{spectrum}{谱}. 
    When two or more linearly independent eigenfunctions share the same eigenvalue, it is called degeneracy of spectra, and the number of such eigenfunctions is the \itr{degeneracy degree}{简并度}.
    
    Given $L_y = \alpha L_x$ ($\alpha > 1$), if the degeneracy degree of the second excited state is 2, find the $\alpha$ satisfying this condition.
\end{example}