\chapter[量子力学]{\itr{Quantum Mechanics}{量子力学}}
\begin{prove}[The Compton Effect]
\begin{minipage}{0.45\textwidth}
    \begin{singlefigure}[散射前图示]{chapter10_康普顿散射1.png}[0.99]
    \end{singlefigure}
\end{minipage}
\begin{minipage}{0.45\textwidth}
    \begin{singlefigure}[散射后图示]{chapter10_康普顿散射2.png}[0.99]
    \end{singlefigure}
\end{minipage}
\vspace{2mm}

以上图为例分析。假设粒子最初为静止状态。

对于光而言,其能量为$E=h\nu$,动量为$p=\dfrac{h\nu}{c}$。

对于粒子而言,其动能为$E=m_0 c^2$(质能公式),动量为$p=mv$。

由于碰撞后速度变化导致的相对论条件下的质量变化,因此电子质量不守恒。我们得到下面这几个公式:
\begin{align}
    m_0 c^2 + h\nu &= mc^2 + h\nu'\qquad \text{能量守恒}\tag{1}\\
    X方向:\quad \dfrac{h\nu}{c} &= \dfrac{h\nu'}{c}\cos\varphi + mv\cos\theta\tag{2}\\
    Y方向:\quad 0 &= \dfrac{h\nu'}{c}\sin\varphi - mv\sin\theta\tag{3}
\end{align}

先处理动量公式,我们整理出含$mv$的部分并单独放到公式一侧,平方后相加,使得三角函数被消掉,得到:
\[(\dfrac{h\nu}{c})^2 + (\dfrac{h\nu'}{c})^2 - \dfrac{2h^2\nu\nu'}{c^2}\cos\varphi = m^2 v^2\]

再处理能量守恒公式,对两边平方:
\[h^2(\nu - \nu')^2 + m_0^2c^4 + 2m_0c^2h(\nu - \nu') = m^2c^4\]

对上式使用相对论质量公式,采取原公式直接平方的方法,得到:
\[m^2 = \dfrac{m_0^2c^2}{c^2 - v^2}\]

将其代入我们整理的两个公式中,消去$m^2$,并将$m_0c^4$移到另一侧与$m^2c^4$相减,约掉后发现等式右边一样,因此得到:
\[h(\nu^2 + \nu'^2 - 2\nu\nu'\cos\varphi) = h(\nu - \nu')^2 + 2m_0c^2(\nu - \nu')\]

即:
\[h\nu\nu'(1-\cos\varphi) = m_0c^2(\nu - \nu')\]

我们希望求解波长差,观察发现:
\[\dfrac{\Delta \lambda}{c} = \dfrac{1}{\nu'} - \dfrac{1}{\nu} = \dfrac{\nu - \nu'}{\nu\nu'} = \dfrac{h}{m_0c^2}(1-\cos\varphi)\]

最终得到:
\[\Delta \lambda = \dfrac{h}{m_0c}(1-\cos\varphi)\]
\end{prove}
\begin{prove}[Quantized Orbits and Energy]
    在波尔的氢原子模型下应用经典力学以及原子角动量的结论,即:
    \begin{align*}
        \dfrac{1}{4\pi\varepsilon_0}\dfrac{e^2}{r^2} &= m\dfrac{v^2}{r}\\
        L = mv_nr_n &= n\hbar
    \end{align*}
    整理以上两个公式,我们知道$n$不同则轨道半径也不同,因此可得:
    \[\dfrac{1}{4\pi\varepsilon_0}\dfrac{e^2}{r_n^2} = m\dfrac{v_n^2}{r_n}\quad\Rightarrow\quad r_n = n^2 \dfrac{\varepsilon_0 h^2}{\pi me^2}\]

    另外可以得到能量为:
    \begin{align*}
        E_n &= \dfrac{1}{2}mv_n^2 - \dfrac{1}{4\pi\varepsilon_0}\dfrac{e^2}{r_n} = \dfrac{1}{8\pi\varepsilon_0}\dfrac{e^2}{r_n} - \dfrac{1}{4\pi\varepsilon_0}\dfrac{e^2}{r_n} = -\dfrac{1}{8\pi\varepsilon_0}\dfrac{e^2}{r_n} = -\dfrac{1}{n^2}\dfrac{me^4}{8\varepsilon_0^2 h^2} \circ
    \end{align*}

    其他结论可以因此迅速导出,此处不再赘述,读者可以尝试使用上面的推导结论理解。
\end{prove}
\begin{prove}[Schrödinger Equation]
    首先,我们知道能量和动量的关系式:
    \[E=\dfrac{p^2}{2m}\]

    由德布罗意关系,$p = k\hbar$、$E = \hbar \omega$\mgnote{这个可以根据德布罗意波的结论,结合机械波中各个变量的关系得到}
    因此我们替换原本波函数的符号,并将其乘到等式两边,得到:
    \[E\psi_0 e^{i(px - Et)/\hbar} = \dfrac{p^2}{2m}\psi_0 e^{i(px - Et)/\hbar}\]

    其中波函数表示为$\Psi = \psi_0 e^{i(px - Et)/\hbar}$,观察得到:
    \[i\hbar \dfrac{\partial \Psi(x, t)}{\partial t} = -\dfrac{\hbar^2}{2m} \dfrac{\partial^2 \Psi(x, t)}{\partial x^2}\]

    进一步,考虑能量包含外加势场$U(x,t)$,即$E = \dfrac{p^2}{2m} + U(x,t)$,则:
    \[i\hbar \dfrac{\partial \Psi(x, t)}{\partial t} = \left[-\dfrac{\hbar^2}{2m}\dfrac{\partial^2}{\partial x^2} + U(x,t)\right] \Psi(x, t)\]

    当定态时,波函数可以简化为$\Phi = \psi e^{- iEt/\hbar}$,则:
    \[\dfrac{\partial^2 \Psi}{\partial x^2} = \dfrac{\partial^2 \psi}{\partial x^2} e^{-iEt/\hbar}\qquad \dfrac{\partial \Psi}{\partial t} = -\dfrac{iE}{\hbar}\psi e^{-iEt/\hbar}\]

    代回原式并整理,得到:
    \[\dfrac{\dif^2 \psi(x)}{\dif x^2}+\dfrac{2m}{\hbar^2}[E-U(x)]\psi(x)=0\]

    对于三维空间的情况,把对于位置的二阶导数$\dfrac{\partial^2}{\partial x^2}$换成$\nabla^2$即可。
\end{prove}
\begin{prove}[An Infinitely Deep Potential Well in 1D]
    \begin{singlefigure}[一种一维无限深势力井]{chapter10_一维无限深势力井.png}[0.6]
    \end{singlefigure}
    
    我们对于该势力井的势函数的定义为:
    \[U(x) = 
    \begin{cases}
        0 & x\in(0,a)\\
        \infty & x\notin[0,a]
    \end{cases}\]
    在势函数无限大的位置,波函数为0,因为如此大的能量并不存在,以至于粒子不会在这些地方出现。我们只考虑$0<x<a$的部分,这里使用薛定谔方程。类似与一维定态自由粒子的结论,我们有:
    \[\Psi(x) = A\sin(kx) + B\cos(kx)\qquad k^2 = \dfrac{2mE}{\hbar^2}\]

    根据无限深势力井,有如下等式存在:
    \begin{align*}
        \Psi(0) &= B = 0\\
        \Psi(a) &= A\sin(ka) + B\cos(ka) = 0\\
    \end{align*} 

    因此$A\sin(ka) = 0 = \sin(ka)$。因此我们有$ka = n\pi$。原本的波函数因此可以写作:
    \[\Psi(x) = A\sin(\dfrac{n\pi}{a}x)\qquad n = 1,2,3,\cdots\]

    应用波函数的概率性质,我们可以最终求出系数$A$:
    \[\int_{0}^{\infty}\Psi(x)\Psi(x)^{*}\dif{x} = A^2\int_{0}^{a}sin^2(\dfrac{n\pi}{a}x)\dif{x} = A^2\dfrac{a}{2} = 1\quad\Rightarrow\quad A = \sqrt{\dfrac{2}{a}}\]

    并得到最终的波函数表达式:
    \begin{align*}
        \Psi(x) = 
        \begin{cases}
            \sqrt{\dfrac{2}{a}}\sin(\dfrac{n\pi}{a}x) & x\in(0,a)\\
            0 & x\notin(0,a)
        \end{cases}
    \end{align*}
\end{prove}
\begin{prove}[Tunnelling Effect]
    \begin{singlefigure}[隧穿效应示意图]{chapter10_隧穿效应.png}[0.6]
    \end{singlefigure}
    我们依然考虑一维定态薛定谔方程:
    \[\dfrac{\dif^2 \psi(x)}{\dif{x}^2} + \dfrac{2m}{\hbar^2} [ E - U(x) ] \psi(x) = 0\]
    其中$\dfrac{2m}{\hbar^2}[E-U(x)]$为分段函数,每一段内是定值。因此方程可以简化为二阶微分方程:
    \[\dfrac{\dif^2 \psi(x)}{\dif{x}^2} + k'^2 \psi(x) = 0\]
    根据微分方程,可以得到$(-\infty,0)$、$[0,a]$、$(a,\infty)$的波函数表达式可以分别写作:
    \begin{align*}
        \psi(x) = 
        \begin{cases}
            A_1 e^{i\alpha x}+A_2 e^{-i\alpha x}& x\in(-\infty,0)\\
            B_1 e^{i\beta x} + B_2 e^{-i\beta x}& x\in[0,a]\\
            C_1 e^{i\gamma x}+C_2 e^{-i\gamma x}& x\in(a,\infty)
        \end{cases}
    \end{align*}
    其中$\alpha=\gamma = \dfrac{\sqrt{2mE}}{\hbar}$,$\beta = \dfrac{\sqrt{2m(E-U_0)}}{\hbar}$。

    同时,根据波动方程的物理含义,正指数部分代表函数向$x$轴正向传播,负指数部分代表向$x$轴负方向传播。在上述三个区间,前两个区间都有原有正向传播的波动以及遇到高势垒的反射波动,
    但第三个区间没有反射波(该区间的波动仅来自于穿过势垒的波而没有反射波)。因此$C_2 = 0$。

    结合波函数的单值连续的性质,我们有以下等式存在:
    \begin{align*}
        \psi_1(0) &= \psi_2(0) \\
        \dfrac{\dif \psi_1(0)}{\dif{x}} &= \dfrac{\dif \psi_2(0)}{\dif{x}}\\
        \psi_2(a) &= \psi_3(a)\\
        \dfrac{\dif \psi_2(a)}{\dif{x}} &= \dfrac{\dif \psi_3(a)}{\dif{x}}\\
    \end{align*}
    
    隧穿概率的定义为$T = \dfrac{|C_1|^2}{|A_1|^2}$,因此有:
    \[T\propto  e^{-2ka} \qquad k = \dfrac{1}{\hbar} \sqrt{2m(U_0 - E)}\]
    
    尽管这里只是正比关系,但明显$U_0 = 0$时这个概率应该为1,因此正比关系可以直接写作等号(在概率意义下)。
\end{prove}